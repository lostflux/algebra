\begin{problem}{(\textsf{sorta DF 3.1.40--41})}

  \noindent
  Let $G$ be a group and $N \trianglelefteq G$ be a normal subgroup.  
  \begin{enumalph}
    \item Show that if $G$ is abelian, then $G/N$ is abelian.
    \begin{Answer}
      We know each element of $G/N$ is a coset $gN$ for some $g \in G$.
      Let's take two such elements, $xN$ and $yN$.
      Then we can see that $xN \cdot yN = yN \cdot xN$, and $G/N$ is abelian:

      \begin{align*}
        xN \cdot yN &= xy \cdot N \\
        &= yx \cdot N\qquad \zaff{\text{(Since $G$ is Abelian)}} \\
        &= yN \cdot xN
      \end{align*}
    \end{Answer}
    \item Show that $G/N$ is abelian if and only if $aba^{-1}b^{-1} \in N$ for all $a,b
    \in G$.  An element of $G$ of the form $aba^{-1}b^{-1}$
    is called a \emph{commutator}.
    \begin{Answer}
      If $G/N$ is abelian, then $G$ is abelian.
      Let's take two elements $a,b \in G$.

      Suppose $aba^{-1}b^{-1} \not \in N$.
      Then:

      \begin{align*}
        aba^{-1}b^{-1} &= a \cdot (ba^{-1}) \cdot b^{-1} \\
        &= a \cdot (a^{-1}b) \cdot b^{-1} \qquad \zaff{\text{(Since $G$ is Abelian)}} \\
        &= aa^{-1} \cdot bb^{-1} \\
        &= 1 \cdot 1 \\
        &= 1 \in N
      \end{align*}

      This clearly contradicts the assumption that $aba^{-1}b^{-1} \not \in N$.
      Therefore, if $G/N$ is abelian, then $aba^{-1}b^{-1} \in N$ for all $a,b \in G$.
    \end{Answer} 
    \item Let $H \colonequals \langle aba^{-1}b^{-1} : a,b \in G \rangle$ be the 
    subgroup of $G$ generated by commutators, called the \emph{commutator subgroup} of 
    $G$.  Show that $H \trianglelefteq G$ is a normal subgroup of $G$ and that $G/H$ is
    abelian.  \emph{[Hint: it is enough to check that the conjugate of a commutator is 
    a commutator.]}
    \begin{Answer}
      Let $h \in H$.
      Then $h = aba^{-1}b^{-1}$ for some $a,b \in G$.
      Let $g \in G$.
      Then:

      \begin{align*}
        ghg^{-1} &= g \cdot h \cdot g^{-1} \\
        &= (h \cdot h^{-1}) \cdot g \cdot h \cdot g^{-1} \\
        &= h \cdot (h^{-1}ghg^{-1}) \\
      \end{align*}

      It is straightforward to see that $h^{-1}ghg^{-1}$ is a commutator.
      Since we took $h$ from the commutator subgroup,
      the product of $h$ and $h^{-1}ghg^{-1}$
      must also be in the commutator subgroup.
      Therefore, $ghg^{-1} \in H$ for all $h \in H$, $g \in G$, and $G/H$ is abelian.
    \end{Answer}
  \end{enumalph}
\end{problem}

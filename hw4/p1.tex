\begin{problem}
  Let $H \colonequals \{\epsilon,(1\ 2)(3\ 4),(1\ 3)(2\ 4),(1\ 4)(2\ 3)\} \subset 
  S_4$. 
  \begin{enumalph}
    \item Show that $H$ is a subgroup of $S_4$ and that $H$ is isomorphic to
    $D_4 = \{1,r,s,sr\}$.
    \begin{Answer}
      Let's name the components of $H$, such that
      $a = \epsilon$, $b = (1\ 2)(3\ 4)$, $c = (1\ 3)(2\ 4)$, and $d = (1\ 4)(2\ 3)$.
      For $H$ to be a subgroup, it needs to:
      \begin{enumalph}
        \item \textbf{Contain the identity element.}
        
          \noindent
          This is trivial to prove, since $\epsilon$ is in $H$.
        \item \textbf{Be closed under inversion.}
          
          \noindent
          All members of $H$ aside from the identity have cycles of order $2$.
          This makes each element its own inverse, since applying
          a two-cycle twice is the identity.
        \item \textbf{Be closed under composition.}
        
          \noindent
          \begin{align*}
            bc &= (1\ 2)(3\ 4)\ \circ\ (1\ 3)(2\ 4) = (1\ 4)(2\ 3) = d \\
            bd &= (1\ 2)(3\ 4)\ \circ\ (1\ 4)(2\ 3) = (1\ 3)(2\ 4) = c \\
            cb &= (1\ 3)(2\ 4)\ \circ\ (1\ 2)(3\ 4) = (1\ 4)(2\ 3) = d \\
            cd &= (1\ 3)(2\ 4)\ \circ\ (1\ 4)(2\ 3) = (1\ 2)(3\ 4) = b \\
            db &= (1\ 4)(2\ 3)\ \circ\ (1\ 2)(3\ 4) = (1\ 3)(2\ 4) = c \\
            dc &= (1\ 4)(2\ 3)\ \circ\ (1\ 3)(2\ 4) = (1\ 2)(3\ 4) = b \\
          \end{align*}
          \textbf{Cayley Table:}
          \begin{center}
            \begin{tabular}{ c c c c c}
              \toprule
              & $\epsilon$ & $b$ & $c$ & $d$ \\
              \midrule
              $\epsilon$ & $\epsilon$ & $b$ & $c$ & $d$ \\
              $b$ & $b$ & $\epsilon$ & $d$ & $c$ \\
              $c$ & $c$ & $d$ & $\epsilon$ & $b$ \\
              $d$ & $d$ & $c$ & $b$ & $\epsilon$ \\
              \bottomrule
            \end{tabular}
          \end{center}
      \end{enumalph}
      Now, we need to show that $H$ is isomorphic to $D_4$.
      First, $H$ and $D_4$ have the same order, since $|H| = |D_4| = 4$.
      Looking at $H$ under composition, we see that every element is its own inverse,
      and $H = \{\epsilon, b, c, cb\}$
      where $b = (1\ 2)(3\ 4)$, $c = (1\ 3)(2\ 4)$, and $bc = d = (1\ 4)(2\ 3)$.
      
      
      \noindent
      Possible isomorphism:
      \begin{align*}
        \phi\ \colon\ &H \to D_4 \\
        &\epsilon \mapsto 1 \\
        &b \mapsto r \\
        &c \mapsto s \\
      \end{align*}
      Then, $\phi(cb) = \phi(c)\cdot\phi(b) = s\cdot r = sr = \phi(d)$.

      \noindent
    \end{Answer}

    \item What are the left cosets of $H$ in $S_4$?
      How many are there, and how many elements are in each coset?
      Write each coset in the form $xH$ for some $x \in S_4$.
    \begin{Answer}
      $|H| = 4$ and $|S_4| = 4! = 24$.
      Therefore, $H$ has a total of $5$ left cosets in $S_4$.
      Each coset has $4$ elements.
      \begin{align*}
        \epsilon \circ H = H &= \{\epsilon,(1\ 2)(3\ 4),(1\ 3)(2\ 4),(1\ 4)(2\ 3)\} \\
        (1\ 2) \circ H &= \{(1\ 2), (3\ 4), (1\ 3\ 2\ 4), (1\ 4\ 2\ 3)\} \\
        (1\ 3) \circ H &= \{(1\ 3), (1\ 2\ 3\ 4), (2\ 4), (1\ 4\ 3\ 2)\} \\
        (1\ 4) \circ H &= \{(1\ 4), (1\ 2\ 4\ 3), (1\ 3\ 4\ 2), (2\ 3)\} \\
        (1\ 2\ 3) \circ H &= \{(1\ 2\ 3), (1\ 3\ 4), (2\ 4\ 3), (1\ 4\ 2)\} \\
        (1\ 2\ 4) \circ H &= \{(1\ 2\ 4), (1\ 4\ 3), (1\ 3\ 2), (2\ 3\ 4)\} \\
      \end{align*}
    \end{Answer}
    \item Show that $H \trianglelefteq S_4$ is normal.  \emph{[Hint: the cycle type is 
    preserved under conjugation!]}
    \begin{Answer}
      For $H$ to be normal,
      It must hold that $hxh^{-1} = x$ for all $x \in S_4$ and $h \in H$.
      This condition is valid for all elements that have cycles of even order.
      Since all elements in $H$ have cycles of order $2$, 
      this condition is valid for all elements in $H$,
      and $H$ is normal.
    \end{Answer}
    \item Show that $S_4/H \simeq S_3$.  \emph{[Hint: choose good representatives of 
    the cosets.]}
    \begin{Answer}
      Choosing the coset representatives
      $\epsilon,\ (1\ 2),\ (1\ 3),\ (2\ 3),\ (1\ 2\ 3),\ (1\ 3\ 2)$,
      We can interpret $S_4/H$ as the set of permutations of $S_4$ that fix the element $4$,
      i.e the stabilizers of $4$.
      This is homomorphic to $S_3$ (if we disconsidered the element $4$ altogether).
      \begin{align*}
        S_3 = \{\epsilon, (1\ 2), (1\ 3), (2\ 3), (1\ 2\ 3), (1\ 3\ 2)\} \\
        S_4/H = \{\epsilon, (1\ 2), (1\ 3), (2\ 3), (1\ 2\ 3), (1\ 3\ 2)\}
      \end{align*}

    \end{Answer}
  \end{enumalph}
  The group $H$ is denoted $V_4$ and called the \emph{Klein four-group}
  (\emph{Vierergruppe} in German, hence the label).  
\end{problem}

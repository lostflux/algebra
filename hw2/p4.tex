\begin{problem}{\textsf{(some of DF 1.6.6)}}
  \begin{enumalph}
    \item Let $\R^\times = \R \smallsetminus \{0\}$ be the set of nonzero real numbers.
    Then $\R^\times$ is a group under multiplication.  Define a second binary operation
    on $\R^\times$ by $x*y=xy/2$ for $x,y \in \R^\times$.  Show that $(\R^\times,*)$ is
    a group, and find an isomorphism $\phi \colon (\R^\times,\cdot) \xrightarrow{\sim} 
    (\R^\times,*)$.  \emph{[Hint: if it helps, write $G=\R^\times$ in the second case 
    with the nonstandard operation.]}
    \begin{Answer}
      \item Let's pick arbitrary $x,y,z \in \R^\times$.  Then:
      \begin{align*}
        x*y &= \frac{xy}{2} \in \R^\times\qquad \zaff{(\text{Closure})} \\
        \\
        (x*y)*z &= \frac{xy}{2}*z = \frac{xyz}{4} = x*\frac{yz}{2} = x*(y*z)\qquad \zaff{(\text{Associative})}\\
        \\
        x * 2 &= x \cdot \frac{2}{2} = x = 2 \cdot \frac{x}{2} = 2 * x\qquad \zaff{(\text{Identity = $2$})} \\
        \\
        x * (4/x) &= x \cdot \frac{4}{2x} = 2 = \frac{4}{x}\cdot \frac{x}{2} = (4/x) * x\qquad \zaff{(\text{Inverse of $x$ is = $4/x$})} \\
      \end{align*}
      Thus, $(\R^\times,*)$ is a group.
      
      \noindent
      Let's define $\phi \colon (\R^\times,\cdot) \xrightarrow{\sim} 
      (\R^\times,*)$ by $\phi(r) = 2r$ for $r \in \R^\times$. Then:
      \begin{align*}
        \phi(xy) &= \phi(x) * \phi(y)\qquad \zaff{(\text{Required condition})}\\
        2xy &= 2x * 2y \\
        2xy &= 2x \cdot \frac{2y}{2} \\
        2xy &= 2xy\\
      \end{align*}

      \noindent
      Furthermore, if $\phi$ is an isomorphism then it needs to map
      the identity in $(\R^\times,\cdot)$ to the identity in $(\R^\times,*)$.

      \begin{align*}
        \phi(e_1) &= \phi(e_2) \\
        e_1 &= 1 \\
        e_2 &= 2 \\
        \phi(e_1) &= \phi(1) = 2 \cdot 1 = 2 = e_2  \\
      \end{align*}
      Thus, $\phi$ is \textit{proven consistent} as an isomorphism between
      $(\R^\times,\cdot)$ and $(\R^\times,*)$.
    \end{Answer}
    \newpage
    \item Prove that the groups $\Z$ (under $+$) is not isomorphic to $\Q$ (under $+$).
    \emph{[Remark: there is a bijection from $\Z$ to $\Q$ that is not a homomorphism, 
    and a homomorphism that is not a bijection!]}

    \begin{Answer}

      Let's take $\phi \colon \Q \xrightarrow{\sim} \Z$ to be an isomorphism. Then:
      \begin{enumroman}
        \item By definition, $\phi$ needs to map the identity in $\Q$ to the identity in $\Z$.
        \item By definition, $\phi$ needs to be distributive over the group operations $(+)$.
        
        \noindent
        That is: $\phi(x+y) = \phi(x) + \phi(y)$ for all $x,y \in \Q$.
        
      \end{enumroman}

      \noindent
      Let's take an arbitrary $q \in \Q$ such that $2 \nmid q$.
      Let's take a corresponding $z \in \Z$ such that $\phi(q) = z$.
      
      \noindent
      Then, by the distributivity of $\phi$: $\phi(q) = \phi(q/2 + q/2) = \phi(q/2) + \phi(q/2)$.

      \noindent
      Let's define $z' \in \Z \colon z' = \phi(q/2)$. Then:
      \begin{align*}
        \phi(q) &= z \\
        \phi(\frac{q}{2} + \frac{q}{2}) &= z \\
        2z' &= z \\
        z' &= \frac{z}{2} \\
      \end{align*}

      \noindent
      We can conclude that, given $\phi(q) = z \in Z$,
      then $\phi(q/2) = z/2$ is not in $\Z$
      for any $q$ such that $2 \nmid \phi(q)$.
      For a specific example, consider the instances of $q$ such that
      $\phi(q) \in \{1, 3, 5, 7, \ldots\}$ (the odd positive integers).
      Then, $\phi(q/2) \in \{\frac{1}{2}, \frac{3}{2}, \frac{5}{2}, \frac{7}{2}, \ldots\} \notin \Z$.

      \noindent
      This contradiction ($\phi$ mapping elements from $Q$ to $\Z$ yet the same elements are
      seen to not be in $\Z$) proves that $\phi$ is not an isomorphism, and $\Z$ is not isomorphic
      to $\Q$ under addition.
    \end{Answer}
  \end{enumalph}
\end{problem}


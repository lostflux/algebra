

\begin{problem}{\textsf{(DF 0.1.7)}}
  Let $f \colon A \to B$ be a surjective map of sets.  For $y \in B$, let
  \[ f^{-1}(y) \colonequals \{x \in A : f(x) = y\} \]
  be the \emph{preimage} or \emph{fiber} of $f$ over $y$.  (The map $f$ is bijective 
  if and only if $f^{-1}(y)=\{x\}$ consists of a single element $x \in A$, in which 
  case we can define $f^{-1}$ as a function, removing the set brackets.  But we 
  always have fibers.)
  Define a relation by $a \sim b$ if $f(a)=f(b)$.  Show that this relation is an 
  equivalence relation whose equivalence classes are the fibers of $f$.
\end{problem}
\begin{Answer}
  What we know (so far):
  \begin{enumalph}
    \item $f$ is surjective, meaning, for every $y \in B$, there exists
    \textbf{at least one} $x \in A$ such that $f(x)=y$.
    \item We define the relation $a \sim b$ to hold if $f(a)=f(b)$.
    From this, we can note:
    \begin{enumroman}
      \item \textbf{Symmetry: }$a \sim b \implies f(a)=f(b) \implies f(b) = f(a) \implies b \sim a$.
      \item \textbf{Reflexivity: } For every $a \in A$ acted on by $f$, $f(a)=f(a)$, so $a \sim a$.
      \item \textbf{Transitivity: } If $a \sim b$ and $b \sim c$, then $f(a)=f(b)=f(c)$, so $a \sim c$.
    \end{enumroman}
  \end{enumalph}

  \noindent
  Since $\sim$ has symmetry, reflexivity, and transitivity,
  we can conclude that $\sim$ is an equivalence relation.

  \bigskip
  \noindent
  Next, we show that the equivalence classes of $\sim$ are the fibers of $f$.

  \bigskip
  \noindent
  First, let's define the equivalence classes of $\sim$.

  \noindent
  Since $f$ is surjective, for every $y \in B$,
  there exists at least one $x \in A$ such that $f(x)=y$.

  \noindent
  Let's take one such element, $x_0 \in A$ and its corresponding $y_0 \in B$
  such that $f(x_0) = y_0$.
  
  
  \noindent
  The equivalence class of $x_0$ under $f$ is the set of all elements $x \in A$
  such that $f(x)=f(x_0)=y_0$.

  \noindent
  This, by definition, implies that $x \sim x_0$, and $x \in f^{-1}(y_0)$.

  \begin{align*}
    [x_0] = \{x \in A \colon x \sim x_0\qquad \text{(meaning $f(x) = f(x_0)$)} \}
  \end{align*}

  \bigskip
  \noindent
  Next, we need to show that the equivalence classes of $\sim$ are the fibers of $f$.

  \noindent
  Let's take an arbitrary equivalence class $[x_0]$ such as the one derived above.

  \noindent
  We know that $[x_0] \subseteq A$ and $f(x) = y_0$ for all $x \in [x_0]$.

  \noindent
  Then, by definition of inverses, $f^{-1}(y_0) = [x_0]$.

  \noindent
  Generally,
  $[x] = f^{-1}(f(x))$ for all $x \in A$, and $[x]$ is the equivalence class of $x$ under $\sim$.

  
\end{Answer}

\begin{problem}{\textsf{(sorta-not-really DF 0.3.15(b))}}
  \begin{enumalph}
  \item For $a=69$ and $n=372$,
  determine the greatest common divisor $g \colonequals \gcd{a}{n}$,
  the least common multiple $\lcm{a}{b}$,
  and write $g=ax+by$ with $x,y \in \Z$.
  Is $\overline{a} \in {(\Z/n\Z)}^\times$?  If so, what is $\overline{a}^{-1}$?
  \begin{Answer}
    Factoring, we get $69 = 3 \cdot 23$ and $372 = 2^2 \cdot 3 \cdot 31$.

    \noindent
    By definition, given:
    
    \begin{align*}
      a &= 1^{a_1} \cdot 2^{a_2} \cdot 3^{a_3} \cdots {(n-1)}^{a_{n-1}} \cdot n^{a_{n}} \\
      b &= 1^{b_1} \cdot 2^{b_2} \cdot 3^{b_3} \cdots {(n-1)}^{b_{n-1}} \cdot n^{b_{n}} \\
    \end{align*}

    \noindent
    Then we can define the $gcd$ and $lcm$ as:
    \begin{align*}
      \gcd{a}{b} &= 1^{\min(a_1, b_1)} \cdot 2^{\min(a_2, b_2)} \cdot 3^{\min(a_3, b_3)} \cdots {(n-1)}^{\min(a_{n-1}, b_{n-1})} \cdot n^{\min(a_{n}, b_{n})} \\
      \lcm{a}{b} &= 1^{\max(a_1, b_1)} \cdot 2^{\max(a_2, b_2)} \cdot 3^{\max(a_3, b_3)} \cdots {(n-1)}^{\max(a_{n-1}, b_{n-1})} \cdot n^{\max(a_{n}, b_{n})} \\
    \end{align*}

    \noindent
    For $a=69$ and $b=372$, we get:
    \begin{align*}
      \gcd{69}{372} &= 2^0 \cdot 3 \cdot 23^0 \cdot 31^0 = 3 \\
      \lcm{69}{372} &= 2^2 \cdot 3 \cdot 23 \cdot 31 = 8556 
    \end{align*}

    \noindent
    Using the Euclidean algorithm:
    \begin{align*}
      372 &= 69 \cdot 5 + 27 \\
      69 &= 27 \cdot 2 + 15 \\
      27 &= 15 \cdot 1 + 12 \\
      15 &= 12 \cdot 1 + 3 \\
      12 &= 3 \cdot 4 + 0
    \end{align*}

    \noindent
    Back-substituting, we get:
    \begin{align*}
      3 &= 15 - 12 \\
        &= 15 - (27 - 15) = 2 \cdot 15 - 27 \\
        &= 2 (69 - 2 \cdot 27) - 27 = 2 \cdot 69 - 5 \cdot 27 \\
        &= 2 \cdot 69 - 5 (372 - 5 \cdot 69) = 27 \cdot 69 - 5 \cdot 372 \\
        &= 27 \cdot 69 - 5 \cdot 372 \\ 
    \end{align*}
    Thus, we can write $3 = 27 \cdot 69 - 5 \cdot 372$, with $x = 27$ and $y = -5$.

    \bigskip\noindent
    \textbf{Is $\overline{a} \in {(\Z/n\Z)}^\times$?  If so, what is $\overline{a}^{-1}$?}
    
    \noindent
    No, $\overline{69} \notin {(\Z/372\Z)}^\times$ because $\gcd{69}{372} \neq 0$
    (that is, $69$ and $372$ are not coprime). 
  \end{Answer}
  \newpage
  \item Taking $n = 89$, what is the order of $\overline{2}$ in ${(\Z/n\Z)}^\times$?
  \begin{Answer}
    The order $o(\overline{a})$ of an element $\overline{a} \in {(\Z/n\Z)}^\times$
    is the smallest positive integer $k$ such that $\overline{a}^k \equiv 1 \pmod{n}$.

    \noindent
    For a single element, we can use the following algorithm to find the order:

    \bigskip
    \begin{Verbatim}[fontsize=\footnotesize, frame=single, framesep=2mm, commandchars=\\\{\}]
      function order (a, n)
        k = 1
        while a^k is not congruent to 1 mod n
          k = k + 1
        return k
    \end{Verbatim}
    \noindent
    We get:
    \begin{Verbatim}[fontsize=\footnotesize, frame=single]
      ghci> order 2 89
      Found 2 ^ 11 = 2048 == 1 (mod 89)
    \end{Verbatim}

    \bigskip
    \noindent
    The order of $\overline{2}$ in ${(\Z/89\Z)}^\times$ is $11$.
  \end{Answer}
  \item How many elements are there in ${(\Z/360\Z)}^\times$?
  \begin{Answer}
    All elements in ${(\Z/n\Z)}^\times$ have to be coprime to $n$.

    \noindent
    There are a total of $\phi(n)$ relatively prime numbers less than $n$.

    \noindent
    We can calculate the value of $\phi(n)$ for reasonably small $n$ using a simple algorithm:

    \begin{Verbatim}[fontsize=\footnotesize, frame=single, framesep=2mm, commandchars=\\\{\}]
      function phi (n)
        count = 0
        for i = 1 to n
          if gcd (i, n) == 1
            count = count + 1
        return count
    \end{Verbatim}
    \noindent
    We get:
    \begin{Verbatim}[fontsize=\footnotesize, frame=single]
      ghci> phi 360
      96
    \end{Verbatim}

    \noindent
    Optionally, we can also factor $360 = 2^3 \cdot 3^2 \cdot 5$ and use the
    multiplicative property of the $phi$ function to get:
    
    \begin{align*}
      \phi(360) &= \phi(2^3 \cdot 3^2 \cdot 5) \\
                &= \phi(2^3) \cdot \phi(3^2) \cdot \phi(5) \\
                &= (2^3 - 2^2) \cdot (3^2 - 3) \cdot (5 - 1) \\
                &= 8 \cdot 6 \cdot 4 \\
                &= 96
    \end{align*}
    Hence, there are a total of $96$ elements in ${(\Z/360\Z)}^\times$.
  \end{Answer}
  \end{enumalph}
\end{problem}


\begin{problem}{(\textsf{DF 7.1.7})} \\
  The \textsf{center} of a ring $R$ is 
  \[ Z(R) \colonequals \{z \in R : zr=rz \text{ for all $r \in R$}\}. \]
  \begin{enumalph}
    \item For a ring $R$, show that $Z(R) \subseteq R$ is a subring
      (in particular, containing $1$).
      \begin{Answer}
        \begin{enumalph}
          \item $1$ is in $Z(R)$ since $1\cdot r=r \cdot 1 = r$ for all $r \in R$.
          \item Since $Z(R) \subseteq R$, its elements inherit associativity and distributivity from $R$.
          \item \textbf{Closure:} \\
            Given $a, b \in Z(R)$, then $a \cdot r = r \cdot a$
            and $b \cdot r = r \cdot b$ for all $r \in R$.
            Then:
            \begin{align*}
              (a + b) \cdot r &= ar + br = r \cdot (a + b) \implies a + b \in Z(R) \\
              (a \cdot b) \cdot r &= ar \cdot br = r \cdot (a \cdot b) \implies a \cdot b \in Z(R)
            \end{align*}
            Therefore, $Z(R)$ contains the identity and is closed
            under both addition and multiplication, making it a subring.
        \end{enumalph}
      \end{Answer}
    \item Show that the center of a division ring is a field.
    \begin{Answer}
      \begin{enumalph}
        \item By definition, the elements of $Z(R)$ commute with all other elements of $R$.
        \item As shown above --- the elements of $Z(R)$ form a subring.
        \item Given $R$ is a \emph{division} ring, then
          all elements in $Z(R)$ have multiplicative inverses
          since $Z(R) \subseteq R$.
        \item Given (b) and (c), then $Z(R)$ is a division ring.
        \item Given (a) and (d), then $Z(R)$ is a field.
      \end{enumalph}
    \end{Answer}
  \end{enumalph}
\end{problem}

\begin{problem}{(\textsf{DF 7.2.6--7.2.7})} \\
  Let $R$ be a commutative ring and let $n \in \Z_{\geq 1}$.
  Let $A={(a_{ij})}_{i,j} \in \M_n(R)$ be an $n \times n$-matrix
  whose $(i,j)$-entry is $a_{ij} \in R$.
  Let $E_{ij} \in \M_n(R)$ be the matrix whose $(i,j)$ entry is $1$ with all other entries zero.
  For example,
  \[E_{12}=\begin{pmatrix} 0 & 1 & 0 \\ 0 & 0 & 0 \\ 0 & 0 & 0\end{pmatrix} \in \M_3(R)\]  
  \begin{enumalph}
    \item Prove that $E_{ij}A$ is the $n \times n$-matrix whose $i$th row
      is equal to the $j$th row of $A$, with all other rows zero:
      \[
        E_{ij}A=\begin{pmatrix}
        0 & \cdots & 0 \\
        \vdots & \ddots & \vdots \\
        0 & \cdots & 0 \\
        a_{j1} & \cdots & a_{jn} \\
        0 & \cdots & 0 \\
        \vdots & \ddots & \vdots \\
        0 & \cdots & 0
        \end{pmatrix}
      \] (in the $i$th row)
      \begin{Answer}
        Let's consider two matrices $X, Y \in \M_n(R)$ and their product $XY$. \\
        By definition of matrix multiplication, ${XY}_{ij} = \sum_{k=1}^n X_{ik}Y_{kj}$.
        Setting $X=E_{ij}$ and $Y=A$, let's consider arbitrary indices $i', j'$ in the
        matrix $E_{ij}A$:
        \[
          {(E_{ij}A)}_{i'j'} = \sum_{k=1}^n {(E_{ij})}_{i'k}A_{kj'} \\
        \]
        When $i' \neq i$, then the index in $E_{ij}A$ has a $0$,
        since $E_{ij}$ has all zeroes in every row except the $i$-th row. \\
        When $i' = i$, then the row $i'$ in $E_{ij}$ has a $1$ in the $j$-th column,
        and the corresponding row in $E_{ij}A$ is $1 \times$ the $j$-th row of $A$.
      \end{Answer}
    \newpage
    \item Prove that $AE_{ij}$ is the $n \times n$-matrix whose $j$th column
      equals the $i$th column of $A$, with all other columns zero:
      \[
        AE_{ij}=\begin{pmatrix}
        0 & \cdots & 0 & a_{1i} & 0 & \cdots & 0 \\
        \vdots & \ddots & \vdots & \vdots & \vdots & \ddots & \vdots \\
        0 & \cdots & 0 & a_{ni} & 0 & \cdots & 0 \\
        \end{pmatrix}
      \]
      (in the $j$th column)
      \begin{Answer}
        Let's consider two matrices $X, Y \in \M_n(R)$ and their product $XY$. \\
        By definition of matrix multiplication, ${XY}_{ij} = \sum_{k=1}^n X_{ik}Y_{kj}$.
        Setting $X=A$ and $Y=E_{ij}$, let's consider arbitrary indices $i', j'$ in the
        matrix $AE_{ij}$:
        \[
          {(AE_{ij})}_{i'j'} = \sum_{k=1}^n A_{ik}{(E_{ij})}_{kj'} \\
        \]
        When $j' \neq j$, then the index in $AE_{ij}$ has a $0$,
        since $E_{ij}$ has all zeroes in every column except the $j$-th column. \\
        When $j' = j$, then the column $j'$ in $E_{ij}$ has a $1$ in the $i$-th row,
        and the corresponding column in $AE_{ij}$ is $1 \times$ the $i$-th column of $A$.
      \end{Answer}
    \item Prove that $E_{pq}AE_{rs}$ is the matrix whose $(p,s)$-entry
      is $a_{qr}$, with all other entries zero.
      \begin{Answer}
        Following the demonstration in the previous two parts, let's consider
        arbitrary indices $i', j'$ in the matrix $E_{pq}AE_{rs}$
        after the multiplication in part (a) is followed by the multiplication in part  (b):
        When $i' \neq p$ or $j' \neq s$, then the index in $E_{pq}AE_{rs}$ has a $0$,
        since either $E_{pq}$ has all zeroes in that row (when $i' \neq p$)
        or $E_{rs}$ has all zeroes in that column (when $j' \neq s$). \\
        When $i' = p$ and $j' = s$, then:
        \begin{enumalph}
          \item The row $i' = p$ in $E_{pq}$ has a $1$ in the $q$-th column,
            therefore the corresponding row in $E_{pq}A$ is the $q$-th row of $A$.
          \item The column $j' = s$ in $E_{rs}$ has a $1$ in the $r$-th row,
            therefore the corresponding column in $AE_{rs}$ is the $r$-th column of $E_{pq}A$.
          \item After the two operations, only the intersection of the $p$-th row
            and the $s$-th row will retain an entry from $A$.
            Particularly, the $p$-th row of the product will equal the  $q$-th column of $A$
            as shown in (a), and the $s$-th column of the product will equal the $r$-th
            column of $E_{pq}A$. Consequently, the intersection of the row and column
            will contain the element initially at position $(q, r)$ in $A$.
        \end{enumalph}
      \end{Answer}
    \newpage
    \item Prove that the center of $\M_n(R)$ is the set (subring!) of scalar matrices 
      (i.e., diagonal matrices with the same entry down the diagonal). \\
      \emph{[Hint: if you get lost in the indices, do the cases $n=2$ and maybe $n=3$ first.]}
      \begin{Answer}
        By definition, the center of a ring must commute with all other elements in the ring.

        For a simpler case, let's consider $M_3(R)$, with \\
        $X = \begin{pmatrix} a & b & c \\ d & e & f \\ g & h & i \end{pmatrix} \in Z(\M_3(R))$
        and $Y = \begin{pmatrix} j & k & l \\ m & n & o \\ p & q & r \end{pmatrix} \in \M_3(R)$.
        We must have that $XY = YX$.
        \begin{align*}
          XY &= \begin{pmatrix} a & b & c \\ d & e & f \\ g & h & i \end{pmatrix}
                \times \begin{pmatrix} j & k & l \\ m & n & o \\ p & q & r \end{pmatrix}
                = \begin{pmatrix}
                    aj + bm + cp & ak + bn + cq & al + bo + cr \\
                    dj + em + fp & dk + en + fq & dl + eo + fr \\
                    gj + hm + ip & gk + hn + iq & gl + ho + ir \\
                  \end{pmatrix} \\
          YX &= \begin{pmatrix} j & k & l \\ m & n & o \\ p & q & r \end{pmatrix}
                \times \begin{pmatrix} a & b & c \\ d & e & f \\ g & h & i \end{pmatrix}
                = \begin{pmatrix}
                    ja + kd + {l}g & jb + ke + lh & jc + kf + li \\
                    ma + nd + og & mb + ne + oh & mc + nf + oi \\
                    pa + qd + rg & pb + qe + rh & pc + qf + ri \\
                  \end{pmatrix} \\
        \end{align*}

        Comparing the two matrices, we see that the elements have different constituent terms,
        therefore $XY \neq YX$ in general unless $X = Y$. \\
        However, the terms at positions $(1, 1)$, $(2, 2)$, and $(3, 3)$ share a component in
        each matrix. Particularly;  \\
        \centering
        $XY_{11} = \crim{aj} + bm + cp$ and $YX_{11} = \crim{aj} + kd + {l}g$. \\
        $XY_{22} = dk + \crim{en} + fq$ and $YX_{22} = mb + \crim{en} + oh$. \\
        $XY_{33} = gl + ho + \crim{ir}$ and $YX_{33} = pc + qf + \crim{ri}$. \\
        \flushleft
        The common terms are those occurring yielded from the diagonal elements.
        One way to make $XY = YX$ is to set $X$ and $Y$ such that the diagonal elements
        are nonzero and all other terms are $0$.

        \begin{align*}
          X &= \begin{pmatrix} a & 0 & 0 \\ 0 & e & 0 \\ 0 & 0 & i \end{pmatrix}, \quad
          Y = \begin{pmatrix} d & 0 & 0 \\ 0 & n & 0 \\ 0 & 0 & r \end{pmatrix} \\
          XY &= \begin{pmatrix} ad & 0 & 0 \\ 0 & en & 0 \\ 0 & 0 & ir \end{pmatrix} \\
          YX &= \begin{pmatrix} ad & 0 & 0 \\ 0 & en & 0 \\ 0 & 0 & ir \end{pmatrix} \\
        \end{align*}

        However, in the case of the center, we need to have no restrictions on the second matrix.
        Thus, consider the less restricted case when $X = \begin{pmatrix} a & 0 & 0 \\ 0 & e & 0 \\ 0 & 0 & i \end{pmatrix}$
        and $Y = \begin{pmatrix} j & k & l \\ m & n & o \\ p & q & r \end{pmatrix}$. We see that:
        \begin{align*}
          XY &= \begin{pmatrix} aj & ak & al \\ em & en & eo \\ ip & iq & ir \end{pmatrix} \\
          YX &= \begin{pmatrix} aj & ek & il \\ am & en & io \\ ap & eq & ir \end{pmatrix}
        \end{align*}
        Other than the diagonal entries, the product matrices have entirely different elements!
        However, we see that each pair of positions for indices $(i, j)$ in $X$ and $Y$
        involves the same element from $Y$ and only the elements from $X$ change.
        
        For instance;
        \centering
        $XY_{12} = ak$ and $YX_{12} = ek$. \\
        $XY_{13} = al$ and $YX_{13} = il$. \\
        $XY_{21} = em$ and $YX_{21} = am$. \\
        $XY_{23} = eo$ and $YX_{23} = io$. \\
        $XY_{31} = ip$ and $YX_{31} = ap$. \\
        $XY_{32} = iq$ and $YX_{32} = eq$. \\

        \flushleft
        Therefore, setting $a = e = i$ and setting every other entry equal to $0$ in $X$
        makes $XY = YX$ for any unrestricted matrix $Y \in \M_3(R)$.
      \end{Answer}
  \end{enumalph}
\end{problem}

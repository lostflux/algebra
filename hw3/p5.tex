\begin{problem} \textsf{(DF 2.3.10)}
  
  \noindent
  \begin{enumalph}
    \item Let $G=\langle a \rangle$ be a cyclic group of order $n \in \Z_{\geq 1}$.  
      For $k \in \Z$, show that $a^k$ has order $n/g$ where $g=\gcd{k}{n}$.  \emph{[Hint: 
      what is $\#\langle a^k \rangle$?]}
      \begin{Answer}

        \noindent
        We can first observe that $\#\langle a \rangle = n \implies a^n = e$.

        \noindent
        Suppose $m = \#\langle a^k \rangle$. Then ${(a^k)}^m = a^{km} = e$.
        
        \noindent
        Since $G$ is cyclic, this implies that $n \mid km$ (only powers of $a$
        that are multiples of $n$ equal the identity).
        \begin{align*}
          \#\langle a \rangle = n &\implies a^n = e \\
          \#\langle a^k \rangle = m &\implies {(a^k)}^m = a^{km} = e \\
        \end{align*}

        \noindent
        Since $a$ has order $n$, only powers
        of $a$ that are equal to $e$ are multiples of $n$. Let's write
        $km$ as $km = pn, p \in \Z$. 
        
        \noindent
        Then:
        
        \begin{align*}
          a^{km} = a^{pn} \\
        \end{align*}
      \end{Answer}

    \item What is the order of $\overline{30}$ in $\Z/54\Z$?  Write out all of the 
      elements in $\langle \overline{30} \rangle$ and their orders.
      \begin{Answer}
        \noindent
        We can first observe that $\gcd{30}{54} = 6$. Then:
        \begin{align*}
          o(\overline{30}) = \frac{54}{6} = 9 \\
        \end{align*}

        \noindent
        We can then write out all of the elements in $\langle \overline{30} \rangle$:
        \begin{align*}
          \langle \overline{30} \rangle = \{ \overline{0}, \overline{30},\overline{30}, \overline{6}, \overline{36}, \overline{312}, \overline{42}, \overline{18}, \overline{48}, \overline{24} \} \\
        \end{align*}
      \end{Answer}
    \item For which values of $n \in \{8,9,10,11,12\}$ is
      $(\Z/n\Z)^\times$ a cyclic group?
      \begin{Answer}
        \noindent
        \begin{align*}
          {(\Z/8\Z)}^\times &= \{1, 3, 5, 7 \} \\
          {(\Z/9\Z)}^\times &= \{1, 2, 4, 5, 7, 8 \} \\
          {(\Z/10\Z)}^\times &= \{1, 3, 7, 9 \} \\
          {(\Z/11\Z)}^\times &= \{1, 2, 3, 4, 5, 6, 7, 8, 9, 10 \} \\
          {(\Z/12\Z)}^\times &= \{1, 5, 7, 11 \} \\
        \end{align*}

        \noindent
        Cyclic groups must have a generator $g$ such that $\langle g \rangle = G$.

        If we check the groups, we see:
        \begin{enumalph}
          \item ${(\Z/8 \Z)}^\times$ lacks a generator, therefore it cannot be cyclic.
            \begin{align*}
              \langle 1 \rangle &= \{ 1 \} \\
              \langle 3 \rangle &= \{ 3, 1 \} \\
              \langle 5 \rangle &= \{ 5, 1 \} \\
              \langle 7 \rangle &= \{ 7, 1 \} \\
            \end{align*} 
          \item ${(\Z/9 \Z)}^\times$ is cyclic, since $\langle 2 \rangle
            = \{ 2, 4, 8, 7, 5, 1 \}$.
          \item ${(\Z/10 \Z)}^\times$ is cyclic because $\langle 3 \rangle
            = \{ 3, 9, 7, 1 \}$.
          \begin{align*}
            \langle 1 \rangle &= \{ 1 \} \\
            \langle 3 \rangle &= \{ 3, 9, 7, 1 \} \\
            \langle 7 \rangle &= \{ 7, 9, 3, 1 \} \\
            \langle 9 \rangle &= \{ 9, 1 \} \\
          \end{align*}
          \item $(\Z/11\Z)^\times$ is cyclic because $\langle 2 \rangle
            = \{ 2, 4, 8, 5, 10, 9, 7, 3, 6, 1 \}$.
        \end{enumalph}
      \end{Answer}

  \end{enumalph}
\end{problem}

\begin{problem}{\textsf{(DF 1.3.1, sorta 1.3.7)}}
  
  \noindent
  Let $G$ be a group and let $A \subseteq G$ be a subset.
  For $g \in G$, write \[ gAg^{-1} \colonequals \{gag^{-1} : a \in A\}. \]
  The \emph{normalizer} of $A$ in $G$ is defined to be:
  \[ N_G(A) \colonequals \{g \in G : gAg^{-1}= A\}. \]
  \begin{enumalph}
    \item Let $g \in G$.  Show that the map 
      \begin{align*}
        \phi_g \colon G &\to G \\
        x &\mapsto gxg^{-1}
        \end{align*}
      is an isomorphism of groups.  (We call an isomorphism from a group to itself an \
      emph{automorphism}.)
      \begin{Answer}
        \begin{enumroman}
          \item $\phi_g$ is a homomorphism.
            
            \noindent
            Let $x,y \in A, g \in G$. Then:
            \begin{align*}
              \phi_g(x) &= gxg^{-1} \\
              \phi_g(y) &= gyg^{-1} \\
              \phi_g(xy) &= gxyg^{-1} = (gxg^{-1}) \cdot (gyg^{-1}) = \phi_g(x) \cdot \phi_g(y) \\
            \end{align*}
          \item $\phi_g$ is injective.
            
              \noindent
              Suppose that $\phi_g(x) = \phi_g(y)$.

              \noindent
              Then, $gxg^{-1} = gyg^{-1}$, which implies that $x = y$.
          \item $\phi_g$ is surjective.
            
            \noindent
            Suppose $a \in A$ is some element acted on by $\phi_g$,
            such that $a = \phi_g(a_0) = ga_0g^{-1}$ for some $a_0$,
            but $a \neq \phi_g(a_1)$ for all $a_1 \in A$.
            
            \noindent
            Then:
            \begin{align*}
              ga_0g^{-1} \neq ga_1g^{-a} \quad \forall \quad a_1 \in A \\
              \implies a_0 \neq a_1 \quad \forall \quad a_1 \in A \\
              \implies a_0 \notin A \\
            \end{align*}

            \noindent
            We see that any such element must be the result of $\phi_g$
            acting on an element not contained in $A$, yet we defined $\phi_g$
            to act on elements in $A$.
        \end{enumroman}
      \end{Answer}

    \newpage
    \item Show that $N_G(A)$ is a subgroup of $G$ that contains the centralizer
      $C_G(A)$.
      \emph{[Hint: if $gAg^{-1}=A$ then $h(gAg^{-1})h^{-1} = hAh^{-1}$, we just took 
      two equal sets and conjugated their elements by $h$.]}
      \begin{Answer}
        \begin{enumroman}
          \item $N_G(A)$ is a subgroup of $G$.
            
            \noindent
            Let $g,h \in N_G(A)$.

            Then, $gAg^{-1} = A$ and $hAh^{-1} = A$.

            Therefore, $ghAg^{-1}h^{-1} = A$ and $hAg^{-1}h^{-1} = A$.

            Thus, $ghAh^{-1}g^{-1} = g(hAh^{-1}) g^{-1} = gAg^{-1} = A$, which implies that
            $gh \in N_G(A)$.

            Similarly, $hg \in N_G(A)$.
          \item $C_G(A) \subseteq N_G(A)$.
            
            \noindent
            Let $g \in C_G(A)$.

            Then, $gA = Ag$.

            Therefore, $gAg^{-1} = Agg^{-1} = A$.
            Thus, $g \in N_G(A)$.
        \end{enumroman}
      \end{Answer}
    \item Let $G=Q_8$ and let $A=\{\pm i\}$.  Compute $C_G(A)$ and $N_G(A)$.
    \begin{Answer}
      Cayley Table for $Q_8$:

      \begin{tabular}{c c c c c c c c c }
        \toprule
          & $1$ & $i$ & $j$ & $k$ & $-1$ & $-i$ & $-j$ & $-k$ \\
        \midrule
          $1$ & $1$ & $i$ & $j$ & $k$ & $-1$ & $-i$ & $-j$ & $-k$ \\
          $i$ & $i$ & $-1$ & $k$ & $-j$ & $i$ & $1$ & $-k$ & $j$ \\
          $j$ & $ j $ & $-k$ & $-1$ & $i$ & $j$ & $k$ & $1$ & $-i$ \\
          $k$ & $k$ & $j$ & $-i$ & $-1$ & $k$ & $-j$ & $i$ & $1$ \\
          $-1$ & $-1$ & $-i$ & $-j$ & $-k$ & $1$ & $i$ & $j$ & $k$ \\
          $-i$ & $-i$ & $1$ & $-k$ & $j$ & $-i$ & $-1$ & $k$ & $-j$ \\
          $-j$ & $-j$ & $k$ & $1$ & $-i$ & $-j$ & $-k$ & $-1$ & $i$ \\
          $-k$ & $-k$ & $-j$ & $i$ & $1$ & $-k$ & $j$ & $-i$ & $-1$ \\
        \bottomrule
      \end{tabular}

      \bigskip
      \begin{enumroman}
        \item $C_G(A) = \{\pm 1 \}$.
          \begin{align*}
            gA &= Ag \\
            \\
            1\cdot A &= A\cdot 1 \\
            -1 \cdot A &= A \cdot -1 \\
          \end{align*}
          All other elements do not satisfy this property.
          For instance, $i \cdot j = k$ but $j \cdot i = -k$.
        \item $N_G(A) = \{\pm 1, \pm i, \pm j, \pm k\}$.
          \begin{align*}
            gAg^{-1} &= A \\
            \\
            1^{-1} &= 1 \\
                   & 1 \cdot A \cdot 1 = A \\
                   \\
            -1^{-1} &= -1 \\
                    & -1 \cdot A \cdot -1 = A \\
                    \\
            i^{-1} &= -i \\
                   & i \cdot j = k, k \cdot -j = i \\
                    & -i \cdot j = -k, -k \cdot -j = -i \\
                \\
            j^{-1} &= -j \\
                   & j \cdot k = i, -j \cdot i = k \\
                   & -j \cdot k = -i, j \cdot -i = -k \\
                \\
            k^{-1} &= -k \\
                   & k \cdot i = j, -k \cdot j = i \\
                   & -k \cdot i = -j, k \cdot -j = i \\
          \end{align*}
      \end{enumroman}
    \end{Answer}
    \item Show that if $H \leq G$ is a subgroup, then $H \leq N_G(H)$.  \emph{[Hint: 
    use (a), with $G=H$.]}
    \begin{Answer}
      Let $g \in H$.

      Then, $ghg^{-1} = h$ for all $h \in H$ (by definition of the group operation).

      However, this implies that $g \in N_G(H)$.

      Therefore, it must hold that $H \leq N_G(H)$.
    \end{Answer}
  \end{enumalph}
\end{problem}

\begin{problem} \textsf{(sorta DF 1.7.18)}

  \noindent
  Let $F$ be a field, let $G=\GL_2(F)$, and let 
  \begin{equation*}
    H \colonequals \left\{
    \begin{pmatrix}
      a & b \\
      0 & d
    \end{pmatrix}
    \in G : a,d \in F^\times,\ b \in F\right\}. 
  \end{equation*}
  \begin{enumalph}
    \item Show that $H$ is a subgroup of $G$,
    and show $H$ is nonabelian whenever $\#F > 2$.
    What happens when $\#F=2$ (so $F \simeq \Z/2\Z$)?
    \begin{Answer}
      
      For $H$ to be a subgroup of $G$, it must:
      \begin{enumalph}
        \item contain the identity element $e = 
        \left(
          \begin{smallmatrix}
            1 & 0 \\
            0 & 1
          \end{smallmatrix}
        \right)
        \in G$.

          We see that $a = 1 \in F^\times$, $d = 1 \in F^\times$.
          and $b = 0 \in F$, so $e \in H$.
        \item be closed under multiplication.

          Let $A,B \in H$.
          Suppose $A = \left(
            \begin{smallmatrix}
              a & b \\
              0 & d
            \end{smallmatrix}
          \right)
          \in H$ and $B = \left(
            \begin{smallmatrix}
              x & y \\
              0 & z
            \end{smallmatrix}
          \right)
          \in H$.
          Then $AB = \left(
            \begin{smallmatrix}
              ax & ay + bz \\
              0 & dz
            \end{smallmatrix}
          \right)$.

          For $AB$ to be in $H$, its elements must satisfy the conditions of $H$.
          Particularly:
          \begin{enumroman}
            \item $a \in F^\times \; \land \; x \in F^\times \implies ax \in F^\times$.
            \item $d \in F^\times \; \land \; z \in F^\times \implies dz \in F^\times$.
            \item $ay + bz \in F$ since:
              \begin{itemize}
                \item $ a \in F^\times \; \land \; y \in F^\times \implies ay \in F^\times \subset F$.
                \item $ b \in F^\times \; \land \; z \in F^\times \implies bz \in F^\times \subset F$.
                \item $ay \in F \; \land \; bz \in F \implies ay + bz \in F$.
              \end{itemize}          
          \end{enumroman}
          Therefore, we can infer that $H$ is closed under multiplication.
        \item be closed under inversion.

          Let $A \in H$, such that 
          $A = \left(
            \begin{smallmatrix}
              a & b \\
              0 & d
            \end{smallmatrix}
          \right)$.
          Then $A^{-1} = \left(
            \begin{smallmatrix}
              a^{-1} & -{(ad)}^{-1}b \\
              0 & d^{-1}
            \end{smallmatrix}
          \right)$.

          We see that $A^{-1} \in H$ since its elements fit the specified domains.
          Particularly:
          \begin{enumroman}
            \item $a \in F^\times \implies a^{-1} \in F^\times$.
            \item $d \in F^\times \implies d^{-1} \in F^\times$.
            \item $a \in F^\times \; \land \; d \in F \implies ad \in F^\times
            \implies {(ad)}^{-1} \in F^\times \subset F \implies -{(ad)}^{-1}b \in F$.
          \end{enumroman}
          Therefore, we can infer that $G$ is closed under inversion.
      \end{enumalph}
    \end{Answer}

    \newpage
    \item Show that the map 
    \begin{align*}
      \phi \colon &H \to F^\times \\
                  &\begin{pmatrix} a & b \\ 0 & d \end{pmatrix} \mapsto a 
    \end{align*}
    is a surjective group homomorphism that is not an isomorphism.
    \begin{Answer}

      To prove homomorphism:
      \begin{enumalph}
        \item We need to prove that $\phi$ maps the identity element in $H$
         to the identity element in $F^\times$. Indeed, we see that:
         \begin{align*}
            e_H =
              \begin{pmatrix}
                1 & 0 \\
                0 & 1
              \end{pmatrix}
              &\implies
            \phi(e) = 1 = e_{F^\times}
         \end{align*}
        
        \item We need to show that $\phi(x \cdot y) = \phi(x) \cdot \phi(y)$.
        
        \noindent
         Suppose that:
          \begin{align*}
              A = \begin{pmatrix} a & b \\ 0 & d \end{pmatrix} \in H
              \; \land \;
              B = \begin{pmatrix} x & y \\ 0 & z \end{pmatrix} \in H
          \end{align*}
          Then:
          \begin{align*}
            AB = \begin{pmatrix} ax & ay + bz \\ 0 & dz \end{pmatrix} \in H \\
          \end{align*}
          and:
          \begin{align*}
            \phi(A) &= a \in F^\times \\
            \phi(B) &= x \in F^\times \\
            \phi(AB) &= ax = \phi(A) \cdot \phi(B) \in F^\times \\
          \end{align*}        
      \end{enumalph}

      \noindent
      To prove isomorphism, we would have to prove that $\phi$ is bijective.

        \begin{enumalph}
          \item It is trivial to prove that $\phi$ is surjective, since it maps
          a single matrix member $a \in F^\times$ of matrices in $H$ back to $F^\times$,
          and the map does not change $a$.
          \item However, $\phi$ is not injective, since it multiple different matrices
            in $H$ to the same element in $F^\times$. For example, we can take:
            \begin{align*}
              A = \begin{pmatrix} a & b \\ 0 & d \end{pmatrix} \in H
              \; &\land \;
              B = \begin{pmatrix} a & x \\ 0 & y \end{pmatrix} \in H
              \quad \colon \quad b \neq x \land d \neq y \\
              A &\neq B \\
              \phi(A) &= \phi(B) = a \\
            \end{align*}
        \end{enumalph}
      Therefore, $\phi$ cannot be an isomorphism because it is not injective.
    \end{Answer}
  \end{enumalph}
\end{problem}

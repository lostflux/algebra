\begin{problem}{(\textsf{DF 3.5.3})}
  \begin{enumalph}
    \item Prove that $S_n$ is generated by the set$\{(1\ 2),(2\ 3),\dots,(n-1\ n)\}$. 
      \emph{[Hint: Consider conjugates, e.g.\ ${(2\ 3)(1\ 2)(2\ 3)}^{-1}$.]}
      \begin{Answer}
        First, we note that every permutation can be written
        as a product of transpositions.
        Consider the trivial cases of $n = 1$. Then $S_1$ is generated by the single
        identity element.
        For $S_2$, we have the two elements $(1\ 2)$ and $(2\ 1)$, which
        are both transpositions generated by $(1\ 2)$.
        For each $n \geq 3$, supposing we have already shown that $S_{(n-1)}$ is
        generated by the set $\{(1\ 2),(2\ 3),\dots,(n-2\ n-1)\}$, we can
        generate any element $(i\ \ n)$ by conjugation:
        \begin{align*}
          (i\ \ (n-1)) \cdot ((n-1)\ \ n) \cdot ((n-1) \ \ i) &= (i\ \ n) \\
        \end{align*}
      \end{Answer}
    \item Show that every element in $A_n$ for $n \geq 3$
      can be written as the product of (not necessarily disjoint) $3$-cycles.
      \begin{Answer}
        The alternating group $A_n$ is generated by all cycles that can be written
        as an even product of transpositions.
        For any $a, b, c \in \{1, 2, \dots, n\}$,
        The cycle $(a\ b\ c)$ can be written as two transpositions of $a$:
        \begin{align*}
          (a\ b\ c) &= (a\ b)(b\ c) \\
        \end{align*}
        Essentially, every $3$-cycle is a product of two transpositions.
        Since every elements in $A_n$ can be written as a product of \textbf{even}
        transpositions, every element in $A_n$ can also
        be written as a product of the $3$-cycles constituting contiguous pairs of
        those transpositions.
      \end{Answer}
  \end{enumalph}
\end{problem}

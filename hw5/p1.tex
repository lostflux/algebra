\begin{problem}
  In this problem, we prove the main ingredients in the fourth isomorphism theorem.  
  Throughout, let $G$ be a group, let $N \trianglelefteq G$ be a normal subgroup.  
  Let $H \leq G$ be a subgroup.  
  \begin{enumalph}
    \item Define $H/N \colonequals \{hN : h \in H\}$.
      Show that $H/N \leq G/N$ is a subgroup.
      (So given a subgroup of $G$, we can make a subgroup of $G/N$.)
      \begin{Answer}
        \begin{enumalph}
          \item Since both $H$ and $N$ are subgroups of $G$, 
            they both contain the identity element of $G$, $\epsilon$.
            So the set $H/N = \{ hn \colon h \in H,\ n \in N \}$ contains $\epsilon$,
            particularly when $h = \epsilon$ and $n = \epsilon$.
          \item Let $h_1n_1 \in H/N$ and $h_2n_2 \in H/N$.
            Consider the product, $h_1n_1 \cdot h_2n_2$.
            Since elements of normal subgroups commute:
            \begin{align*}
              h_1n_1h_2n_2 &= (h_1h_2)(n_1n_2) \qquad \zaff{\text{($n_1h_2 = h_2n_1$)}} \\
              &= h_3n_3 \in H/N \qquad \zaff{\text{($h_3 = h_1h_2 \in H,\ n_3 = n_1n_2 \in N$)}} \\
            \end{align*}
            So $H/N$ is closed under multiplication.
          \item Let $h_1n_1 \in H/N$.
            Consider the inverse, $(h_1n_1)^{-1}$.
            Since elements of normal subgroups commute:
            \begin{align*}
              {(h_1n_1)}^{-1} &= n_1^{-1}h_1^{-1} \\
              &= h_1^{-1}n_1^{-1} \qquad \zaff{\text{($n_1^{-1}h_1^{-1} = h_1^{-1}n_1^{-1}$ since $n_1^{-1} \in N$)}} \\
              &= h_2n_2 \in H/N \qquad \zaff{\text{($h_2 = h_1^{-1} \in H,\ n_2 = n_1^{-1} \in N$)}} \\
            \end{align*}
            So $H/N$ is closed under inversion.
        \end{enumalph}
      \end{Answer}
    \newpage
    \item Show that $(HN)/N=H/N$.
      (So we get the same subgroups of $G/N$ by taking 
      those subgroups $H \leq G$ containing $N$.)
      \begin{Answer}
        Consider the map $\phi \colon H \to (HN)/N$ defined by
        $\phi(h) \colonequals hnN,\ n \in N$.
        Take the instance $h \in \ker(G)$, then $hn = \epsilon$,
        and $hnN = N$.
        Thus, we see that one of the cosets of $HN/H$ is $N$.
        Using this coset, we can define the image of $\phi$
        as a product of $H$ and $N$ --- that is, $H/N$.
      \end{Answer}
    \item If $\phi \colon G \to G'$ is a group homomorphism
      and $H' \leq G'$ is a subgroup,
      show that $\phi^{-1}(H')$ is a subgroup of $G$.
      Apply this to the map $\pi \colon G \to G/N$ to conclude that
      if $H' \leq G/N$ is a subgroup, then $\pi^{-1}(H') \leq G$
      is a subgroup of $G$ containing $N$.
      (So we can go backwards.)
      \emph{[Hint: recall that if $f \colon A \to B$ is a map
      and $Y \subseteq B$ is a subset, then the preimage is
      $f^{-1}(Y)=\{x \in A : f(x) \in Y\}$.
      The use of this symbol does not imply that $f$ has an inverse.]}
      \begin{Answer}
        \begin{enumalph}
          \item Let $\epsilon'$ be the identity element of $G'$.
            Then $\phi(\epsilon) = \epsilon' \in H'$.
            So $\phi^{-1}(\{\epsilon'\}) = \{\epsilon\}$.
            Therefore, $\phi^{-1}(H')$ contains the identity element of $G$.
          \item For $h_1, h_2 \in H'$, let $g_1 = \phi^{-1}(h_1)$ and $g_2 \in \phi^{-1}(h_2)$.
            Then $\phi(g_1) \in H'$ and $\phi(g_2) \in H'$.
            So $\phi(g_1g_2) = \phi(g_1)\phi(g_2) \in H'$.
            By definition of the isomorphism mapping elements in $G$ to elements in $H'$,
        \end{enumalph}
        Let $g \in \ker(\phi)$, then $\phi(g) = \epsilon$.
        So $\phi^{-1}(\epsilon) = \{g \in G : \phi(g) = \epsilon\} = \{g \in G : g = \epsilon\} = \{\epsilon\}$.
        Thus, $\phi^{-1}(H')$ is closed under the identity element.
        Let $g_1 \in \phi^{-1}(H')$ and $g_2 \in \phi^{-1}(H')$.
        Then $\phi(g_1) \in H'$ and $\phi(g_2) \in H'$.
        So $\phi(g_1g_2) = \phi(g_1)\phi(g_2) \in H'$.
        Thus, $\phi^{-1}(H')$ is closed under multiplication.
        Let $g \in \phi^{-1}(H')$.
        Then $\phi(g) \in H'$.
        So $\phi(g^{-1}) = (\phi(g))^{-1} \in H'$.
        Thus, $\phi^{-1}(H')$ is closed under inversion.
      \end{Answer}
    \item Show that if $H \trianglelefteq G$ is normal,
    then $H/N \trianglelefteq G/N$ is normal.
  \end{enumalph}
\end{problem}

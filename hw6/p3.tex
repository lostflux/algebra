\begin{problem}{(\textsf{DF 4.5.13})}
  \begin{enumalph}
    \item Prove that a group $G$ of order $\#G=21$ has a normal $7$-Sylow subgroup.
    \begin{Answer}
      $21 = 3 \cdot 7$.
      \begin{enumalph}
        \item $n_7 \mid 3$ and $n_7 \equiv 1 \pmod 7 \implies n_7 = 1$.
        
        \noindent
        In this case, the $7$-sylow subgroup will have order $3$.
        However, there is only one group of order $3$ up to isomorphism,
        that is $C_3$, which is abelian.

        \noindent
        Additionally, we know that the conjugates of a $p$-sylow subgroup
        are also $p$-sylow subgroups. In this case, there is only one
        $7$-sylow subgroup, $C_3$, and its conjugates must therefore
        be equal to itself. This implies that the subgroup is normal in $G$.
      \end{enumalph}
    \end{Answer}
    \item Prove that a group $G$ of order $\#G=56$ has a normal Sylow $p$-subgroup
      for some prime $p$ dividing its order.
      \emph{[Hint: count elements of order $7$, arguing that distinct
      $7$-Sylow subgroups only intersect in the identity.]}
    \begin{Answer}
      
      FIrst, Cauchy's theorem tells us that $G$ must have an element of order $7$,
      and an element of order $2$.
      Given $a \in G$ of order $74$, then the entire set generated by $a$,
      that is $\{ a^i \colon 0 \le i \le 6 \}$, will have order $7$ (since $7$ is prime).
      Thus, there are a total of $8$ elements of order $7$ in $G$.
      
      First, observe that $56 = 2^3 \cdot 7$.
      We know  $n_7 \mid 8$ and $n_7 \equiv 1 \pmod 7$, which implies that $n_7 = 1$.
      Thus, there is only one Sylow $7$-subgroup in $G$, and it must have $56/7 = 8$ elements.
      There are also only $8$ elements of order $56$ in $G$, so the Sylow $7$-subgroup
      must be the group generated by $a$ as defined above, and containing the powers
      of $a$ including the identity.

      As demonstrated in part (a) above, a singular Sylow p-subgroup must be normal
      in $G$, since its conjugates must also be Sylow p-subgroups, which in this case
      means its conjugates must be the group itself.
    \end{Answer}
  \end{enumalph}
\end{problem}

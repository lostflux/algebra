\begin{problem}{(\textsf{DF 4.5.10})}
  \begin{enumalph}
    \item Let $F$ be a field and let $n \geq 1$.
      Show that \[ \SL_n(F) \colonequals \{A \in \GL_n(F) : \det(A)=1\} \]
      is a normal subgroup of $\GL_n(F)$, called the \textsf{special linear group}.
      \emph{[Hint: kernel of determinant.]}

      \begin{Answer}
        \begin{enumalph}
          \item  $\SL_n(F) \subseteq \GL_n(F)$:
            
            \noindent
            By definition, $\det(A)=1$ for all $A \in \SL_n(F)$.

            \noindent
            Also, by definition, $\det(M) \neq 0$ for all $M \in \GL_n(F)$.

            \noindent
            Since $1 \neq 0$, it follows that  $\SL_n(F) \subseteq \GL_n(F)$.
          \item  $\SL_n(F)$ contains the identity of $\GL_n(F)$:
          
            \noindent
            Under matrix multiplication, the identity matrix
            in $GL_n(F)$ is the matrix
            \[I_n = \begin{pmatrix}
              1 & 0 & \cdots & 0 \\
              0 & 1 & \cdots & 0 \\
              \vdots & \vdots & \ddots & \vdots \\
              0 & 0 & \cdots & 1
            \end{pmatrix}\]
            where $0$ is the additive identity, and $1$ is the multiplicative identity.

            \noindent
            Since the determinant of a diagonal matrix is the product of its diagonal entries,
            in this case $\det(I_n) = 1 \cdot 1 \cdots 1 = 1$.

            \noindent
            Therefore, $I_n \in \SL_n(F)$.
          \item  $\SL_n(F)$ is closed under multiplication:
          
            \noindent
            The determinant of a product of matrices is the product of the determinants.

            \noindent
            Taking $A,B \in \SL_n(F)$, we have $\det(A) = 1$ and $\det(B) = 1$.

            \noindent
            Therefore, $\det(AB) = \det(A) \det(B) = 1 \cdot 1 = 1$ and $AB \in \SL_n(F)$.
          \item  $\SL_n(F)$ is closed under inversion:
        
            \noindent
            The determinant of the inverse of a matrix is the inverse of the determinant.

            \noindent
            Taking $A \in \SL_n(F)$, Consider $A^{-1}$ as the inverse of $A$.
            
            \noindent
            By matrix properties, $\det(A^{-1}) = \frac{1}{\det(A)}$.
            
            \noindent
            Since $\det(A) = 1$, $\det(A^{-1}) = 1$ and $A^{-1} \in \SL_n(F)$.
          
          \item  $\SL_n(F)$ is normal in $GL_n(F)$.
          
          For  $\SL_n(F)$ to be normal in $GL_n(F)$, its conjugates must be in  $\SL_n(F)$.
          
          Let $S \in \SL_n(F), \det(S) = 1$ and $G \in \GL_n(F), \det(G) = g \neq 0$.

          \noindent
          Consider the conjugate $GSG^{-1}$.
          
          \noindent
          By matrix properties, $\det(GSG^{-1}) = \det(G) \cdot \det(S) \cdot \det(G^{-1}) = g \cdot 1 \cdot \frac{1}{g} = 1$.

          \noindent
          Therefore, $GSG^{-1} \in \SL_n(F)$, and  $\SL_n(F)$ is normal in $GL_n(F)$.
          
        \end{enumalph}
      \end{Answer}
      
    \newpage
    \item Let $F=\Z/p\Z$ for $p$ prime.
      Let's write $\F_p \colonequals \Z/p\Z$ to remind ourselves that it is a field.
      Show that $\#\GL_2(\F_p)=(p^2-1)(p^2-p)$.
      \emph{[Hint: a matrix $A \in \M_2(\F_p)$ is invertible if and only if its columns 
      are linearly independent.  So the first column must be nonzero, and then \ldots ]}

      \begin{Answer}
        Consider the matrix \[A = \begin{pmatrix} a & b \\ c & d \end{pmatrix} \in \GL_2(\F_p) \subset M_2(\F_p)\]

        \noindent
        For the matrix to be invertible, its first column must be nonzero.
        
        First, observe that $\#(\F_p) = p$ (since $p$ is prime).
        If unrestricted, each position in $A$ could be filled up by $p$ different values.
        and the entire matrix could be filled by $p^4$ different \textit{combinations} of values
        from $\F_p$.
        In our case, $A$ must be invertible, so its first column must be nonzero.
        Let's consider the ordered pairs $(a,c)$ to be the first column
        and $(b,d)$ to be the second column.
        
        Then, $(a, c) \neq (0,0)$. This takes away one option for the first column,
        leaving $p^2 - 1$ options ($p$~options for $a$ and $p$ options for $c$
        multiply to $p^2$, take away $(0,0)$ to give $p^2 - 1$ remaining options).
        
        Next, the second column $(b,d)$ must not be a linear multiple of the first column $(a,c)$.
        There are $p$ multiples of any pair $(a,c)$, so there are $p$ options that are invalid
        for the pair $(b,d)$. This leaves $p^2 - p$ options.

        Therefore, the total number of distinct invertible matrices will be
        $(p^2 - 1)(p^2 - p)$.
      \end{Answer}
    \item Compute $\#\SL_2(\F_p)$ as a polynomial in $p$,
      then the numerical value for $p=2,3$.

      \begin{Answer}
        By the same reasoning above, we have $p^2 - 1$ options for the first column.

        However, once the pair $(a, c)$ has been fixed, we must now pick
        the pair $(b, d)$ such that $ad \times bc = 1$.

        \noindent
        We can pick any of $p$ options for $d$, but only a single option for $b$ thereafter.

        This gives a total of $\crim{p(p^2 - 1)}$ possible combinations for (and, therefore,
        elements in)  $\SL_n(\F_p)$.

        \begin{enumalph}
          \item $p = 2$:
            \[ p(p^2 - 1) = 2 \cdot 3 = 6\]
          \item $p = 3$:
            \[ p(p^2 - 1) = 3 \cdot 8 = 24\]
        \end{enumalph}
      \end{Answer}

    \item Show that the subgroup $P \leq \SL_2(\F_3)$ generated by
      $\begin{pmatrix} 0 & -1 \\ 1 & 0 \end{pmatrix}$
      and $\begin{pmatrix} 1 & 1 \\ 1 & -1 \end{pmatrix}$
      is isomorphic to $Q_8$.
      \emph{[Hint: map $i,j$ to the given generators; it's not too 
      many matrix multiplications, because $-1$.]}

      \begin{Answer}
        Let $P$ be the subgroup generated by the matrices
        \[ A = \begin{pmatrix} 0 & -1 \\ 1 & 0 \end{pmatrix} \text{ and }
          B = \begin{pmatrix} 1 & 1 \\ 1 & -1 \end{pmatrix} \text{ and }
          C = AB = \begin{pmatrix} -1 & 1 \\ 1 & 1 \end{pmatrix} \]
        
        \noindent
        Through matrix multiplication, we see that:
        \begin{align*}
          A \cdot A &= \begin{pmatrix} 0 & -1 \\ 1 & 0 \end{pmatrix}
          \cdot \begin{pmatrix} 0 & -1 \\ 1 & 0 \end{pmatrix}
          = \begin{pmatrix} -1 & 0 \\ 0 & -1 \end{pmatrix} = -I \\
          A \cdot B &= \begin{pmatrix} 0 & -1 \\ 1 & 0 \end{pmatrix}
          \cdot \begin{pmatrix} 1 & 1 \\ 1 & -1 \end{pmatrix}
          = \begin{pmatrix} -1 & 1 \\ 1 & 1 \end{pmatrix} = C \\
          A \cdot C &= \begin{pmatrix} 0 & -1 \\ 1 & 0 \end{pmatrix}
          \cdot \begin{pmatrix} -1 & 1 \\ 1 & 1 \end{pmatrix}
          = \begin{pmatrix} -1 & -1 \\ -1 & 1 \end{pmatrix} = -B \\
          B \cdot A &= \begin{pmatrix} 1 & 1 \\ 1 & -1 \end{pmatrix}
          \cdot \begin{pmatrix} 0 & -1 \\ 1 & 0 \end{pmatrix}
          = \begin{pmatrix} 1 & -1 \\ -1 & -1 \end{pmatrix} = -C \\
          B \cdot B &= \begin{pmatrix} 1 & 1 \\ 1 & -1 \end{pmatrix}
          \cdot \begin{pmatrix} 1 & 1 \\ 1 & -1 \end{pmatrix}
          = \begin{pmatrix} 2 & 0 \\ 0 & 2 \end{pmatrix}
          \equiv \begin{pmatrix} -1 & 0 \\ 0 & -1 \end{pmatrix} = -I \\
          B \cdot C &= \begin{pmatrix} 1 & 1 \\ 1 & -1 \end{pmatrix}
          \cdot \begin{pmatrix} -1 & 1 \\ 1 & 1 \end{pmatrix}
          = \begin{pmatrix} 0 & 2 \\ -2 & 0 \end{pmatrix}
          \equiv \begin{pmatrix} 0 & -1 \\ 1 & 0 \end{pmatrix} = A \\
          C \cdot A &= \begin{pmatrix} -1 & 1 \\ 1 & 1 \end{pmatrix}
          \cdot \begin{pmatrix} 0 & -1 \\ 1 & 0 \end{pmatrix}
          = \begin{pmatrix} 1 & 1 \\ 1 & -1 \end{pmatrix} = B \\
          C \cdot B &= \begin{pmatrix} -1 & 1 \\ 1 & 1 \end{pmatrix}
          \cdot \begin{pmatrix} 1 & 1 \\ 1 & -1 \end{pmatrix}
          = \begin{pmatrix} 0 & -2 \\ 2 & 0 \end{pmatrix}
          \equiv \begin{pmatrix} 0 & 1 \\ -1 & 0 \end{pmatrix} = -A \\
          C \cdot C &= \begin{pmatrix} -1 & 1 \\ 1 & 1 \end{pmatrix}
          \cdot \begin{pmatrix} -1 & 1 \\ 1 & 1 \end{pmatrix}
          = \begin{pmatrix} 2 & 0 \\ 0 & 2 \end{pmatrix}
          \equiv \begin{pmatrix} -1 & 0 \\ 0 & -1 \end{pmatrix} = -I
        \end{align*}

        \noindent
        First, notice that $P$ has $8$ elements generated by $A$ and $B$:
        \[P = \{\pm I, \pm A, \pm B, \pm C \colon C = AB\}\]\\
        Suppose $A \sim i, B \sim j, C \sim k$, and $I \sim 1$.
        Then:

        \begin{align*}
          i \cdot i &= -1 \\
          i \cdot j &= k \\
          i \cdot k &= -j \\
          j \cdot i &= -k \\
          j \cdot j &= -1 \\
          j \cdot k &= i \\
          k \cdot i &= j \\
          k \cdot j &= -i \\
          k \cdot k &= -1
        \end{align*}

      We see that $P$ under matrix multiplication is isomorphic to $Q_8$
      under multiplication.
        
      \end{Answer}
    \newpage
    \item Continuing part (d), show that $P$ is a Sylow $2$-subgroup of $\SL_2(\F_3)$,
      and conclude that $\SL_2(\F_3) \not\simeq S_4$.

      \begin{Answer}
        As we saw in part (c) above, $\SL_2(\F_3)$ has $24$ elements.

        \noindent
        $24 = 2^3 \cdot 3$, so $\SL_2(\F_3)$ has a Sylow $2$-subgroup.\\
        Precisely, $n_2 \mid 3$ and $n_2 \equiv 1 \pmod 2$, which implies that
        either $n_2 = 1$ or $n_2 = 3$. Furthermore, Sylow $2$-subgroups
        must have size $24/3 = 8$.
        Since $P$ has order $8$, it is a Sylow $2$-subgroup of $\SL_2(\F_3)$.

        \noindent
        Since we demonstrated in part (e) above that $P$ is isomorphic to $Q_8$,
        it must follow that $P$ is closed under conjugation!
        Therefore, \emph{$P$ must be the only Sylow $2$-subgroup of $\SL_2(\F_3)$}.
        
        \bigskip
        In contrast, $S_4$ has $24$ elements, and $16$ of those elements
        have an order that divides $8$. These are:
        \begin{align*}
          1 &\quad \text{(order $1$)} \\
          (1\ 2) &\quad \text{(order $2$)} \\
          (1\ 3) &\quad \text{(order $2$)} \\
          (1\ 4) &\quad \text{(order $2$)} \\
          (2\ 3) &\quad \text{(order $2$)} \\
          (2\ 4) &\quad \text{(order $2$)} \\
          (3\ 4) &\quad \text{(order $2$)} \\
          (1\ 2)(3\ 4) &\quad \text{(order $2$)} \\
          (1\ 3)(2\ 4) &\quad \text{(order $2$)} \\
          (1\ 4)(2\ 3) &\quad \text{(order $2$)} \\
          (1\ 2\ 3\ 4) &\quad \text{(order $4$)} \\
          (1\ 2\ 4\ 3) &\quad \text{(order $4$)} \\
          (1\ 3\ 2\ 4) &\quad \text{(order $4$)} \\
          (1\ 3\ 4\ 2) &\quad \text{(order $4$)} \\
          (1\ 4\ 2\ 3) &\quad \text{(order $4$)} \\
          (1\ 4\ 3\ 2) &\quad \text{(order $4$)}
        \end{align*}
        These elements must occur in the Sylow $2$-subgroups of $S_4$,
        yet each Sylow $2$-subgroup of $S_4$ must have size $8$.
        \crim{
          Therefore, $S_4$ cannot have a single Sylow $2$-subgroup --- it must have $3$,
          and $\SL_2(\F_3) \not\simeq S_4$ because $\SL_2(\F_3)$ has $1$ Sylow $2$-subgroup.
        }
      \end{Answer}
  \end{enumalph}
\end{problem}

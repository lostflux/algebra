\begin{problem}{(\textsf{DF 4.5.4})}

  \noindent
  For each of $G=D_{12}$ and $G=S_3 \times S_3$, do the following:
  \begin{enumalph}
    \item Exhibit all Sylow $p$-subgroups.
    \begin{Answer}
      \begin{enumalph}
        \item $D_{12}$
          
          \noindent
          $D_{12}$ has order $12 = 2^2 \cdot 3$.
          \begin{enumalph}
            \item $p=2$:
              $n_2 \mid 3$ and $n_2 \equiv 1 \pmod 2 \implies n_2 \in \{1, 3\}$.

              \noindent
              Since no singular element in $D_{12}$ has order $4$,
              we can find them as groups generated by
              pairs of order-$2$ elements.
              One pair of such elements is $r^3$ and $s$,
              generating the subgroup \[\{\eps, r^3, s, sr^3 \} \]

              \noindent
              The other subgroups can be discovered by conjugation, to give the full set:
              \begin{align*}
                \langle r^3, s \rangle &= \{\eps, r^3, s, sr^3 \} \\
                \langle r^3, sr \rangle &= \{\eps, r^3, sr, sr^4 \} \\
                \langle r^3, sr^2 \rangle &= \{\eps, r^3, sr^2, sr^5 \}
              \end{align*}
              \noindent
              Thus, $n_2 = 3$ and $D_{12}$ has $3$ unique $2$-Sylow subgroup,
              each having order $4$.
              All the $2$-Sylow subgroups are isomorphic to $V_4$.

            \item $p=3$:
              $n_3 \mid 2^2$ and $n_3 \equiv 1 \pmod 3 \implies n_3 \in \{1, 4\}$.

              \noindent
              Since $r^2 \in D_{12}$ has order $3$,
              We can generate a $3$-Sylow subgroup using $r^2$ as the generator.
              Thus, we have the subgroup \[\langle r^2 \rangle = \{\eps, r^2, r^4 \}\]

              By conjugation, we notice that $r^2$ is centralized by $r$,
              and the set $\langle r^2 \rangle$ is normalized by $s$.
              Therefore, there are no other $3$-Sylow subgroups.
              \noindent
              Thus, $n_3 = 1$ and $D_{12}$ has $1$ unique $3$-Sylow subgroup,
              each having order $3$.
          \end{enumalph}

        \item $S_3 \times S_3$
            
            \noindent
            $S_3 \times S_3$ has order $36 = 2^2 \cdot 3^2$.
            \begin{enumalph}
              \item $p=2$:
                $n_2 \mid 3^2$ and $n_2 \equiv 1 \pmod 2 \implies n_2 \in \{1, 9\}$.
                In this case, each subgroup will have $4$ elements.
                If a member of the group has order $4$, then it could be a generator
                for one of the subgroups. However, no such element can exist in
                $S_3 \times S_3$ (since $4 \nmid 6$, the order of $S_3$).
                
                \noindent
                Thus, we can find $2$-Sylow subgroups generated by pairs of elements
                with order $2$ each. One way of generating such elements is
                by taking the identity element in $S_3$ and pairing it with
                a $2$-cycle (or vice versa).

                The $2$-Sylow subgroups are:
                \begin{align}
                  \langle [\eps, (1\ 2)], [(1\ 2), \eps] \rangle &= \{[\eps, \eps], [\eps, (1\ 2)], [(1\ 2), \eps], [(1\ 2), (1\ 2)] \} \\
                  \langle [\eps, (1\ 2)], [(1\ 3), \eps] \rangle &= \{[\eps, \eps], [\eps, (1\ 2)], [(1\ 3), \eps], [(1\ 3), (1\ 2)] \} \\
                  \langle [\eps, (1\ 2)], [(2\ 3), \eps] \rangle &= \{[\eps, \eps], [\eps, (1\ 2)], [(2\ 3), \eps], [(2\ 3), (1\ 2)] \} \\
                  \langle [\eps, (1\ 3)], [(1\ 2), \eps] \rangle &= \{[\eps, \eps], [\eps, (1\ 3)], [(1\ 2), \eps], [(1\ 2), (1\ 3)] \} \\
                  \langle [\eps, (1\ 3)], [(1\ 3), \eps] \rangle &= \{[\eps, \eps], [\eps, (1\ 3)], [(1\ 3), \eps], [(1\ 3), (1\ 3)] \} \\
                  \langle [\eps, (1\ 3)], [(2\ 3), \eps] \rangle &= \{[\eps, \eps], [\eps, (1\ 3)], [(2\ 3), \eps], [(2\ 3), (1\ 3)] \} \\
                  \langle [\eps, (2\ 3)], [(1\ 2), \eps] \rangle &= \{[\eps, \eps], [\eps, (2\ 3)], [(1\ 2), \eps], [(1\ 2), (2\ 3)] \} \\
                  \langle [\eps, (2\ 3)], [(1\ 3), \eps] \rangle &= \{[\eps, \eps], [\eps, (2\ 3)], [(1\ 3), \eps], [(1\ 3), (2\ 3)] \} \\
                  \langle [\eps, (2\ 3)], [(2\ 3), \eps] \rangle &= \{[\eps, \eps], [\eps, (2\ 3)], [(2\ 3), \eps], [(2\ 3), (2\ 3)] \}
                \end{align}
  
              \item $p=3$:
                $n_3 \mid 2^2$ and $n_3 \equiv 1 \pmod 3 \implies n_3 \in \{1, 4\}$.
                In this case, each subgroup will have $9$ elements.
                We can generate a $3$-Sylow subgroup using an order-$3$
                element as a generator. The only such groups are $(1\ 2\ 3)$ and $(1\ 3\ 2)$.
                The $3$-Sylow subgroup is the group $G = \langle (1\ 2\ 3), (1\ 2\ 3) \rangle$.
                \begin{align*}
                  \Biggl\{\quad
                      &[\eps, \eps],
                      [\eps, (1\ 2\ 3)],
                      [\eps, (1\ 3\ 2)],\\
                      &[(1\ 2\ 3), \eps],
                      [(1\ 2\ 3), (1\ 2\ 3)],
                      [(1\ 2\ 3), (1\ 3\ 2)],\\
                      &[(1\ 3\ 2), \eps],
                      [(1\ 3\ 2), (1\ 2\ 3)],
                      [(1\ 3\ 2), (1\ 3\ 2)]
                  \quad \Biggl\}
              \end{align*}
              In this case, each generator in the direct product
              generates a group of order $3$, isomorphic to $Z_3$.
              The $3$-Sylow subgroup is isomorphic
              to $Z_3 \times Z_3$ and $C_3 \times C_3$.
            \end{enumalph}
      \end{enumalph}
    \end{Answer}
    \item Verify the conclusion of Sylow's theorem in each case (i.e., $n_p(G) \equiv 1
    \pmod{p}$ and $n_p(G) \mid m$).
    \item In each case where a $p$-Sylow subgroup $P \trianglelefteq G$ is normal, 
    describe the group $G/P$ up to isomorphism (i.e., what more recognizable group is 
    it isomorphic to?).  
    \end{enumalph}
\end{problem}

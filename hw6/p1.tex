\begin{problem}{(\textsf{DF 4.5.4})}

  \noindent
  For each of $G=D_{12}$ and $G=S_3 \times S_3$, do the following:
  \begin{enumalph}
    \item Exhibit all Sylow $p$-subgroups.
    \begin{Answer}
      \begin{enumalph}
        \item $D_{12}$
          
          \noindent
          $D_{12}$ has order $12 = 2^2 \cdot 3$.
          \begin{enumalph}

            \bigskip
            \item $p=2$: \\
              $n_2 \mid 3\ \; \land \;\ n_2 \equiv 1 \pmod 2 \implies n_2 \in \{1, 3\}$
              and each Sylow $2$-subgroup has order $12/3 = 4$.
              Since no singular element in $D_{12}$ has order $4$,
              let's consider the subgroups generated by pairs
              of elements of order $2$.
              we can find them as groups generated by
              pairs of order-$2$ elements.
              One such pair is $\{r^3, s\}$,
              generating the subgroup $\{\eps, r^3, s, sr^3 \}$.

              \noindent
              The other subgroups can be discovered by conjugation, to give the full set:
              \begin{align*}
                \langle r^3,\ s \rangle &= \{\eps, r^3, s, sr^3 \} \\
                \langle r^3,\ sr \rangle &= \{\eps, r^3, sr, sr^4 \} \\
                \langle r^3,\ sr^2 \rangle &= \{\eps, r^3, sr^2, sr^5 \}
              \end{align*}
              \noindent
              Thus, $n_2 = 3$ and $D_{12}$ has $3$ unique Sylow $2$-subgroups,
              each having order $4$.
              All the Sylow $2$-subgroups are isomorphic to $V_4$.

              \noindent
              By Sylow's theorem, we see that $3 \mid 3 \; \land \; 3 \equiv 1 \pmod 2$.

            \bigskip
            \item $p=3$:\\
              $n_3 \mid 2^2$ and $n_3 \equiv 1 \pmod 3 \implies n_3 \in \{1, 4\}$.

              \noindent
              Since $r^2 \in D_{12}$ has order $3$,
              We can generate a Sylow $3$-subgroup using $r^2$ as the generator.
              Thus, we have the subgroup \[\langle r^2 \rangle = \{\eps, r^2, r^4 \}\]

              Under conjugation, we notice that $r^2$ is centralized by $r$.
              We also notice that $r^2$ and $r^4$ are conjugate under all elements
              $sr^k, r \in \Z$.
              Take conjugation by $sr^2$ as an example:

              \begin{align*}
                sr^2 \cdot r^2 \cdot sr^2 &= sr^4 \cdot sr^2 \\
                &= s \cdot sr^{-4} \cdot r^2 \\
                &= r^{-2} \\
                &= r^4
              \end{align*}
              Therefore, the group generated by $r^2$ is normal in $D_{12}$.
              Since \textbf{all} Sylow $3$-subgroups are conjugates,
              and all conjugates of elements in $\langle r^2 \rangle$ are in
              $\langle r^2 \rangle$,
              it must be that $\langle r^2 \rangle$ is the only Sylow
              $3$-subgroup of $D_{12}$.\\
              Thus, $n_3 = 1$ and $D_{12}$ has $1$ unique Sylow $3$-subgroup,
              each having order $3$.
              
              \noindent
              By Sylow's theorem, we see that $1 \mid 4\ \; \land \;\ 1 \equiv 1 \pmod 3$.
          \end{enumalph}

        \bigskip
        \item $S_3 \times S_3$
            
            \noindent
            $S_3 \times S_3$ has order $36 = 2^2 \cdot 3^2$.
            \begin{enumalph}
              \bigskip
              \item $p=2$:\\
                $n_2 \mid 3^2 \; \land \; n_2 \equiv 1 \pmod 2 \implies n_2 \in \{1, 9\}$
                and each Sylow $2$-subgroup has order $36 / 9 = 4$.
                
                \noindent
                Since $4 \nmid 6$, no order $4$ element exists in $S_3$.
                Since the order of direct products is equivalent to the
                $lcm$ of the orders of the direct group, there therefore cannot exist
                an element of order $4$ in $S_3 \times S_3$ (that would have been
                a potential generator for a Sylow $2$-subgroup).
                Next, we can consider the groups generated by pairs of elements of order $2$.
                In $S_3  \times S_3$, such elements will be pairs constituted of
                the identity element and a transposition.

                The Sylow $2$-subgroups are:
                \begin{align}
                  \langle [\eps, (1\ 2)], [(1\ 2), \eps] \rangle &= \{[\eps, \eps], [\eps, (1\ 2)], [(1\ 2), \eps], [(1\ 2), (1\ 2)] \} \\
                  \langle [\eps, (1\ 2)], [(1\ 3), \eps] \rangle &= \{[\eps, \eps], [\eps, (1\ 2)], [(1\ 3), \eps], [(1\ 3), (1\ 2)] \} \\
                  \langle [\eps, (1\ 2)], [(2\ 3), \eps] \rangle &= \{[\eps, \eps], [\eps, (1\ 2)], [(2\ 3), \eps], [(2\ 3), (1\ 2)] \} \\
                  \langle [\eps, (1\ 3)], [(1\ 2), \eps] \rangle &= \{[\eps, \eps], [\eps, (1\ 3)], [(1\ 2), \eps], [(1\ 2), (1\ 3)] \} \\
                  \langle [\eps, (1\ 3)], [(1\ 3), \eps] \rangle &= \{[\eps, \eps], [\eps, (1\ 3)], [(1\ 3), \eps], [(1\ 3), (1\ 3)] \} \\
                  \langle [\eps, (1\ 3)], [(2\ 3), \eps] \rangle &= \{[\eps, \eps], [\eps, (1\ 3)], [(2\ 3), \eps], [(2\ 3), (1\ 3)] \} \\
                  \langle [\eps, (2\ 3)], [(1\ 2), \eps] \rangle &= \{[\eps, \eps], [\eps, (2\ 3)], [(1\ 2), \eps], [(1\ 2), (2\ 3)] \} \\
                  \langle [\eps, (2\ 3)], [(1\ 3), \eps] \rangle &= \{[\eps, \eps], [\eps, (2\ 3)], [(1\ 3), \eps], [(1\ 3), (2\ 3)] \} \\
                  \langle [\eps, (2\ 3)], [(2\ 3), \eps] \rangle &= \{[\eps, \eps], [\eps, (2\ 3)], [(2\ 3), \eps], [(2\ 3), (2\ 3)] \}
                \end{align}

                We can confirm that these subgroups are conjugate to each other,
                but the entire set is closed under conjugation.
                Therefore, there are $9$ Sylow $2$-subgroups in $S_3 \times S_3$.
                
                \noindent
                By Sylow's theorem, we see that $9 \mid 9 \; \land \; 9 \equiv 0 \pmod 2$.
  
              \bigskip
              \item $p=3$:
                $n_3 \mid 2^2$ and $n_3 \equiv 1 \pmod 3 \implies n_3 \in \{1, 4\}$.
                In this case, each subgroup will have $9$ elements.
                We can generate a Sylow $3$-subgroup using an order-$3$
                element as a generator. The only such groups are $(1\ 2\ 3)$ and $(1\ 3\ 2)$.
                The Sylow $3$-subgroup is the group $G = \langle (1\ 2\ 3), (1\ 2\ 3) \rangle$.
                \begin{align*}
                  \Biggl\{\quad
                      &[\eps, \eps],
                      [\eps, (1\ 2\ 3)],
                      [\eps, (1\ 3\ 2)],\\
                      &[(1\ 2\ 3), \eps],
                      [(1\ 2\ 3), (1\ 2\ 3)],
                      [(1\ 2\ 3), (1\ 3\ 2)],\\
                      &[(1\ 3\ 2), \eps],
                      [(1\ 3\ 2), (1\ 2\ 3)],
                      [(1\ 3\ 2), (1\ 3\ 2)]
                  \quad \Biggl\}
              \end{align*}
              In this case, each generator in the direct product
              generates a group of order $3$, isomorphic to $Z_3$.
              The Sylow $3$-subgroup is isomorphic
              to $Z_3 \times Z_3$ and $C_3 \times C_3$.

              \noindent
              By Sylow's theorem, we see that $4 \mid 4 \; \land \; 4 \equiv 1 \pmod 3$.
            \end{enumalph}
      \end{enumalph}
    \end{Answer}
    \item Verify the conclusion of Sylow's theorem in each case (i.e., $n_p(G) \equiv 1
    \pmod{p}$ and $n_p(G) \mid m$).
    \begin{Answer}
      \crim{Demonstrated above.}
    \end{Answer}
    \item In each case where a Sylow $p$-subgroup $P \trianglelefteq G$ is normal, 
    describe the group $G/P$ up to isomorphism (i.e., what more recognizable group is 
    it isomorphic to?).
    \begin{Answer}
      \crim{Demonstrated above.}
    \end{Answer}
    \end{enumalph}
\end{problem}

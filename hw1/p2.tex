
\begin{problem}{\textsf{(sorta DF 1.1.1, 1.1.8)}}
  Determine which of the following are groups.  Justify your answer.
  \begin{enumalph}
    \item The set $G=\R\setminus \{0\}$ under the binary operation $*$ defined by 
    $a*b=a/b$ for $a,b \in G$.
    \begin{Answer}
      $G$ is not a group.
      \begin{enumalph}
        \item \crim{$G$ is not closed under the operation $*$,
        since $a*b=a/b$ gives elements in $\Q$
        for any $a, b \in \R$ such that $b\ \nmid\ a$.}
        \item \crim{$G$ is not associative, since $a*(b*c)=a/(b/c) = ac/b \neq a/bc = (a/b)/c = (a * b)*c$}
        \item \crim{$G$ does not have an identity element,
        since:
        \begin{align*}
          e * a &= e/a = a \implies e = a^{2} \\
          a * e &= a/e = a \implies e = 1
        \end{align*}
        $e/a = a/e = a\quad \implies\quad e = a^2 \wedge e = 1\quad \implies\quad a=\pm 1$.}

        There is no unique identity unique identity that leaves \textbf{all} elements in $G$ invariant.
        \item However, \textbf{if we invented the identity to be $1$},
        then each element in $G$ would be it's own inverse since:
        \begin{align*}
          \forall a \in G \colon \qquad
          a / a^{-1} = a^{-1} / a = 1 \quad\implies \quad a = a^{-1} \\
        \end{align*}
      
      \end{enumalph}
    \end{Answer}
    \item The set $G=\R$ under the binary operation $*$ defined by $a*b=a+b+ab$ for 
    $a,b\in G$.
    \begin{Answer}
      $G$ is not a group.
      \begin{enumalph}
        \item $G$ has a unique identity $e = 0$.
        \item $G$ is closed under $*$.
        \item \crim{$*$ is not an associative operation on $G$.
        \begin{align*}
          a * (b * c) &= a * (b + c + bc) = a * b + a * c + a * bc \\
          &= a + b + ab + a + c + ac + a + bc + abc \\
          &= 3a + b + c + ab + ac + bc + abc \\
          \\
          (a * b) * c &= (a + b + ab) * c = a * c + b * c + ab * c \\
          &= a + c + ac + b + c + bc + ab + c + abc \\
          &= a + b + 3c + ab + ac + bc + abc \\
        \end{align*}
        As we can see, $a * (b * c) \neq (a * b) * c$.}
      \end{enumalph}
    \end{Answer}
    \newpage
    \item The set $G=\{z \in \C:z^n=1\text{ for some $n \in \Z_{>0}$}\}$ under 
    multiplication.  \emph{[Hint: be sure to check multiplication is a binary operation
    on $G$ in the first place!]}
    \begin{Answer}
      $G$ is a group.
      \begin{enumalph}
        \item $G$ has a unique identity $e = 1 + 0i = 1$, 
        since $1$ leaves all elements in the group invariant after multiplication.
        \item $G$ is closed under multiplication.
        
        \noindent
        $\forall z_{1}, z_{2} \in G, \exists n_{1}, n_{2} \in \Z_{>0} \colon$
        
        \begin{align*}
          z_{1}*z_{2} &= z_{1}z_{2}\\
        \end{align*}

        \noindent
        We need to find an exponent $n$ such that $(z_1z_2)^n = 1$.
        
        Taking $n = n_1n_2$ gives:
        \begin{align*}
          (z_1z_2)^{n_1n_2} &= z_1^{n_1n_2}z_2^{n_1n_2} \\
          &= {(z_1^{n_1})}^{n_2}{(z_2^{n_2})}^{n_{1}} \\
          &= 1^{n_2}1^{n_1} \\
          &= 1
        \end{align*}
        Therefore, it holds that multiplication of $2$ elements in $G$ gives an element in $G$.
        \item $*$ is associative following from associativity of $\C$.
        \item $G$ has inverses. By definition, if $z \in G$ then $z^n = 1$ for some $n \in \Z_{>0}$.
        
        \noindent
        The multiplicative inverse of $z$ is $1/z$, and $z^{n} = 1$ implies ${(1/z)}^{n} = 1$.
      \end{enumalph}
    \end{Answer}
  \end{enumalph}
\end{problem}
  
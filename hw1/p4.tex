\begin{problem}{\textsf{(sorta DF 1.2.2--1.2.5)}}
  Let $D_{2n}=\{1,r,\dots,r^{n-1},s,sr,\dots,sr^{n-1}\}$ be the dihedral group of 
  order $2n$ with presentation $D_{2n}=\langle r,s \mid r^n=s^2=1, rs=sr^{-1} \
  rangle$.
  \begin{enumalph}
  \item Write out the multiplication (Cayley) table for $D_6$.
  \begin{Answer}
    \begin{center}
      \begin{tabular}{ |c||c|c|c|c|c|c|| } 
       \hline
        & $1$ & $r$ & $r^2$ & $s$ & $sr$ & $sr^2$\\
       \hline
       \hline
        $1$ & $1$ & $r$ & $r^2$ & $s$ & $sr$ & $sr^2$\\
        \hline
        $r$ & $r$ & $r^2$ & $1$ & $sr$ & $sr^2$ & $s$\\
        \hline
        $r^2$ & $r^2$ & $1$ & $r$ & $sr^2$ & $s$ & $sr$\\
        \hline{}
        $s$ & $s$ & $sr^2$ & $sr$ & $1$ & $r^2$ & $r$\\
        \hline{}
        $sr$ & $sr$ & $s$ & $sr^2$ & $r$ & $1$ & $r^2$\\
        \hline
        $sr^2$ & $sr^2$ & $sr$ & $s$ & $r^2$ & $r$ & $1$\\
       \hline
      \end{tabular}
      \end{center}
  \end{Answer}
  \item Show that if $x \in D_{2n}$ is a power of $r$ (including $x=r^0=1$!), then 
  $rx=xr$ and $x$ has order at most $n$.
  \begin{Answer}
    Suppose $x = r^k$ for some $k \in \Z$.  Then $rx = rr^{k} = r^{k+1} = r^{k}r = xr$.

    \noindent
    Furthermore, we know that $r^{n} = 1$ in $D_{2n}$.

    Therefore:
    \begin{align*}
      x &= r^k \\
      x^{n} &= {(r^{k})}^{n} \\
      x^{n} &= r^{kn} \qquad \text{(multiplication of exponents for nested exponentiation)}\\
      x^{n} &= {(r^{n})}^{k} \qquad \text{(rearranging the powers)}\\
      \However r^{{n}} &= 1\\
      \therefore x^{n} &= 1^k = 1
    \end{align*}
    Therefore, for any element $x \in D_{2n}$ such that $x = r^k$ for some $k \in \Z$,
    we see that $x^{n} = 1$ and $x$ has order at most $n$.
    
    \noindent
    \textbf{The order can be lower, if $k\ >\ 1$ and $k\ |\ n$}
  \end{Answer}
  \item Otherwise, if $x \in \{s,sr,\dots,sr^{n-1}\}$ (not a power of $r$), then show
  that $rx=xr^{-1}$ and $x$ has order $2$.  \emph{[Hint: first prove by induction 
  that $r^m s = s r^{-m}$ for all $m \geq 1$.]}
  \begin{Answer}
    We aim to prove that $r^m s = s r^{-m}$ for all $m \geq 1$.

    \noindent
    \textbf{Base case:} We aim to show that $r^m s = sr^{-m}$ for $m = 1$.

    \begin{align*}
      rs &= sr\\
    \end{align*}
    The proof is trivial. Since $s$ is a flip, it translates a rotation done in any direction (counter-clockwise, in this case)
    \textit{before the flip} into equivalent rotations done in the opposite direction (clockwise) \textit{after the flip}. 
    
    \bigskip
    \noindent
    \textbf{Induction hypothesis:} incrementally, we aim to show that $r^{m}s = sr^{-m}$
    if $r^{m-1}s = sr^{-(m-1)}$.

    \noindent
    Let $m = i + 1$, assuming it has been proven that $r^{i}s = sr^{-i}$.
    \begin{align*}
      r^i s &= sr^{-i} \\
      r(r^i s) &= r(sr^{-i}) \\ 
      r^{(i+1)}s &= rsr^{-i} \qquad \text{(Simplifying the left side of the equation)} \\
      r^{(i+1)}s &= (rs)r^{-i} \qquad \text{(Grouping the right side of the equation)}\\
      r^{(i+1)}s &= (sr^{-1})r^{-i}\\
      r^{(i+1)}s &= sr^{-i-1} \qquad \text{(Simplifying the left side of the equation)}\\
      r^{(i+1)}s &= sr^{-(i+1)}\\
    \end{align*}

    \bigskip
    \noindent
    We now aim to show that if $x = r^m s$ for some $m \in \Z$ then $x$ has order $2$, i.e. $x^2 = 1$.
    \begin{align*}
      x^2 &= {(r^m s)}^2 \\
      x^{2} &= (r^m s)(s r^{-m}) \qquad \text{(Substituting the equality)}\\
      x^{2} &= r^m s^2 r^{-m} \qquad \text{(Expanding the right side of the equation)}\\
      x^{2} &= r^m r^{-m} \qquad \text{(Simplifying $s^{2} = 1$)}\\
      x^{2} &= r^{m - m} = r^{0} = 1\\
    \end{align*}
  \end{Answer}
  % \newpage
  \item For a group $G$ under $*$, we say that $a$ is \emph{central} if $a * x = x * 
  a$ for all $x \in G$.  Show that if $n=2k$ is even that $z=r^k$ is an element of 
  order $2$ which is central.
  \begin{Answer}
    \begin{enumalph}
      \item$z = r^k$ is an element of order $2$
      
      Given $z = r^k$, then $z^{2} = {(r^k)}^{2} = r^{2k}$.

      \noindent
      Furthermore, we know that $n = 2k$, which means the Dihedral group is
       $D_{4k}$.  It then follows from the Group axioms that $r^{2k} = 1$ in $D_{4k}$.

       \noindent
       Thus, $(z^{k})^2 = 1$ and $z$ has order $2$.

       \item $z$ is central
      
       \noindent
        We aim to show that $z * x = x * z$ for all $x \in D_{4k}$.
      
        \noindent
        Given $z = r^{k} = r^{n/2}$:

        \begin{align*}
          z * x &= r^{(n/2)} * x \\
          {(zx)}^{2} &= (r^{n/2} x)^{2} \qquad \text{(Squaring both sides)} \\
          z^2x^2 &= r^n x^2 \\
          z^2x^2 &= x^2\\
          zx &= x\\
          \\
          x * z &= x r^{(n/2)} \\
          {(xz)}^{2} &= {(xr^{n/2})}^{2} \qquad \text{(Squaring both sides)} \\
          {xz}^2 &= x^2 r^n \\
          {xz}^2 &= x^2\\
          xz &= x\\
        \end{align*}
        Thus, we see that $zx = xz = x$ for all $x \in D_{4k}$.

    \end{enumalph}
  \end{Answer}
  \end{enumalph}
\end{problem}

\author{\Large{Amittai Siavava \and John Voight}}
\documentclass[11pt, reqno]{amsart}

% Include the macros file from `../common'
% ENCODING
\usepackage[utf8]{inputenc}
\usepackage{pmboxdraw}
% \usepackage{pmboxdraw-extras}

% text alignment
\usepackage{ragged2e}

% URLs
\usepackage{hyperref}

% Sets
\newcommand{\FF}{\mathbb{F}}
\newcommand{\NN}{\mathbb{N}}
\newcommand{\QQ}{\mathbb{Q}}
\newcommand{\RR}{\mathbb{R}}
\newcommand{\ZZ}{\mathbb{Z}}

\newcommand{\F}{\mathbb{F}}
\newcommand{\N}{\mathbb{N}}
\newcommand{\Q}{\mathbb{Q}}
\newcommand{\R}{\mathbb{R}}
\newcommand{\Z}{\mathbb{Z}}
\renewcommand{\b}{\{0,1\}}

% More Math operators
\newcommand{\GL}{\mathrm{GL}}
\newcommand{\calC}{\mathcal{C}}
\newcommand{\calP}{\mathcal{P}}
\newcommand{\calF}{\mathcal{F}}
\newcommand{\nequiv}{\not\equiv}
\renewcommand{\notin}{\not\in}

% Functions
\usepackage{amsmath}
\DeclareMathOperator{\cost}{cost}
\DeclareMathOperator{\len}{len}
\DeclareMathOperator{\rank}{rank}
\DeclareMathOperator{\sgn}{sgn}
\DeclareMathOperator{\wt}{wt}

% Combinatorics

% Probability
\newcommand{\Hd}{\texttt{\color{BrickRed}H}}
\newcommand{\Tl}{\texttt{T}}
\newcommand{\EE}{\mathop{\mathbb{E}}}
\newcommand{\PP}{\mathop{\mathbb{P}}}
\newcommand{\indic}{\mathbb{1}}
\DeclareMathOperator{\Var}{Var}

% End of proof marker
\newcommand{\qedblack}{\hfill\ensuremath{\blacksquare}}
\newcommand{\qedwhite}{\hfill\ensuremath{\square}}

% Caligraphic caps
\newcommand{\cA}{\mathcal{A}}
\newcommand{\cB}{\mathcal{B}}
\newcommand{\cC}{\mathcal{C}}
\newcommand{\cD}{\mathcal{D}}
\newcommand{\cE}{\mathcal{E}}
\newcommand{\cF}{\mathcal{F}}
\newcommand{\cI}{\mathcal{I}}
\newcommand{\cL}{\mathcal{L}}
\newcommand{\cP}{\mathcal{P}}
\newcommand{\cS}{\mathcal{S}}
\newcommand{\cT}{\mathcal{T}}
\newcommand{\cU}{\mathcal{U}}
\newcommand{\cX}{\mathcal{X}}
\newcommand{\cY}{\mathcal{Y}}
\newcommand{\cZ}{\mathcal{Z}}

% Boldface letters
\newcommand{\ba}{\mathbf{a}}
\newcommand{\bb}{\mathbf{b}}
\newcommand{\bp}{\mathbf{p}}
\newcommand{\bq}{\mathbf{q}}
\newcommand{\br}{\mathbf{r}}
\newcommand{\bu}{\mathbf{u}}
\newcommand{\bv}{\mathbf{v}}
\newcommand{\bx}{\mathbf{x}}
\newcommand{\by}{\mathbf{y}}
\newcommand{\bz}{\mathbf{z}}
\newcommand{\tbp}{\mathbf{\widetilde{p}}}

% Special math symbols
\newcommand{\eps}{\varepsilon}
\newcommand{\ceq}{\subseteq}
\newcommand{\ang}[1]{\langle{} #1 \rangle}
\newcommand{\ceil}[1]{\lceil{} #1 \rceil}
\newcommand{\floor}[1]{\lfloor{} #1 \rfloor}

% Problem names and other small-caps constants
\newcommand{\inv}{\textsc{inv}\xspace}

% Useful for marking steps of a derivation to explain later
\newcommand{\circled}[1]{\raisebox{.5pt}{\textcircled{\raisebox{-.1pt}{\scriptsize #1}}}}


% Page size and margins
% \usepackage[left=1in,right=1in,top=1.3in,bottom=1.3in,nofoot]{geometry}
\usepackage{fancyhdr}   % for fancy header
\usepackage{fancyvrb}   % for fancy verbatim
\usepackage{graphicx}   % for including images
\usepackage{enumerate}  % for enumerating lists
\usepackage{enumitem}
\usepackage[rgb, dvipsnames]{xcolor}
\usepackage{tcolorbox}
% \newtcolorbox{codebox}[1]{
%   box align=top,
%   colback=white!5!white,
%   colframe=white!75!black,
%   title=#1
% }
\usepackage{multicol}
% \usepackage{accode}

% My Problem set macros

% Answer BOX
% \usepackage{xcolor}
\usepackage{microtype}
\usepackage{mdframed}
\newmdenv[%
  leftmargin=-5pt,
  rightmargin=-5pt, 
  innerleftmargin=5pt,
  innerrightmargin=5pt,
  backgroundcolor=black!10
]{Answer}%


% Header BOX
\newcommand{\handout}[6]{
  \noindent
  \begin{center}
  \setlength{\fboxrule}{1.2pt}
  \framebox{
    \vbox{
      \hbox to 5.78in { \textbf{#6} \hfill {\bf #2} }
      \vspace{4mm}
      \hbox to 5.78in { {\Large \hfill {\textbf{ #5 }}  \hfill} }
      \vspace{2mm}
      \hbox to 5.78in { {\textit{\textbf{#3 \hfill #4}}} }
    }
  }
  \setlength{\fboxrule}{0.2pt}
  \end{center}
  \vspace*{4mm}
}

% Header BOX
\newcommand{\PSET}[5]{\handout{#1}{#2}{Prof.\ #3}{Student: #4}{PSET #1}{#5}}

% Credit Statement
\newcommand{\CreditStatement}[1]{
  \noindent
  \begin{center} {
    \bf Credit Statement
  }
  \end{center}
  { #1 }
}

% Problem counter
\newenvironment{problem}[1][]%
{%
\stepcounter{problem} \vspace{.2cm} \noindent {\bf \arabic{problem}.} {\textit{#1}}~%
}{%
\vspace{.2cm}%
}

% Package Imports
\usepackage{amssymb,amsthm,amsmath,amstext}
\usepackage{mathdots} % for \dots
  % \dotsc -- dots with commas.
  % \dotsb -- dots with binary operators.
  % \dotsm -- multiplication dots.
  % \dotsi -- dots with integrals.
  % \dotso -- "other dots".
\usepackage{wasysym, stackengine, makebox, tikz-cd}
\newcommand\isom{\mathrel{\stackon[-0.1ex]{\makebox*{\scalebox{1.08}{\AC}}{=\hfill\llap{=}}}{{\AC}}}}
\newcommand\nvisom{\rotatebox[origin=cc] {-90}{$ \isom $}}
\newcommand\visom{\rotatebox[origin=cc] {90} {$ \isom $}}


% Custom colors
\definecolor{crimson}{rgb}{0.86, 0.08, 0.24}
\definecolor{teal}{rgb}{0.0, 0.5, 0.5}
\definecolor{zaffre}{rgb}{0.0, 0.08, 0.66}
\newcommand{\crim}{\textcolor{crimson}}
\newcommand{\teal}{\textcolor{teal}}
\newcommand{\zaff}{\textcolor{zaffre}}
\newcommand{\black}{\textcolor{black}}
\definecolor{DarkOliveGreen}{rgb}{0.33, 0.42, 0.18}
\newcommand{\darkgreen}{\textcolor{DarkOliveGreen}}
\newcommand{\green}{\textcolor{OliveGreen}}

\newcommand{\id}{\mathbf{id}\;}

% matrices -- vertical separators
\makeatletter
\renewcommand*\env@matrix[1][*\c@MaxMatrixCols{ c}]{%
  \hskip -\arraycolsep{}
  \let\@ifnextchar\new@ifnextchar{}
  \array{#1}}
\makeatother

% \usepackage{accode}
\usepackage{tikz}

% long multiplications
\usepackage{xlop}

% custom functions.
\newcommand{\lcm}[2]{\mathbf{lcm}\;(#1,\;#2)}
\newcommand{\Therefore}{\dot{.\hspace{.095in}.}\hspace{.095in}}
\newcommand{\However}{\dot{}\hspace{.045in}.\hspace{.045in} \dot{}\hspace{.095in}}

% resume includes
\usepackage[utf8]{inputenc}
\usepackage[full]{textcomp}
\usepackage{CJKutf8}
\usepackage[lf]{ebgaramond}

\usepackage[OT1]{fontenc}
\usepackage{enumitem}
\usepackage[scale=.75]{geometry}
\usepackage{url}
% \usepackage[dvipsnames]{xcolor}
% package settings
% \usepackage[
%   hidelinks,
%   pdfnewwindow=true,
%   pdfauthor={Amittai},
%   pdftitle={resume},
% ]{hyperref}

\pagestyle{headings}
% \markright{siavava}

\setlength\parindent{2em}

\thispagestyle{empty}

\newcommand{\cvsubsection}[1]{\subsection*{\hspace{1.45em}#1}}


\pagestyle{fancy}                       % fancy (allow headers, footers)
\fancyhf{}                              % clear all header/footer settings.
\cfoot{\thepage}                        % set page-numbers in footer.
\lhead{\textit{\textbf{ Amittai, S}}}   % set name in header, left.
\rhead{\textsc{Math 71: Algebra}}       % set class name in header, right.

\newcounter{problem}
\setcounter{problem}{0}

\renewcommand{\theenumi}{\alph{enumi}}

\begin{document}
\setlength{\headheight}{13.0pt}
\setlength{\footskip}{13.0pt}


% TITLE
\PSET{1 --- \today}{Fall 2022}{Prof.\ Voight}{Amittai Siavava}{Math 71: Algebra}

%CREDIT STATEMENT
\CreditStatement{
  I worked on these problems alone,
  with reference to class notes and the following books:
  \begin{enumerate}
    \item \textit{\textbf{Abstract Algebra}} by \textbf {David S. Dummit \& Richard M. Foote}.
    \item \textit{\textbf{Algebra}} by \textbf {Jacob K. Goldhaber \& Gertrude Ehrlich}
  \end{enumerate}
}

\bigskip

% \title{Cayley's theorem}
\date{\today}

\maketitle

\emph{
  [This is an example of a FLEX paper for Math 71, Fall 2022.
  It is meant as a sample so you can see how you might approach the paper,
  a \LaTeX\ template to help you to get started (if you want to),
  and to provide some general indications of what this might look like.
  Your paper might have a different structure, length, approach, voice, and composition.]
}

\section{Introduction}\label{sec:intro}

When groups are first introduced, they are described as sets equipped with a binary operation
satisfying certain axioms (associativity, identity, inverses).
As examples, we considered the symmetric groups $S_n$, the group of permutations of the set $\{1,\dots,n\}$.  However, these are not so far apart.  Even for Galois (see Dummit--Foote~\cite[p.~14 (3)]{DummitFoote}), groups were made of ``substitutions''---i.e., Galois was working with permutation groups!  

For now, we restrict attention to \emph{finite} groups (but see section~\ref{sec:conc} below for infinite groups).  So the symmetric groups $S_n$ (for $n \geq 1$) are finite groups.  Is every finite a group a permutation group?  No: we have $\#S_n=n!$, so a group of order $4$ cannot be isomorphic to $S_n$ since $2! < 4 < 3!$.  But if we all ourselves \emph{subgroups} of permutation groups, the answer is yes.  Our main result is as follows.

\begin{thm}[Cayley] \label{thm:cayley}
Every finite group is isomorphic to a subgroup of $S_n$ for some $n \geq 1$.
\end{thm}

\subsection*{Contents} In section \ref{sec:setup}, we get set up by describing how the group operation naturally describes permutations of the elements of the group.  We then prove Cayley's theorem in section \ref{sec:thm}.  We then conclude in section \ref{sec:conc} with some applications and next steps.

\section{Setup}~\label{sec:setup}

We start with a motivating example.  Recall the Cayley table for $D_6$, the dihedral group of order $6$:
\[ \begin{array}{c  cccccccc} 
& 1 & r & r^2 & s & sr & sr^2 \\
\toprule
1 & 1 & r & r^2 & s & sr & sr^2 \\
r & r & r^2 & 1 & sr^2 & s & sr \\
r^2 & r^2 & 1 & r & sr & sr^2 & s \\
s & s & sr & sr^2 & 1 & r & r^2 \\
sr & sr & sr^2 & s & r^2 & 1 & r \\
sr^2 & sr^2 & s & sr & r & r^2 & 1
\end{array} \]

We observed that Cayley table have the Sudoku property, as in the following lemma.

\begin{lem}
Each row (and column) of the Cayley table of a finite group $G$ contains all elements of $G$.
\end{lem}

As a reminder, this lemma follows directly from the cancellation law.

Returning to the above example, if we pick off just one row---say the row $sr$---by this property we get a permutation of the set $D_6$:
\begin{equation}  \label{eqn:sr}
\begin{pmatrix}
1 & r & r^2 & s & sr & sr^2 \\
sr & sr^2 & s & r^2 & 1 & r 
\end{pmatrix} 
\end{equation}
We denote this element $\sigma_{sr} \colon D_6 \to D_6$, since it is a symmetry that depends on $sr$: it is defined by $\sigma_{sr}(sr) = 1$, \dots, $\sigma_{sr}(sr^2)=r$, reading the input from the top row of the table and the output from the bottom row.  This is visibly a bijection from $D_6$ to itself: each element of $D_6$ appears exactly once.  

Recall that we write the set of bijections from a set $A$ to itself as
\[ \Sym(A) \colonequals \{\sigma \colon A \to A \textup{ bijection}\} \]
and this forms a group under composition.  This works for every set $A$, even though we mostly worked with $A=\{1,\dots,n\}$ and then abbreviate $S_n=\Sym(\{1,\dots,n\})$.  There is no loss of generality here.

\begin{lem}~\label{lem:An}
If $A$ is a finite set with $\#A=n$, then the groups $\Sym(A) \simeq S_n$ are isomorphic.
\end{lem}

\begin{proof}
Since $\#A=n$, there is a bijection from $A$ to $\{1,\dots,n\}$.  Each permutation of the elements of $A$ gives a permutation of the elements $\{1,\dots,n\}$ by how they are numbered.  
\end{proof}

Putting these together, we can define a function
\begin{equation}~\label{eqn:D3map}
\begin{aligned}
\sigma \colon D_6 &\to \Sym(D_6) \simeq S_6 \\
1 &\mapsto \sigma_1 = \begin{pmatrix}
1 & r & r^2 & s & sr & sr^2 \\
1 & r & r^2 & s & sr & sr^2
\end{pmatrix} \\
&\vdots \\
sr^2 &\mapsto \sigma_{sr^2} = \begin{pmatrix}
1 & r & r^2 & s & sr & sr^2 \\
sr^2 & s & sr & r & r^2 & 1
\end{pmatrix} 
\end{aligned}
\end{equation}
Usually we write functions like $f(x)$ with input $x$ from the domain; but here the output is itself a function which wants input, so in order not to get confused, we use a subscript.  

That is a start, but of course in group theory we want more than just a map of sets: we want to know it is a homomorphism!  One case of the homomorphism property in this example would read
\begin{equation} 
\sigma_{sr}\sigma_{r} \overset{?}{=} \sigma_{sr^2} 
\end{equation}
This is an equality we need to check on the right-hand side of \eqref{eqn:D3map}.  Composing the permutations, we see it checks out!  Once we have a homomorphism, we can also see that the kernel of the map $\sigma$ consists only of the identity: if an element maps to the identity permutation in $\Sym(D_6)$, it would come from a row of the Cayley table where they elements line up according to the identity, and that happens only for the top row.  So we get an injective map $D_6 \hookrightarrow S_6$.  By the fundamental homomorphism theorem, we see that $D_6$ is isomorphic to its image under this map; the following lemma reminds us of how this works in general.

\begin{thm}[Fundamental homomorphism theorem]
Let $\phi \colon G \to H$ be a group homomorphism.  Then $G/\ker \phi \simeq \phi(G) \leq H$.
\end{thm}

\begin{corollary}~\label{cor:injG}
If $\phi \colon G \to H$ is an injective group homomorphism, then $G \simeq \phi(G)$.
\end{corollary}

\begin{proof}
If $\phi$ is injective, then $\ker \phi =\{1\}$, and then $G/\ker \phi \simeq G$ (the cosets of the identity are just the elements of $G$!).
\end{proof}

We conclude that $D_6$ is isomorphic to a subgroup of $S_6$.  This is Cayley's theorem!  

\section{Main result}~\label{sec:thm}

We are now ready to prove Cayley's theorem, which we will prove in a slightly stronger form.

\begin{thm}[Cayley]  \label{thm:Cayleyn}
Let $G$ be a finite group of order $\#G = n$.  Then $G$ is isomorphic to a subgroup of $S_n$.
\end{thm}

We follow what we observed in the case $G=D_6$ in the previous section one step at a time.

Throughout, let $G$ be a finite group.  First, we define the permutations that come from the rows of the Cayley table.  Recall that in the row labelled $a$ (for $a \in G$), with $b \in G$ the column we have entry $ab$ (as usual suppressing $*$ and writing the group multiplicatively).

\begin{lem}
Let $a \in G$.  Define the map
\begin{align*}
\sigma_a \colon G &\to G \\
x &\mapsto ax
\end{align*}
Then $\sigma_a$ is a bijection, i.e., $\sigma_a \in \Sym(G)$.   
\end{lem}

\begin{proof}
We proved this in class, showing it is both injective and surjective and then that it has inverse $\sigma_{a^{-1}} \colon G \to G$ defined by $b \mapsto a^{-1}b$.
\end{proof}

We then built on this considering all of these bijections at once.

\begin{prop}~\label{prop:sigmaG}
Define the map
\begin{align*}
\sigma \colon G &\to \Sym(G) \\
a &\mapsto \sigma_a 
\end{align*}
Then $\sigma$ is an injective group homomorphism.  
\end{prop}

\begin{proof}
We first show that the map is a homomorphism.  Let $a,b \in G$.  We want to check
\[ \sigma_a \sigma_b \overset{?}{=} \sigma_{ab}. \]
These are two permutations of the set $G$.  To show that two functions are equal, we show that they give the same outputs.  So let $x \in G$.  Then on the left-hand side, by definition
\[ (\sigma_a \sigma_b)(x) = \sigma_a(\sigma_b(x)) = \sigma_a(bx) = a(bx) = abx. \]
This matches the right-hand side:
\[ \sigma_{ab}(x)=(ab)x = abx. \]
Thus $(\sigma_a\sigma_b)(x)=\sigma_{ab}(x)$ for all $x \in G$, so then $\sigma_a \sigma_b = \sigma_{ab} \in \Sym(G)$ as functions.  

To show that $\sigma$ is injective, we show that $\ker \sigma \subseteq \{1\}$.  Let $a \in G$ be such that $a$ maps to the identity: $\sigma_a=\id_G$.  Then for all $x \in G$ we have $\sigma_a(x)=\id_G(x)$, which means $ax=x$ for all $x \in G$.  If we plug in $x=1$ we get $a=1$, as desired.
\end{proof}

We may now conclude.

\begin{proof}[Proof of Theorem \ref{thm:Cayleyn}]
By Proposition \ref{prop:sigmaG}, we have an injective group homomorphism $G \hookrightarrow \Sym(G)$.  By Lemma \ref{lem:An}, we have an isomorphism $\Sym(G) \simeq S_n$, so composing these we get an injective group homomorphism $G \hookrightarrow S_n$.  Finally, by the fundamental homomorphism theorem (Corollary 
\ref{cor:injG}), we conclude that $G$ is isomorphic to its image, a subgroup of $S_n$.  
\end{proof}

\section{Conclusion}~\label{sec:conc}

Cayley's theorem shows that we can see every finite group as a subgroup of some permutation group.  There can be more than one way to realize a finite group $G$ as a subgroup of permutations: already for $D_6$ we showed that $D_6 \simeq S_3$ by considering the action of the dihedral group on the vertices of the triangle.  

Returning to our proof, we see that the arguments also work when $G$ is an infinite group: we still get an injective group homomorphism $G \hookrightarrow \Sym(G)$; however, now $\Sym(G)$ consists of permutations of an infinite set, so it is not of the form $S_n$!  

The homomorphism constructed in Proposition \ref{prop:sigmaG} arises naturally in the context of groups acting on themselves (by left multiplication): this is described in detail by Dummit--Foote \cite[\S 4.2]{DummitFoote}, with Cayley's theorem as a corollary \cite[\S 4.2, Corollary 4, p.~120]{DummitFoote}.  Indeed, \emph{group actions} allow us to see all homomorphisms from a finite group into permutation groups, whether that be on vertices of an $n$-gon, or on the cosets of a group.  

\begin{thebibliography}{99}

% \bibitem{DosReis2}
% Laura L.~Dos Reis and Anthony J.~Dos Reis, \emph{Abstract algebra: a student-friendly approach}, CreateSpace Independent Publishing Platform, 2017.

\bibitem{DummitFoote}
David S.~Dummit and Richard M.~Foote, \emph{Abstract algebra}, 3rd ed., John Wiley \& Sons, Hoboken, 2003.

% \bibitem{Pinter}
% Charles C.~Pinter, \emph{A book of abstract algebra}, 2nd ed., McGraw--Hill, New York, 1990.

\end{thebibliography}

\end{document}

\documentclass[11pt, reqno]{amsart}

% Include the macros file from `../common'
% ENCODING
\usepackage[utf8]{inputenc}
\usepackage{pmboxdraw}
% \usepackage{pmboxdraw-extras}

% text alignment
\usepackage{ragged2e}

% URLs
\usepackage{hyperref}

% Sets
\newcommand{\FF}{\mathbb{F}}
\newcommand{\NN}{\mathbb{N}}
\newcommand{\QQ}{\mathbb{Q}}
\newcommand{\RR}{\mathbb{R}}
\newcommand{\ZZ}{\mathbb{Z}}

\newcommand{\F}{\mathbb{F}}
\newcommand{\N}{\mathbb{N}}
\newcommand{\Q}{\mathbb{Q}}
\newcommand{\R}{\mathbb{R}}
\newcommand{\Z}{\mathbb{Z}}
\renewcommand{\b}{\{0,1\}}

% More Math operators
\newcommand{\GL}{\mathrm{GL}}
\newcommand{\calC}{\mathcal{C}}
\newcommand{\calP}{\mathcal{P}}
\newcommand{\calF}{\mathcal{F}}
\newcommand{\nequiv}{\not\equiv}
\renewcommand{\notin}{\not\in}

% Functions
\usepackage{amsmath}
\DeclareMathOperator{\cost}{cost}
\DeclareMathOperator{\len}{len}
\DeclareMathOperator{\rank}{rank}
\DeclareMathOperator{\sgn}{sgn}
\DeclareMathOperator{\wt}{wt}

% Combinatorics

% Probability
\newcommand{\Hd}{\texttt{\color{BrickRed}H}}
\newcommand{\Tl}{\texttt{T}}
\newcommand{\EE}{\mathop{\mathbb{E}}}
\newcommand{\PP}{\mathop{\mathbb{P}}}
\newcommand{\indic}{\mathbb{1}}
\DeclareMathOperator{\Var}{Var}

% End of proof marker
\newcommand{\qedblack}{\hfill\ensuremath{\blacksquare}}
\newcommand{\qedwhite}{\hfill\ensuremath{\square}}

% Caligraphic caps
\newcommand{\cA}{\mathcal{A}}
\newcommand{\cB}{\mathcal{B}}
\newcommand{\cC}{\mathcal{C}}
\newcommand{\cD}{\mathcal{D}}
\newcommand{\cE}{\mathcal{E}}
\newcommand{\cF}{\mathcal{F}}
\newcommand{\cI}{\mathcal{I}}
\newcommand{\cL}{\mathcal{L}}
\newcommand{\cP}{\mathcal{P}}
\newcommand{\cS}{\mathcal{S}}
\newcommand{\cT}{\mathcal{T}}
\newcommand{\cU}{\mathcal{U}}
\newcommand{\cX}{\mathcal{X}}
\newcommand{\cY}{\mathcal{Y}}
\newcommand{\cZ}{\mathcal{Z}}

% Boldface letters
\newcommand{\ba}{\mathbf{a}}
\newcommand{\bb}{\mathbf{b}}
\newcommand{\bp}{\mathbf{p}}
\newcommand{\bq}{\mathbf{q}}
\newcommand{\br}{\mathbf{r}}
\newcommand{\bu}{\mathbf{u}}
\newcommand{\bv}{\mathbf{v}}
\newcommand{\bx}{\mathbf{x}}
\newcommand{\by}{\mathbf{y}}
\newcommand{\bz}{\mathbf{z}}
\newcommand{\tbp}{\mathbf{\widetilde{p}}}

% Special math symbols
\newcommand{\eps}{\varepsilon}
\newcommand{\ceq}{\subseteq}
\newcommand{\ang}[1]{\langle{} #1 \rangle}
\newcommand{\ceil}[1]{\lceil{} #1 \rceil}
\newcommand{\floor}[1]{\lfloor{} #1 \rfloor}

% Problem names and other small-caps constants
\newcommand{\inv}{\textsc{inv}\xspace}

% Useful for marking steps of a derivation to explain later
\newcommand{\circled}[1]{\raisebox{.5pt}{\textcircled{\raisebox{-.1pt}{\scriptsize #1}}}}


% Page size and margins
% \usepackage[left=1in,right=1in,top=1.3in,bottom=1.3in,nofoot]{geometry}
\usepackage{fancyhdr}   % for fancy header
\usepackage{fancyvrb}   % for fancy verbatim
\usepackage{graphicx}   % for including images
\usepackage{enumerate}  % for enumerating lists
\usepackage{enumitem}
\usepackage[rgb, dvipsnames]{xcolor}
\usepackage{tcolorbox}
% \newtcolorbox{codebox}[1]{
%   box align=top,
%   colback=white!5!white,
%   colframe=white!75!black,
%   title=#1
% }
\usepackage{multicol}
% \usepackage{accode}

% My Problem set macros

% Answer BOX
% \usepackage{xcolor}
\usepackage{microtype}
\usepackage{mdframed}
\newmdenv[%
  leftmargin=-5pt,
  rightmargin=-5pt, 
  innerleftmargin=5pt,
  innerrightmargin=5pt,
  backgroundcolor=black!10
]{Answer}%


% Header BOX
\newcommand{\handout}[6]{
  \noindent
  \begin{center}
  \setlength{\fboxrule}{1.2pt}
  \framebox{
    \vbox{
      \hbox to 5.78in { \textbf{#6} \hfill {\bf #2} }
      \vspace{4mm}
      \hbox to 5.78in { {\Large \hfill {\textbf{ #5 }}  \hfill} }
      \vspace{2mm}
      \hbox to 5.78in { {\textit{\textbf{#3 \hfill #4}}} }
    }
  }
  \setlength{\fboxrule}{0.2pt}
  \end{center}
  \vspace*{4mm}
}

% Header BOX
\newcommand{\PSET}[5]{\handout{#1}{#2}{Prof.\ #3}{Student: #4}{PSET #1}{#5}}

% Credit Statement
\newcommand{\CreditStatement}[1]{
  \noindent
  \begin{center} {
    \bf Credit Statement
  }
  \end{center}
  { #1 }
}

% Problem counter
\newenvironment{problem}[1][]%
{%
\stepcounter{problem} \vspace{.2cm} \noindent {\bf \arabic{problem}.} {\textit{#1}}~%
}{%
\vspace{.2cm}%
}

% Package Imports
\usepackage{amssymb,amsthm,amsmath,amstext}
\usepackage{mathdots} % for \dots
  % \dotsc -- dots with commas.
  % \dotsb -- dots with binary operators.
  % \dotsm -- multiplication dots.
  % \dotsi -- dots with integrals.
  % \dotso -- "other dots".
\usepackage{wasysym, stackengine, makebox, tikz-cd}
\newcommand\isom{\mathrel{\stackon[-0.1ex]{\makebox*{\scalebox{1.08}{\AC}}{=\hfill\llap{=}}}{{\AC}}}}
\newcommand\nvisom{\rotatebox[origin=cc] {-90}{$ \isom $}}
\newcommand\visom{\rotatebox[origin=cc] {90} {$ \isom $}}


% Custom colors
\definecolor{crimson}{rgb}{0.86, 0.08, 0.24}
\definecolor{teal}{rgb}{0.0, 0.5, 0.5}
\definecolor{zaffre}{rgb}{0.0, 0.08, 0.66}
\newcommand{\crim}{\textcolor{crimson}}
\newcommand{\teal}{\textcolor{teal}}
\newcommand{\zaff}{\textcolor{zaffre}}
\newcommand{\black}{\textcolor{black}}
\definecolor{DarkOliveGreen}{rgb}{0.33, 0.42, 0.18}
\newcommand{\darkgreen}{\textcolor{DarkOliveGreen}}
\newcommand{\green}{\textcolor{OliveGreen}}

\newcommand{\id}{\mathbf{id}\;}

% matrices -- vertical separators
\makeatletter
\renewcommand*\env@matrix[1][*\c@MaxMatrixCols{ c}]{%
  \hskip -\arraycolsep{}
  \let\@ifnextchar\new@ifnextchar{}
  \array{#1}}
\makeatother

% \usepackage{accode}
\usepackage{tikz}

% long multiplications
\usepackage{xlop}

% custom functions.
\newcommand{\lcm}[2]{\mathbf{lcm}\;(#1,\;#2)}
\newcommand{\Therefore}{\dot{.\hspace{.095in}.}\hspace{.095in}}
\newcommand{\However}{\dot{}\hspace{.045in}.\hspace{.045in} \dot{}\hspace{.095in}}

% resume includes
\usepackage[utf8]{inputenc}
\usepackage[full]{textcomp}
\usepackage{CJKutf8}
\usepackage[lf]{ebgaramond}

\usepackage[OT1]{fontenc}
\usepackage{enumitem}
\usepackage[scale=.75]{geometry}
\usepackage{url}
% \usepackage[dvipsnames]{xcolor}
% package settings
% \usepackage[
%   hidelinks,
%   pdfnewwindow=true,
%   pdfauthor={Amittai},
%   pdftitle={resume},
% ]{hyperref}

\pagestyle{headings}
% \markright{siavava}

\setlength\parindent{2em}

\thispagestyle{empty}

\newcommand{\cvsubsection}[1]{\subsection*{\hspace{1.45em}#1}}


\pagestyle{fancy}                       % fancy (allow headers, footers)
\fancyhf{}                              % clear all header/footer settings.
\cfoot{\thepage}                        % set page-numbers in footer.
\lhead{\textit{\textbf{ Amittai, S}}}   % set name in header, left.
\rhead{\textsc{Math 71: Algebra}}       % set class name in header, right.

\newcounter{problem}
\setcounter{problem}{0}

\renewcommand{\theenumi}{\alph{enumi}}

\begin{document}
\setlength{\headheight}{13.0pt}
\setlength{\footskip}{13.0pt}

% \title{Math 81/111: Abstract Algebra \\ Homework \#1}
\title{Math 71: Algebra \\ Fundamental Long EXercises (FLEX)}
% \author{John Voight}
% \title{your name here}
\date{17 October 2022; due Monday, 14 November 2022}
\maketitle
FLEX (Fundamental Long EXercise) problems concern a theorem in group theory we did 
not prove in class.  You will be guided through the proof and asked to fill in the 
details carefully.  The intended audience is your peers, so you should write in a 
way that you would have wanted something explained to you.  You must choose one of 
the topics on the next page.  Your solution should contain:
\begin{itemize}
  \item An introduction and motivation, explaining context and why we should care, 
  including the statement of the main results, and any relevant acknowledgements 
  \item Examples, lemmas, propositions, preliminary setup, etc.
  \item Proof of the main results
  \item A conclusion, giving applications, corollaries, further examples, future 
  directions
\end{itemize}
The goal is that the reader (a fellow student) could read your paper and understand
very well the theorem and where it comes from---and they should also be convinced 
that you understood it!
There is no minimum or maximum length requirement.  Maybe 5 pages is a general 
ballpark.  More just for the sake of more in describing mathematics isn't helpful: 
you want to be both concise and complete.  Of course, more examples and 
different/reinforcing explanations are good, reviewing concepts and generally being
careful is all good!  But boring us with associativity axioms is not.  
You can write it out by hand (pencil or pen is OK), or you can type it out using \
LaTeX\ or a word processor.  A \LaTeX\ template is provided on Canvas, so you can 
copy-paste into Overleaf (\verb|http://overleaf.com|) and just modify it for your 
paper.  Even if you do not use \LaTeX, the template will hopefully give you some 
sense of what to be aiming for.  
You do not have to prove everything from scratch, but you must give precise 
references when you are using something important (e.g., ``By Lagrange's theorem 
[DF, \S 3.2, Theorem 8, p.~89]'' or ``we follow the proof of Dummit--Foote [DF, \S 
2.4, Proposition 9, p.~63]'').  You are certainly encouraged to use the textbook 
(indeed, some of the steps are worked out by Dummit--Foote, but they could often 
use elaboration or other TLC), but the whole point is not to take anything for 
granted: fill in the details, break things up into steps, explain it in a way that 
makes sense to you, etc.  
You are discouraged (but not forbidden) from using references other than the 
textbook.  If you refer to any materials (a website, another book, a random Youtube
video, a blog, you talk with another student) other than Dummit--Foote at
\emph{any} time while working on this paper, you must explicitly say so: both in the
acknowledgements ``We consulted [Wikipedia, \ldots]'' \emph{and} any specific use in 
the proof as in the previous paragraph.  Yes: each and every thing you consult, 
cite it.  
You are very much encouraged to talk to the instructor!  If you're not sure about a
lemma, or you are stuck on a step, or you're not sure what else to say in terms of 
framing or application, etc.  
\section*{Grading}
\begin{itemize}
\item 10 points: Mathematical correctness/completeness
\item 10 points: Organization and clarity/conciseness
\item 5 points: Copyediting
\end{itemize}
The flex problem is due in class on Monday, 14 November 2022.  You must hand in a 
physical copy (e.g.\ a print out) and do so in person.  No late assignments will be
accepted.  
\vfill\newpage
\section*{Possible topic 1: Cauchy's theorem}
Difficulty level: medium.
\begin{enumerate}
\item Using Lagrange's theorem, motivate and state Cauchy's theorem (DF, section 
3.2, Theorem 11, p.~93).  
\item Read
\begin{center}
\verb|https://www.sciencedirect.com/science/article/pii/S031508600300003X|
\end{center}
and provide a few historical remarks (whatever you find interesting and relevant). 
\item Give three proofs of Cauchy's theorem, as follows.
\begin{enumalph}
\item First, fill in details of the elementary proof in DF, section 3.2, Exercise 
9, p.~96.  Explain what is happening in this proof by group actions (e.g.\ the 
orbit-stabilizer theorem).
\item Second, prove Cauchy's theorem for $G$ an abelian group (see DF, section 3.5,
Proposition 21, p.~102), then finish with the class equation.  \emph{[Hint: if $p \
mid Z(G)$, the case of abelian groups applies; otherwise, show that $p \mid C_G(a)$
for some noncentral $a \in G$.]}
\item Third, observe that Cauchy's theorem is a corollary of Sylow's theorem (DF, 
section 4.5, Exercise 3, p.~146).
\end{enumalph}
\item Show that Cauchy's theorem is best possible: give examples of groups $G$ and 
integers $d \mid \#G$ such that there is no element of order $d$ in $G$, in 
particular give an example of a group $G$ and a prime $p$ with $p^2 \mid \#G$ but 
such that $G$ has no element of order $p^2$.  
\item As a first application, show that for a finite abelian group $G$ with 
  $\#G=n$, the map $G \to G$ by $x \mapsto x^k$ is an isomorphism if and only if
  $\gcd(k,n)=1$.  
\item As a second application, show that a finite abelian group $G$ with $\#G = n$ 
has a subgroup of order $d$ for every positive divisor $d \mid n$ (DF, section 3.4,
Exercise 4, p.~106).
\end{enumerate}
\section*{Possible topic 2: Finite abelian groups}
Difficulty level: additional technicality (+2 bonus).
\begin{enumerate}
\item Motivate the fundamental theorem of finite abelian groups (DF, section 5.2, 
Parts 1 and 2 of Theorem 5) by analogy with a basis for a vector space, and go 
beyond the analogy to a direct statement by considering a finite-dimensional vector
space over the field $F=\Z/p\Z$.    
\item Prove the fundamental theorem of finite abelian groups, using the sketch in 
DF, section 6.1, pp.~196--197, with the following advice:
\begin{enumalph}
\item Give a complete proof of the recognition theorem for direct products (DF, 
section 5.4, Theorem 9, p.~171), including a proof of DF, section 5.4, Proposition 
8, p.~171.  
\item Do not take DF, section 6.1, Corollary 4, for granted.  Argue directly, as 
follows.  Let $G$ be a finite abelian group.
\begin{itemize}
\item For any $m \in \Z_{\geq 1}$, let $G_m \subseteq G$ be the subset of elements 
whose order divides $m$.  Show that $G_m \leq G$ is a subgroup.  
\item Suppose that $\#G = p^e m$ where $p \nmid m$ is prime.  Show that $G \simeq 
G_{p^e} \times G_m$.  \emph{[Hint: to show $G_{p^e} G_m=G$, note that for any $x \
in G$ we have $x^{p^e} \in G_m$ and $x^m \in G_{p^e}$, so then apply the extended 
Euclidean algorithm.]}  
\item Finally, if we have a factorization $\#G = p_1^{e_1} p_2^{e_2} \cdots 
p_r^{e_r}$ into primes, show (by induction) that $G \simeq G_{p_1^{e_1}} \times 
G_{p_2^{e_2}} \times \\cdots \times G_{p_r^{e_r}}$.
\end{itemize}
\end{enumalph}
\item Illustrate the key steps in this somewhat involved proof by showing what 
happens in examples.  (You can do this before, during, or after the proof.)
\item As an application, prove that a finite subgroup $H \leq F^\times$ of the 
multiplicative group of a field $F$ is cyclic (see DF, section 9.5, Proposition 18,
p.~314).  We will almost surely prove DF, section 9.5, Proposition 17, p.~313, in 
class; but you should just prove it directly (it is not necessary to know that 
$F[x]$ is a UFD).
\end{enumerate}
\section*{Possible topic 3: Groups of small order}
Difficulty level: more concepts, but some optionality (starts with +2 bonus, plus 
whatever extra below)
\begin{enumerate}
\item Motivate the classification of groups of small order up to isomorphism; 
quickly review the case of groups of prime order, providing context to why the 
factorization of the order $n=\#G$ is reflected in the list of possible groups of 
order $n$ up to isomorphism.  
\item State Cauchy's theorem and the fundamental theorem of finite abelian groups 
(giving references, but without proofs; if these are of interest, consider one of 
the other projects!).  
\item Recall the proof why every group of order $p^2$ with $p$ prime is abelian.  
\item Classify the \emph{abelian} groups of order $n \leq 15$ up to isomorphism 
using the fundamental theorem.
\item Classify groups of order $6$ by hand: using Cauchy's theorem, there exists $a
\in G$ of order $2$ and $b \in G$ of order $3$;
  show that $G=\{1,b,b^2,a,ab,ab^2\}$, in a direct manner that
  $ba=ab$ or $ba=ab^2$, and show that these two possibilities uniquely determine
  the Cayley table of $G$.
\item Show that every nonabelian group $G$ of order $8$ is isomorphic to either 
$D_8$ or $Q_8$, as follows.
\begin{itemize}
\item Show that $G$ has an element $a \in G$ of order $4$.  \emph{[Hint: what 
happens if every nonidentity element has order $2$?]}
\item Let $H \colonequals \langle a \rangle$ and let $b \not \in H$.  Observe that 
$H \trianglelefteq G$ is normal; argue that $bab^{-1}=a^3$ (the order under 
conjugation is preserved), and then that $G \simeq D_4,Q_8$ according as $b$ has 
order $2$ or $4$.  
\end{itemize}
\item Pause to show we have classified groups of order $n \leq 9$.  
\item Now let $G$ be a group of order $pq$ where $p,q$ are primes with $p < q$ 
(without loss of generality).  
\begin{itemize}
\item Let $P \leq G$ be a $p$-Sylow subgroup and $Q \leq G$ be a $q$-Sylow 
subgroup.  Show that $Q \trianglelefteq G$ is normal.  \emph{[Hint: it has index 
$p$, so go through DF, section 4.2, Corollary 5, pp.~120--121.]}  Write $P=\langle 
x \rangle$ and $Q=\langle y \rangle$ with $x,y \in G$.
\item Show that $xyx^{-1}=y^k$ with $k \in \{1,\\cdots,q-1\}$, and use this to define
a group homomorphism
\[ \phi \colon P \to {(\Z/q\Z)}^\times \]
where $x \mapsto k$.  \emph{[Hint: use $x^i y x^{-i} = y^{\phi(i)}$.]}  Conclude 
that either $\phi$ is the trivial homomorphism (mapping every element to $1$) or $\
phi$ is injective.
\item If $\phi$ is trivial, prove that $G$ is cyclic (DF, section 4.4, Exercise 2, 
p.~137).
\item Show that $\phi$ is injective if and only if $G$ is nonabelian and $p \mid 
(q-1)$.  (In particular, observe that if $p \nmid (q-1)$ then $G$ is abelian.)
\item If $p \mid (q-1)$, exhibit a nonabelian group of order $pq$ (following DF, 
Example, section 4.5, p.~143; see also DF, section 4.3, Exercise 34, p.~132).  Show
that when $p=2$ we obtain the dihedral group $D_{2q}$ of order $2q$ for $q \geq 3$ 
as a subgroup of $S_q$ via the action on the vertices of a $q$-gon.
\item Suppose that $G$ is nonabelian and $p \mid (q-1)$.  Show that $P \not\
trianglelefteq G$ (DF, Example, section 4.5, p.~143), so there exists an injective 
homomorphism $G \hookrightarrow S_q$ whose image up to conjugation lies in the the 
normalizer of the cyclic subgroup generated by the $q$-cycle $(1\ 2\ \\cdots \ q)$ 
(DF, section 4.3, Exercise 28, p.~132).  When $p=2$, show that this group is unique
up to conjugation.  \emph{[Hint: show that there is a unique subgroup of ${(\Z/q\
Z)}^\times$ of order $2$.]}
\end{itemize}
\item[{(8${}^\prime$)}] For +3 bonus going a more conceptual route, replace step 
(8) as follows.
\begin{itemize}
\item Read DF, section 4.4.  
\item Prove that the automorphism group of $Z_p$ is $\Aut(Z_p) \simeq
  {(\Z/p\Z)}^\times$ (DF, section 4.4, Proposition 16, p.~135 proves something more general).  
State (but you do not need to prove) that ${(\Z/p\Z)}^\times \simeq Z_{p-1}$ is 
cyclic.  
\item Suppose that $p \nmid (q-1)$.  Show that $G$ is abelian (DF, Example, section
4.4, p.~135--136)
and therefore cyclic (DF, section 4.4, Exercise 2, p.~137).  
\item Now suppose that $p \mid (q-1)$.  Let $P \leq G$ be a $p$-Sylow subgroup and 
$Q \leq G$ be a $q$-Sylow subgroup.  Show that $Q \trianglelefteq G$ is normal in 
$G$ (DF, Example, section 4.5, p.~143).
\item Read section 5.5.  Show that if $p \mid (q-1)$ then either $G \simeq Z_p \
times Z_q \simeq Z_{pq}$ or $G \simeq Z_p \rtimes Z_q$, the semi-direct product 
with respect to the homomorphism $Z_p \to \Aut(Z_q) \simeq {(\Z/q\Z)}^\times = \
langle g \rangle$ mapping $x \mapsto g^{(q-1)/p}$ (DF, Example, section 5.5, 
pp.~181--182; section 5.5, Exercise 6, pp.~184--185).
\end{itemize}
\item Classify groups of order $n \leq 15$ except $n=12$.
\item For +2 bonus, classify the groups of order $12$.  Let $P_2$ be a $2$-Sylow 
subgroup (of order $4$) and $P_3$ a $3$-Sylow subgroup.  
\begin{itemize}
\item Show that either $P_2 \trianglelefteq G$ is normal or $P_3 \trianglelefteq G$
is normal. 
\emph{[Hint: if $P_3$ is not normal, count the number of elements of order $3$.]}  
Conclude that in either case, $G=P_2P_3$.
\item Show that if both $P_2$ and $P_3$ are normal, then $G \simeq P_2 \times P_3$ 
is abelian.  \emph{[Hint: either state and use the recognition theorem for direct 
products, DF, section 5.4, Theorem 9, p.~171--172; or understand the proof and 
apply the argument here directly.]}
\item Proceeding by cases, first suppose that $P_2 \trianglelefteq G$ and $P_2 \
simeq Z_4$.  Arguing as in (8), show that the homomorphism $P_3 \to \Aut(P_2)$ is 
trivial, so $G$ is abelian and therefore cyclic.  \emph{[Hint: 
$\#\Aut(Z_4)=\#{(\Z/4\Z)}^\times=2$.]}
\item Suppose $P_2 \trianglelefteq G$ and $P_2 \simeq Z_2 \times Z_2$ but $P_3 \
not\trianglelefteq G$.  Show that there is an injective group homomorphism $G \
hookrightarrow S_4$, and conclude that $G \simeq A_4$.
\item Suppose $P_3 \trianglelefteq G$ and $P_2 \simeq Z_4$ but $P_2 \not\
trianglelefteq G$.  Write $P_2=\langle x \rangle$ and $P_3=\langle y \rangle$ and 
show that $xyx^{-1}=y^{-1}$, then show that this uniquely determines the Cayley 
table of $G$, with presentation
\[ G \simeq \langle x,y \,|\, x^4=y^3=1, xyx^{-1}=y^{-1} \rangle. \]
\item Finally, suppose $P_3 \trianglelefteq G$ and $P_2 \simeq Z_2 \times Z_2$ but 
$P_2 \not\trianglelefteq G$.  Writing $P_3 = \langle y \rangle$, show that $P_2 = \
langle x_1 \rangle \times \langle x_2 \rangle$ for some $x_1,x_2 \in P_2$ with 
$x_1yx_1^{-1}=y$ and $x_2yx_2^{-1}=y^{-1}$.  Conclude that $x_1y$ has order $6$ and
then that $G \simeq D_{12}$, the dihedral group of order $12$.   
\end{itemize}
\item Give a table of groups of order $n\leq 15$ up to isomorphism (DF, section 
5.3, p.~168), giving the answer even if you don't do (10).
\item As an application, identify in the table the three groups $G_1=Z_2 \times 
D_6$, 
\[ G_2 \colonequals
  \left\{ \begin{pmatrix}
    1 & a & b \\
    0 & 1 & c \\
    0 & 0 & 1
  \end{pmatrix} : a,b,c \in \Z/2\Z \right\} \leq \GL_3(\Z/2\Z) \]
and $G_3=S/Z(S)$ where  
\[ S=\SL_2(\Z/3\Z)=\left\{ \begin{pmatrix} a & b \\ c & d \end{pmatrix} : a,b,c,d
\in \Z/3\Z \text{ and } ad-bc=1 \right\} \leq \GL_2(\Z/3\Z). \]
\end{enumerate}
\end{document}

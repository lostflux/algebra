\section*{Possible topic 3: Groups of small order}
Difficulty level: more concepts, but some optionality (starts with +2 bonus, plus 
whatever extra below) Is this working?

\begin{enumerate}[label=\arabic{enumi}.]
\item Motivate the classification of groups of small order up to isomorphism; 
quickly review the case of groups of prime order, providing context to why the 
factorization of the order $n=\#G$ is reflected in the list of possible groups of 
order $n$ up to isomorphism.  
\item State Cauchy's theorem and the fundamental theorem of finite abelian groups 
(giving references, but without proofs; if these are of interest, consider one of 
the other projects!).  
\item Recall the proof why every group of order $p^2$ with $p$ prime is abelian.  
\item Classify the \emph{abelian} groups of order $n \leq 15$ up to isomorphism 
using the fundamental theorem.
\item Classify groups of order $6$ by hand: using Cauchy's theorem, there exists $a
\in G$ of order $2$ and $b \in G$ of order $3$;
  show that $G=\{1,b,b^2,a,ab,ab^2\}$, in a direct manner that
  $ba=ab$ or $ba=ab^2$, and show that these two possibilities uniquely determine
  the Cayley table of $G$.
\item Show that every nonabelian group $G$ of order $8$ is isomorphic to either 
$D_8$ or $Q_8$, as follows.
\begin{itemize}
\item Show that $G$ has an element $a \in G$ of order $4$.  \emph{[Hint: what 
happens if every nonidentity element has order $2$?]}
\item Let $H \colonequals \langle a \rangle$ and let $b \not \in H$.  Observe that 
$H \trianglelefteq G$ is normal; argue that $bab^{-1}=a^3$ (the order under 
conjugation is preserved), and then that $G \simeq D_4,Q_8$ according as $b$ has 
order $2$ or $4$.  
\end{itemize}
\item Pause to show we have classified groups of order $n \leq 9$.  
\item Now let $G$ be a group of order $pq$ where $p,q$ are primes with $p < q$ 
(without loss of generality).  
\begin{itemize}
\item Let $P \leq G$ be a $p$-Sylow subgroup and $Q \leq G$ be a $q$-Sylow 
subgroup.  Show that $Q \trianglelefteq G$ is normal.  \emph{[Hint: it has index 
$p$, so go through DF, section 4.2, Corollary 5, pp.~120--121.]}  Write $P=\langle 
x \rangle$ and $Q=\langle y \rangle$ with $x,y \in G$.
\item Show that $xyx^{-1}=y^k$ with $k \in \{1,\\cdots,q-1\}$, and use this to define
a group homomorphism
\[ \phi \colon P \to {(\Z/q\Z)}^\times \]
where $x \mapsto k$.  \emph{[Hint: use $x^i y x^{-i} = y^{\phi(i)}$.]}  Conclude 
that either $\phi$ is the trivial homomorphism (mapping every element to $1$) or $\
phi$ is injective.
\item If $\phi$ is trivial, prove that $G$ is cyclic (DF, section 4.4, Exercise 2, 
p.~137).
\item Show that $\phi$ is injective if and only if $G$ is nonabelian and $p \mid 
(q-1)$.  (In particular, observe that if $p \nmid (q-1)$ then $G$ is abelian.)
\item If $p \mid (q-1)$, exhibit a nonabelian group of order $pq$ (following DF, 
Example, section 4.5, p.~143; see also DF, section 4.3, Exercise 34, p.~132).  Show
that when $p=2$ we obtain the dihedral group $D_{2q}$ of order $2q$ for $q \geq 3$ 
as a subgroup of $S_q$ via the action on the vertices of a $q$-gon.
\item Suppose that $G$ is nonabelian and $p \mid (q-1)$.  Show that $P \not\
trianglelefteq G$ (DF, Example, section 4.5, p.~143), so there exists an injective 
homomorphism $G \hookrightarrow S_q$ whose image up to conjugation lies in the the 
normalizer of the cyclic subgroup generated by the $q$-cycle $(1\ 2\ \\cdots \ q)$ 
(DF, section 4.3, Exercise 28, p.~132).  When $p=2$, show that this group is unique
up to conjugation.  \emph{[Hint: show that there is a unique subgroup of ${(\Z/q\
Z)}^\times$ of order $2$.]}
\end{itemize}
\item[{8${}^\prime$.}] For +3 bonus going a more conceptual route, replace step 
(8) as follows.
\begin{itemize}
\item Read DF, section 4.4.  
\item Prove that the automorphism group of $Z_p$ is $\Aut(Z_p) \simeq
  {(\Z/p\Z)}^\times$ (DF, section 4.4, Proposition 16, p.~135 proves something more general).  
State (but you do not need to prove) that ${(\Z/p\Z)}^\times \simeq Z_{p-1}$ is 
cyclic.  
\item Suppose that $p \nmid (q-1)$.  Show that $G$ is abelian (DF, Example, section
4.4, p.~135--136)
and therefore cyclic (DF, section 4.4, Exercise 2, p.~137).  
\item Now suppose that $p \mid (q-1)$.  Let $P \leq G$ be a $p$-Sylow subgroup and 
$Q \leq G$ be a $q$-Sylow subgroup.  Show that $Q \trianglelefteq G$ is normal in 
$G$ (DF, Example, section 4.5, p.~143).
\item Read section 5.5.  Show that if $p \mid (q-1)$ then either $G \simeq Z_p \
times Z_q \simeq Z_{pq}$ or $G \simeq Z_p \rtimes Z_q$, the semi-direct product 
with respect to the homomorphism $Z_p \to \Aut(Z_q) \simeq {(\Z/q\Z)}^\times = \
langle g \rangle$ mapping $x \mapsto g^{(q-1)/p}$ (DF, Example, section 5.5, 
pp.~181--182; section 5.5, Exercise 6, pp.~184--185).
\end{itemize}
\item Classify groups of order $n \leq 15$ except $n=12$.
\item For +2 bonus, classify the groups of order $12$.  Let $P_2$ be a $2$-Sylow 
subgroup (of order $4$) and $P_3$ a $3$-Sylow subgroup.  
\begin{itemize}
\item Show that either $P_2 \trianglelefteq G$ is normal or $P_3 \trianglelefteq G$
is normal. 
\emph{[Hint: if $P_3$ is not normal, count the number of elements of order $3$.]}  
Conclude that in either case, $G=P_2P_3$.
\item Show that if both $P_2$ and $P_3$ are normal, then $G \simeq P_2 \times P_3$ 
is abelian.  \emph{[Hint: either state and use the recognition theorem for direct 
products, DF, section 5.4, Theorem 9, p.~171--172; or understand the proof and 
apply the argument here directly.]}
\item Proceeding by cases, first suppose that $P_2 \trianglelefteq G$ and $P_2 \
simeq Z_4$.  Arguing as in (8), show that the homomorphism $P_3 \to \Aut(P_2)$ is 
trivial, so $G$ is abelian and therefore cyclic.  \emph{[Hint: 
$\#\Aut(Z_4)=\#{(\Z/4\Z)}^\times=2$.]}
\item Suppose $P_2 \trianglelefteq G$ and $P_2 \simeq Z_2 \times Z_2$ but $P_3 \
not\trianglelefteq G$.  Show that there is an injective group homomorphism $G \
hookrightarrow S_4$, and conclude that $G \simeq A_4$.
\item Suppose $P_3 \trianglelefteq G$ and $P_2 \simeq Z_4$ but $P_2 \not\
trianglelefteq G$.  Write $P_2=\langle x \rangle$ and $P_3=\langle y \rangle$ and 
show that $xyx^{-1}=y^{-1}$, then show that this uniquely determines the Cayley 
table of $G$, with presentation
\[ G \simeq \langle x,y \,|\, x^4=y^3=1, xyx^{-1}=y^{-1} \rangle. \]
\item Finally, suppose $P_3 \trianglelefteq G$ and $P_2 \simeq Z_2 \times Z_2$ but 
$P_2 \not\trianglelefteq G$.  Writing $P_3 = \langle y \rangle$, show that $P_2 = \
langle x_1 \rangle \times \langle x_2 \rangle$ for some $x_1,x_2 \in P_2$ with 
$x_1yx_1^{-1}=y$ and $x_2yx_2^{-1}=y^{-1}$.  Conclude that $x_1y$ has order $6$ and
then that $G \simeq D_{12}$, the dihedral group of order $12$.   
\end{itemize}
\item Give a table of groups of order $n\leq 15$ up to isomorphism (DF, section 
5.3, p.~168), giving the answer even if you don't do (10).
\item As an application, identify in the table the three groups $G_1=Z_2 \times 
D_6$, 
\[ G_2 \colonequals
  \left\{ \begin{pmatrix}
    1 & a & b \\
    0 & 1 & c \\
    0 & 0 & 1
  \end{pmatrix} : a,b,c \in \Z/2\Z \right\} \leq \GL_3(\Z/2\Z) \]
and $G_3=S/Z(S)$ where  
\[ S=\SL_2(\Z/3\Z)=\left\{ \begin{pmatrix} a & b \\ c & d \end{pmatrix} : a,b,c,d
\in \Z/3\Z \text{ and } ad-bc=1 \right\} \leq \GL_2(\Z/3\Z). \]
\end{enumerate}

FLEX (Fundamental Long EXercise) problems concern a theorem in group theory we did 
not prove in class.  You will be guided through the proof and asked to fill in the 
details carefully.  The intended audience is your peers, so you should write in a 
way that you would have wanted something explained to you.  You must choose one of 
the topics on the next page.  Your solution should contain:
\begin{itemize}
  \item An introduction and motivation, explaining context and why we should care, 
  including the statement of the main results, and any relevant acknowledgements 
  \item Examples, lemmas, propositions, preliminary setup, etc.
  \item Proof of the main results
  \item A conclusion, giving applications, corollaries, further examples, future 
  directions
\end{itemize}
The goal is that the reader (a fellow student) could read your paper and understand
very well the theorem and where it comes from---and they should also be convinced 
that you understood it!
There is no minimum or maximum length requirement.  Maybe 5 pages is a general 
ballpark.  More just for the sake of more in describing mathematics isn't helpful: 
you want to be both concise and complete.  Of course, more examples and 
different/reinforcing explanations are good, reviewing concepts and generally being
careful is all good!  But boring us with associativity axioms is not.  
You can write it out by hand (pencil or pen is OK), or you can type it out using \
LaTeX\ or a word processor.  A \LaTeX\ template is provided on Canvas, so you can 
copy-paste into Overleaf (\verb|http://overleaf.com|) and just modify it for your 
paper.  Even if you do not use \LaTeX, the template will hopefully give you some 
sense of what to be aiming for.  
You do not have to prove everything from scratch, but you must give precise 
references when you are using something important (e.g., ``By Lagrange's theorem 
[DF, \S 3.2, Theorem 8, p.~89]'' or ``we follow the proof of Dummit--Foote [DF, \S 
2.4, Proposition 9, p.~63]'').  You are certainly encouraged to use the textbook 
(indeed, some of the steps are worked out by Dummit--Foote, but they could often 
use elaboration or other TLC), but the whole point is not to take anything for 
granted: fill in the details, break things up into steps, explain it in a way that 
makes sense to you, etc.  
You are discouraged (but not forbidden) from using references other than the 
textbook.  If you refer to any materials (a website, another book, a random Youtube
video, a blog, you talk with another student) other than Dummit--Foote at
\emph{any} time while working on this paper, you must explicitly say so: both in the
acknowledgements ``We consulted [Wikipedia, \ldots]'' \emph{and} any specific use in 
the proof as in the previous paragraph.  Yes: each and every thing you consult, 
cite it.  
You are very much encouraged to talk to the instructor!  If you're not sure about a
lemma, or you are stuck on a step, or you're not sure what else to say in terms of 
framing or application, etc.  
\section*{Grading}
\begin{itemize}
\item 10 points: Mathematical correctness/completeness
\item 10 points: Organization and clarity/conciseness
\item 5 points: Copyediting
\end{itemize}
The flex problem is due in class on Monday, 14 November 2022.  You must hand in a 
physical copy (e.g.\ a print out) and do so in person.  No late assignments will be
accepted.  

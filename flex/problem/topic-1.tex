\section*{Possible topic 1: Cauchy's theorem}
Difficulty level: medium.
\begin{enumerate}[label=\arabic{enumi}.]
  \item Using Lagrange's theorem, motivate and state Cauchy's theorem (DF, section 
  3.2, Theorem 11, p.~93).  
  \item Read
  \begin{center}
  \verb|https://www.sciencedirect.com/science/article/pii/S031508600300003X|
  \end{center}
  and provide a few historical remarks (whatever you find interesting and relevant). 
  \item Give three proofs of Cauchy's theorem, as follows.
  \begin{enumalph}
  \item First, fill in details of the elementary proof in DF, section 3.2, Exercise 
  9, p.~96.  Explain what is happening in this proof by group actions (e.g.\ the 
  orbit-stabilizer theorem).
  \item Second, prove Cauchy's theorem for $G$ an abelian group (see DF, section 3.5,
  Proposition 21, p.~102), then finish with the class equation.  \emph{[Hint: if $p \
  mid Z(G)$, the case of abelian groups applies; otherwise, show that $p \mid C_G(a)$
  for some noncentral $a \in G$.]}
  \item Third, observe that Cauchy's theorem is a corollary of Sylow's theorem (DF, 
  section 4.5, Exercise 3, p.~146).
  \end{enumalph}
  \item Show that Cauchy's theorem is best possible: give examples of groups $G$ and 
  integers $d \mid \#G$ such that there is no element of order $d$ in $G$, in 
  particular give an example of a group $G$ and a prime $p$ with $p^2 \mid \#G$ but 
  such that $G$ has no element of order $p^2$.  
  \item As a first application, show that for a finite abelian group $G$ with 
    $\#G=n$, the map $G \to G$ by $x \mapsto x^k$ is an isomorphism if and only if
    $\gcd(k,n)=1$.  
  \item As a second application, show that a finite abelian group $G$ with $\#G = n$ 
  has a subgroup of order $d$ for every positive divisor $d \mid n$ (DF, section 3.4,
  Exercise 4, p.~106).
\end{enumerate}

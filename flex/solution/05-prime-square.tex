\section{Groups of Order Square Prime}~\label{sec:prime-square}

\begin{theorem}[Cauchy's Theorem]\label{thm:cauchy}
  If $G$ is a finite group and $p$ is a prime dividing $\abs{G}$,
  then $G$ has an element of order $p$.~\cite[p.~93,~Theorem~3.1]{DummitFoote} 
\end{theorem}

\begin{proposition}~\label{prop:trivial-isom-direct-product}
  A group $G$ generated by two elements $x$ and $y$
  having $\langle x \rangle \cap \langle y \rangle = 1$
  is isomorphic to the direct product $\langle x \rangle \times \langle y \rangle$.

  \begin{proof}
    Let $G$ be a group generated by $x$ and $y$
    where $\langle x \rangle \cap \langle y \rangle = 1$.
    Define the map $\psi: \langle x \rangle \times \langle y \rangle \to G$ by:
    \[ \psi(x^m, y^n) = x^my^n \]
    for all $m, n \in \Z$.
    Then $\psi$ is an isomorphism,.
    Therefore, $G$ is isomorphic to $\langle x \rangle \times \langle y \rangle$.
  \end{proof}
\end{proposition}

\begin{proposition}~\label{prop:abelian-dir-prod-abelian}
  The direct product of two abelian groups is abelian.

  \begin{proof}
    Let $G$ and $H$ be abelian groups.
    $a, b \in G$ and $c, d \in H$.
    Let $(a, c)$ and $(b, d)$ be elements of $G \times H$.
    Then:
    \begin{align*}
      (a, c) \cdot (b, d) &= (a \cdot b, c \cdot d) \\
                          &= (b \cdot a, d \cdot c) \quad \zaff{\text{(Since $G$ and $H$ are abelian)}} \\
                          &= (b, d) \cdot (a, c)
    \end{align*}
    Therefore, $G \times H$ is abelian.
  \end{proof}
\end{proposition}

\begin{corollary}~\label{thm:p2-abelian}
  If $G$ is a finite group of order $p^2$ with $p$ prime, then $G$ is abelian.
  and isomorphic to either $C_{p^2}$ or $C_p \times C_p$.
  \begin{proof}
    Let $G$ be a finite group of order $p^2$ with $p$ prime.
    Consider the center $Z(G) \le G$.
    By Lagrange's theorem~\ref{thm:lagrange}, we know that $\#Z(G) \mid \#G$,
    therefore $\#Z(G) \in \{1,p,p^2\}$. Considering these cases;
    \begin{enumalph}
      \item By the class equation~\cite[p.~125,~Theorem~8]{DummitFoote},
        we know the center of a group of prime power \emph{must be nontrivial}.
        Therefore, $\#Z(G) \ne 1$.
      \item If $G$ has an element of order $p^2$, then $G$ is cyclic,
        therefore abelian and isomorphic to $C_{p^2}$
        as demonstrated in Section~\ref{sec:cyclic-non-prime}.
      \item Assuming $G$ does not have an element of order $p^2$,
        then every non-identity element must have order $p$ since the order must divide $p^2$
        (see~\ref{thm:lagrange}) and the order is greater than $1$ (since the element is not identity).
        Cauchy's theorem (see~\ref{thm:cauchy}) also tells us that $G$ must have an element of order $p$.
        Let $x$ be one such element, generating the subgroup $\langle x \rangle$ of order $p$.
        Let $y \in G \setminus \langle x \rangle$, then $\langle y \rangle$
        is also a subgroup of order $p$, and the groups $\langle x \rangle$ and $\langle y \rangle$
        intersect trivially. Both $\langle x \rangle$ and $\langle y \rangle$ are cyclic,
        so $\langle x \rangle \isom \langle y \rangle \isom C_p$
        by Corollary~\ref{cor:prime-then-abelian} and Definition~\ref{def:C_n}.
        Now, consider the group $\langle x, y \rangle \simeq
        \langle x \rangle \times \langle y \rangle$ (by~\ref{prop:trivial-isom-direct-product}).
        Since $x \in G$ and $y \in G$, $\langle x, y \rangle \subseteq G$.
        Additionally, $\langle x, y \rangle$ has order $p^2$
        since $x$ has order $p$ and $y$ has order $p$,
        so $\langle x, y \rangle = G$.
        However, $\langle x \rangle$ and $\langle y \rangle$ are abelian,
        so Proposition~\ref{prop:abelian-dir-prod-abelian},
        tells us that $G$, through its isomorphism to a direct product of two abelian  groups,
        must also be abelian.
    \end{enumalph}
  \end{proof}
\end{corollary}

Using Corollary~\ref{thm:p2-abelian}, we can classify all groups of order
$4$ and order $9$:

\begin{center}
  \begin{table}[H]~\label{tab:groups-order-p_2}
    \begin{tabular}{ l r r }
      Order & Group & Isomorphisms \\
      \midrule
      $4$ & $C_4$ & $\Z/4\Z$ \\
          & $C_2 \times C_2 \simeq V_4$ & $\Z/2\Z \times \Z/2\Z$ \\
      \midrule
      $9$ & $C_9$ & $\Z/9\Z$ \\
          & $C_3 \times C_3$ & $\Z/3\Z \times \Z/3\Z$ \\
      \bottomrule
    \end{tabular}
    \caption{Groups of order $p^2 \le 15$ for $p$ prime.}
  \end{table}
\end{center}

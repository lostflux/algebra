
\section{Groups of Order 12}

Let $G$ be a group of order $12$.
Since $12 = 2^2 \cdot 3$, Sylow's theorem (~\ref{thm:sylow})
tells us that $G$ has \emph{at least one} Sylow $2$-subgroup
and \emph{at least one} Sylow $3$-subgroup.
Let $P_2$ be a Sylow $2$-subgroup of order $4$, and $P_3$ a Sylow $3$-subgroup.
By Sylow's theorem, $n_3$ is either $1$ or $4$, and $P_3$ has order $3$.

\begin{proposition}
  Either $P_2$ is normal in $G$ or $P_3$ is normal.

  \begin{proof}
    Suppose $G$ is a group of order $12$ with $P_2$ and $P_3$
    as Sylow subgroups as defined above.
    If $n_3 = 1$, then $P_3$ is normal (since Sylow $2$-subgroups
    have order $4$, thus cannot contain an element of order $3$).
    Suppose $P_3$ is not normal, then there must be more than
    $1$ possible subgroup of order $3$.
    The only other possible value for $n_3$ is $4$.
    Since Sylow subgroups intersect trivially,
    this implies that there are $2 \times 4 = 8$ elements of order $3$ in $G$.
    That leaves $4$ elements not of order $3$.
    A Sylow $2$-subgroup has order $4$, so a single subgroup
    must account for all the remaining elements. Therefore,
    if $n_3 \neq 1$ then $n_2 = 1$ and $P_2$ is normal.
  \end{proof}

  In either case, note that $P_2$ and $P_3$ intersect trivially,
  so the product $P_2P_3$, is not only contained in $G$
  but also has order $4 \times 3 = 12$.
  Therefore, $P_2P_3 = G$.
\end{proposition}

\subsection*{Case 1} Suppose both $P_2$ and $P_3$ are normal,
then $G \simeq P_2 \times P_3$ is abelian.

\begin{theorem}{Recognition Theorem}~\label{thm:recognition~theorem}
  (DF, section 5.4, Theorem 9, p.~171)~\cite{DummitFoote}
  Suppose $H$ and $K$ are normal in $G$, and $H \cap K = 1$,
  then $G \simeq H \times K$.
\end{theorem}

\begin{corollary}
  If $G$ is a group of order $12$ with $P_2$ and $P_3$ as normal subgroups
  of order $2$ and $3$ respectively, then $G$ is abelian.
  
  \begin{proof}
    If both $P_2$ and $P_3$ are normal in $G$,
    then by Theorem~\ref{thm:recognition~theorem}, $G$ is isomorphic to $P_2 \times \P_3$.
    However, since both $P_2$ and $P_3$ are Sylow $p$-subgroups, they are cyclic
    and therefore abelian.
    By Proposition~\ref{prop:abelian-dir-prod-abelian}, $G$ is also abelian.
  \end{proof}
\end{corollary}

\subsection*{Case 2} Suppose $P_2 \trianglelefteq G$ and $P_2 \simeq C_4$.
Then $G$ is abelian.

Consider $\Aut(C_4)$, the set of all automorphisms of $C_4$.
By Proposition~\ref{prop:aut-cyclic}, $\Aut(C_4)$ is isomorphic to
${(\Z/4\Z)}^\times = \{ 1, 3 \}$.
Consider the homomorphism map $\psi : C_3 \to \Aut(C_4) \simeq {(\Z/4\Z)}_4^\times$.
Since ${(\Z/4\Z)}^\times$ has zero elements of order $3$,
$\psi$ is the trivial map. As we saw in case~\ref{case:nontrivial_psi},
$G$ is abelian and equivalent to the direct product $P_3 \times P_2$.

\subsection*{Case 3} Suppose $P_3 \trianglelefteq G$ and $P_2 \simeq Z_4$
but $P_2 \not trianglelefteq G$.
Let $P_2=\langle x \rangle$ and $P_3=\langle y \rangle$.
Then $x$ has order $4$ and $y$ has order $3$.
Since $P_3$ is normal in $G$, it must be closed under conjugation.
Keeping in mind that conjugation preserves order,
consider the element $y \in P_3$ of order $3$,
then its only possible conjugate is either itself or $y^2$.
However, $x$ cannot be in $Z(G)$ since groups of order $12$
have trivial center. Therefore, $xyx^{-1} = y^2 \sim y^{-1}$.

This uniquely determines the Cayley table for $G$.


Then $xyx^{-1}=y^{-1}$, then show that this uniquely determines the Cayley 
table of $G$, with presentation
\[ G \simeq \langle x,y \,|\, x^4=y^3=1, xyx^{-1}=y^{-1} \rangle. \]

\subsection*{Case 4} Suppose $P_3 \trianglelefteq G$ and 
$P_2 \simeq C_2 \times C_2$, but $P_2 \not\trianglelefteq G$.
Let $P_3 = \langle y \rangle$.
If $P_2$ has an element of order $4$, then $P_2$ is isomorphic to
$C_4$ as shown in case (2) above. It would follow that $P_2$ is normal,
but we know that $P_2$ is not normal therefore every element in $P_2$
must have order $2$.
Let $x_1$ be one such element, generating the group $\langle x_1 \rangle$.
Take an element $x_2 \in P_2 - \langle x_1 \rangle$,
then $x_2$ also generates a group of order $2$,
and $P_2 = \langle x_1 \rangle \times \langle x_2 \rangle$.
Consider that $\langle y \rangle$ has 2 elements of order $3$:
$y$ and $y^2 \sim y^{-1}$.
Since $P_2$ is not central in $G$ (group of order $12$ has trivial center),
$x_1 yx_1^{-1} \neq x_2 yx_2^{-1}$. By preservation of order under conjugation,
there are only two options for $x_1 yx_1^{-1}$ and $x_2 yx_2^{-1}$,
so we can reorder them such that $x_1 y x_1 ^{-1} = y$ and $x_2 y x_2^{-1} = y^{-1}$.
This uniquely determines the Cayley table for $G$,
with $y = r$ and $G \simeq D_{12}$.

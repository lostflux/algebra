\section{Setup}~\label{sec:setup}

\begin{Answer}
  \begin{enumalph}
    \item State Cauchy's theorem and the fundamental theorem of finite abelian groups 
      (giving references, but without proofs; if these are of interest, consider one of 
      the other projects!).  
    \item Recall the proof why every group of order $p^2$ with $p$ prime is abelian.  
    \item Classify the \emph{abelian} groups of order $n \leq 15$ up to isomorphism 
    using the fundamental theorem.
    \item Classify groups of order $6$ by hand: using Cauchy's theorem, there exists $a
    \in G$ of order $2$ and $b \in G$ of order $3$;
      show that $G=\{1,b,b^2,a,ab,ab^2\}$, in a direct manner that
      $ba=ab$ or $ba=ab^2$, and show that these two possibilities uniquely determine
      the Cayley table of $G$.
  \end{enumalph}
\end{Answer}

\begin{theorem}[Cauchy's Theorem]\label{thm:cauchy}
  If $G$ is a finite group and $p$ is a prime dividing $\abs{G}$,
  then $G$ has an element of order $p$.~\cite[p.~93,~Theorem~3.1]{DummitFoote} 
\end{theorem}

\begin{theorem}[The Fundamental Theorem of Finitely Generated Abelian Groups]~\label{thm:ftfgag}
  If a group $G$ is a finitely generated abelian group, then:
  \begin{align}
    G \isom \Z^r \times Z_{n_1} \times Z_{n_2} \times \cdots \times Z_{n_s}
    ~\label{eq:finite-generation}
  \end{align}
  where
  \begin{enumalph}
    \item $r \ge 0$ and $n_j \ge 1$ for all $j$; and
    \item $n_{i+1} \mid n_i$ for all $i$.
  \end{enumalph}
  And expression \ref{eq:finite-generation} is unique.
\end{theorem}

\begin{theorem}
  If $G$ is a finite group of order $p^2$ with $p$ prime, then $G$ is abelian.~\label{thm:p2-abelian}
\end{theorem}

\begin{proof}
  Let $G$ be a finite group of order $p^2$ with $p$ prime.
  Consider the center $Z(G) \le G$.
  By Lagrange's theorem~\ref{thm:lagrange}, we know that $\#Z(G) \mid \#G$.
  This implies that $\#Z(G) \in \{1,p,p^2\}$. Considering the following cases:
  \begin{enumalph}
    \item By the class equation~\cite[p.~125,~Theorem~8]{DummitFoote},
      we know the center of a group of prime power \emph{must be nontrivial}.
      Therefore, $\#Z(G) \ne 1$.
    \item If $G$ has an element of order $p^2$, then $G$ is cyclic, therefore
      abelian.
    \item Assuming $G$ does not have an element of order $p^2$,
      then every non-identity element must have order $p$ since it must divide $p^2$
      (see~\ref{thm:lagrange},~\ref{thm:cauchy}).
      Let $x$ be one such element, generating the subgroup $\langle x \rangle$ of order $p$.
      Let $y \in G - \langle x \rangle$, then $\langle y \rangle$ is also a group of order $p$.
      Then the group $\langle x, y \rangle \le G$ has order $p^2 = \#G$,
      implying that $\langle x, y \rangle = G$ (trivially,
      $\langle x \rangle \le G\ \land \langle y \rangle \le G \implies
      \langle x, y \rangle \le G$, and a subgroup of $G$ with the same order as $G$
      must be $G$ itself.) which generates a subgroup $\langle x \rangle$ of order $p$.
      In this case, $G = \langle x, y \rangle \isom Z_p \times Z_p$, and $G$
      is therefore abelian.
  \end{enumalph}
\end{proof}

Using the fundamental theorem, we now extend the earlier classification
of groups of prime order (\ref{tab:prime-groups})
to classify all abelian groups of order $\le 15$.

\begin{center}
  \begin{table}
    \begin{tabular}{ l | r | r | r }
      Order & Invariant Factors & Group & Isomorphic To \\
      \midrule
      $1$ & $1 \times 1$ & $C_1$ & $\{ \epsilon \}$, $S_1$ \\
      \midrule
      $2$ & $2 \times 1$ & $C_2$ & $\Z/2\Z$, $S_2$ \\
      \midrule
      $3$ & $3 \times 1$ & $C_3$ & $\Z/3\Z$ \\
      \midrule
      \multirow{2}{*}{$4$} & $4 \times 1$ & $C_4$ & $\Z/4\Z$ \\
          & $2 \times 2$ & $C_2 \times C_2$ & $V_4$ \\
      \midrule
      $5$ & $5 \times 1$ & $C_5$ & $\Z/5\Z$ \\
      \midrule
      $6$ & $6 \times 1$ & $C_6$ & $\Z/6\Z$ \\
      \midrule
      $7$ & $7 \times 1$ & $C_7$ & $\Z/7\Z$ \\
      \midrule
      \multirow{3}{*}{$8$} & $8 \times 1$ & $C_8$ & $\Z/8\Z$ \\
          & $4 \times 2$ & $C_4 \times C_2$ & \\
          & $2 \times 2 \times 2$ & $C_2 \times C_2 \times C_2$ &  \\
      \midrule
      \multirow{2}{*}{$9$} & $9 \times 1$ & $C_9$ & $\Z/9\Z$ \\
          & $3 \times 3$ & $C_3 \times C_3$ & \\
      \midrule
      $10$ & $10 \times 1$ & $C_{10}$ & $\Z/10\Z$ \\
      \midrule
      $11$ & $11 \times 1$ & $C_{11}$ & $\Z/11\Z$ \\
      \midrule
      \multirow{2}{*}{$12$} & $12 \times 1$ & $C_{12}$ & $\Z/12\Z$ \\
          & $6 \times 2$ & $C_6 \times C_2$ & \\
      \midrule
      $13$ & $13 \times 1$ & $C_{13}$ & $\Z/13\Z$ \\
      \midrule
      $14$ & $14 \times 1$ & $C_{14}$ & $\Z/14\Z$ \\
      \midrule
      $15$ & $15 \times 1$ & $C_{15}$ & $\Z/15\Z$ \\
      \bottomrule
    \end{tabular}~\label{tab:abelian-groups}
    \caption{Abelian Groups of order $n \le 15$}
  \end{table}
\end{center}

If we revisit the classification of groups of order $6$,
Cauchy's theorem tells us that any group of order $6$ must have
an element $a$ of order $2$ and an element $b$ of order $3$.
Consider the groups generated by these two elements:
\begin{align}
  \langle a \rangle & = \{ \epsilon, a \} \\
  \langle b \rangle & = \{ \epsilon, b, b^2 \} \\
  \langle a, b \rangle & = \{ \epsilon, b, b^2, a, ab, ab^2 \}~\label{eq:S_3}
\end{align}

In \ref{eq:S_3} above, we notice that $\langle a, b \rangle$ is isomorphic to $D_6$.
We can demonstrate that $\langle a, b \rangle$ is also isomorphic to $S_3$
under the isomorphism:
\begin{align*}
  \psi &\colon \langle a, b \rangle \to S_3 \\
  &\quad \epsilon \mapsto \epsilon \\
  &\quad a \mapsto (1\ 2) \\
  &\quad b \mapsto (1\ 2\ 3)
\end{align*}

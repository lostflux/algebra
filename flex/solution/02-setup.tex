\section{Setup}~\label{sec:setup}

\begin{Answer}
  \begin{enumalph}
    \item State Cauchy's theorem and the fundamental theorem of finite abelian groups 
      (giving references, but without proofs; if these are of interest, consider one of 
      the other projects!).  
    \item Recall the proof why every group of order $p^2$ with $p$ prime is abelian.  
    \item Classify the \emph{abelian} groups of order $n \leq 15$ up to isomorphism 
    using the fundamental theorem.
    \item Classify groups of order $6$ by hand: using Cauchy's theorem, there exists $a
    \in G$ of order $2$ and $b \in G$ of order $3$;
      show that $G=\{1,b,b^2,a,ab,ab^2\}$, in a direct manner that
      $ba=ab$ or $ba=ab^2$, and show that these two possibilities uniquely determine
      the Cayley table of $G$.
  \end{enumalph}
\end{Answer}


We start with a motivating example.  Recall the Cayley table for $D_6$, the dihedral group of order $6$:
\[ \begin{array}{c  cccccccc} 
& 1 & r & r^2 & s & sr & sr^2 \\
\toprule
1 & 1 & r & r^2 & s & sr & sr^2 \\
r & r & r^2 & 1 & sr^2 & s & sr \\
r^2 & r^2 & 1 & r & sr & sr^2 & s \\
s & s & sr & sr^2 & 1 & r & r^2 \\
sr & sr & sr^2 & s & r^2 & 1 & r \\
sr^2 & sr^2 & s & sr & r & r^2 & 1
\end{array} \]

We observed that Cayley table have the Sudoku property, as in the following lemma.

\begin{lem}
Each row (and column) of the Cayley table of a finite group $G$ contains all elements of $G$.
\end{lem}

As a reminder, this lemma follows directly from the cancellation law.

Returning to the above example, if we pick off just one row---say the row $sr$---by this property we get a permutation of the set $D_6$:
\begin{equation}  \label{eqn:sr}
\begin{pmatrix}
1 & r & r^2 & s & sr & sr^2 \\
sr & sr^2 & s & r^2 & 1 & r 
\end{pmatrix} 
\end{equation}
We denote this element $\sigma_{sr} \colon D_6 \to D_6$, since it is a symmetry that depends on $sr$: it is defined by $\sigma_{sr}(sr) = 1$, \dots, $\sigma_{sr}(sr^2)=r$, reading the input from the top row of the table and the output from the bottom row.  This is visibly a bijection from $D_6$ to itself: each element of $D_6$ appears exactly once.  

Recall that we write the set of bijections from a set $A$ to itself as
\[ \Sym(A) \colonequals \{\sigma \colon A \to A \textup{ bijection}\} \]
and this forms a group under composition.  This works for every set $A$, even though we mostly worked with $A=\{1,\dots,n\}$ and then abbreviate $S_n=\Sym(\{1,\dots,n\})$.  There is no loss of generality here.

\begin{lem}~\label{lem:An}
If $A$ is a finite set with $\#A=n$, then the groups $\Sym(A) \simeq S_n$ are isomorphic.
\end{lem}

\begin{proof}
Since $\#A=n$, there is a bijection from $A$ to $\{1,\dots,n\}$.  Each permutation of the elements of $A$ gives a permutation of the elements $\{1,\dots,n\}$ by how they are numbered.  
\end{proof}

Putting these together, we can define a function
\begin{equation}~\label{eqn:D3map}
\begin{aligned}
\sigma \colon D_6 &\to \Sym(D_6) \simeq S_6 \\
1 &\mapsto \sigma_1 = \begin{pmatrix}
1 & r & r^2 & s & sr & sr^2 \\
1 & r & r^2 & s & sr & sr^2
\end{pmatrix} \\
&\vdots \\
sr^2 &\mapsto \sigma_{sr^2} = \begin{pmatrix}
1 & r & r^2 & s & sr & sr^2 \\
sr^2 & s & sr & r & r^2 & 1
\end{pmatrix} 
\end{aligned}
\end{equation}
Usually we write functions like $f(x)$ with input $x$ from the domain; but here the output is itself a function which wants input, so in order not to get confused, we use a subscript.  

That is a start, but of course in group theory we want more than just a map of sets: we want to know it is a homomorphism!  One case of the homomorphism property in this example would read
\begin{equation} 
\sigma_{sr}\sigma_{r} \overset{?}{=} \sigma_{sr^2} 
\end{equation}
This is an equality we need to check on the right-hand side of \eqref{eqn:D3map}.  Composing the permutations, we see it checks out!  Once we have a homomorphism, we can also see that the kernel of the map $\sigma$ consists only of the identity: if an element maps to the identity permutation in $\Sym(D_6)$, it would come from a row of the Cayley table where they elements line up according to the identity, and that happens only for the top row.  So we get an injective map $D_6 \hookrightarrow S_6$.  By the fundamental homomorphism theorem, we see that $D_6$ is isomorphic to its image under this map; the following lemma reminds us of how this works in general.

\begin{thm}[Fundamental homomorphism theorem]
Let $\phi \colon G \to H$ be a group homomorphism.  Then $G/\ker \phi \simeq \phi(G) \leq H$.
\end{thm}

\begin{corollary}~\label{cor:injG}
If $\phi \colon G \to H$ is an injective group homomorphism, then $G \simeq \phi(G)$.
\end{corollary}

\begin{proof}
If $\phi$ is injective, then $\ker \phi =\{1\}$, and then $G/\ker \phi \simeq G$ (the cosets of the identity are just the elements of $G$!).
\end{proof}

We conclude that $D_6$ is isomorphic to a subgroup of $S_6$.  This is Cayley's theorem!  

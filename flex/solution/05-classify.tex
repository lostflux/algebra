\section{Full Classification}~\label{sec:class}

We may now classify all groups of order $n \le 15$, except those of order $12$.

For $n = 10$, note that $10 = 2 \cdot 5$, which is a product of primes.
Also note that $2 \mid (5 - 1)$.
Let $G$ be a group of order $10$.
There are two possibilities:
\begin{enumalph}
  \item If $G$ contains an element of order $10$, then $G$ is isomorphic to $C_{10}$.
    Therefore, $G$ is cyclic and abelian.
  \item If $G$ does not contain an element of order $10$,
    then $G$ must have an element of order $2$ and an element of order $5$.
    If $p \in G, \abs{p} = 2$ and $q \in G, \abs{q} = 5$,
    then let's define $P = \langle p \rangle$ and $Q = \langle q \rangle$.
    Consider the set $\Aut(Q)$ of automorphisms of $Q$, with $Q \isom C_5$.
    As demonstrated by equation~\ref{eqn:aut_p},
    \begin{align}
      \Aut(Q) \simeq {(Z/5Z)}^\times \simeq C_4
    \end{align}
    In this case, $\Aut(Q)$ contains the unique cyclic subgroup $2\Z/4\Z \simeq C_2$
    of order $2$. The homomorphism between the two groups is defined as:
    \begin{align*}
      \psi_i &: C_2 \to 2\Z/4\Z \\
           & 0 \mapsto 0 \\
           & 1 \mapsto 2 \\
    \end{align*}
    More generally, $\psi_i (y) = 2 \gamma \sim \gamma^2$.
    Therefore, $i = 2$, and the homomorphism is nontrivial.
    This means $G \simeq C_5 \rtimes_{\psi_2} C_2$, making $G$ nonabelian
    by \ref{pro:nonabelian}.
    This group is more commonly seen as the dihedral group $D_{10}$,
    generated by $r$ as an element of order $5$ with
    $Q = \langle r \rangle = \{ 1, r, r^2, r^3, r^4 \}$,
    and $s$ as an element of order $2$ with $P = \langle s \rangle = \{ 1, s \}$.
    The homomorphism 
\end{enumalph}

For $n = 11$, remember that $11$ is prime.
Therefore, the only group of order $11$ is $C_{11}$, which is abelian.
Similarly, for $n = 13$, $C_{13}$ is the only group of order $13$.

For $n = 14$, note that $14 = 2 \cdot 7$, which is a product of primes,
and $2 \mid (7 - 1)$.
Let $G$ be a group of order $14$.
Like above, there are two possibilities:
\begin{enumerate}
  \item If $G$ contains an element of order $14$, then $G$ is isomorphic to $C_{14}$.
    Therefore, $G$ is cyclic and abelian.
  \item Otherwise, $G$ must contain an element of order $7$ and an element of order $2$.
    If $p \in G, \abs{p} = 2$ and $q \in G, \abs{q} = 7$,
    then let's define $P = \langle p \rangle$ and $Q = \langle q \rangle$.
    Consider the set $\Aut(Q)$ of automorphisms of $Q$, with $Q \isom C_7$.
    As demonstrated by equation~\ref{eqn:aut_p},
    \begin{align}
      \Aut(Q) \simeq {(Z/7Z)}^\times \simeq C_6
    \end{align}
    In this case, $\Aut(Q)$ contains the unique cyclic subgroup $3\Z/6\Z \simeq C_3$
    of order $3$. The homomorphism between the two groups is defined as:
    \begin{align*}
      \psi_i &: C_2 \to 2\Z/6\Z \\
           & 0 \mapsto 0 \\
           & 1 \mapsto 3 \\
    \end{align*}
    More generally, $\psi_i (y) = 3 \gamma \sim \gamma^3$.
    Therefore, $i = 3$, and the homomorphism is nontrivial.
    This means $G \simeq C_7 \rtimes_{\psi_3} C_2$, making $G$ nonabelian
    by \ref{pro:nonabelian}.
    This group is more commonly seen as the dihedral group $D_{14}$
    with $r$ as an element of order $7$ with
    $Q = \langle r \rangle = \{ 1, r, r^2, r^3, r^4, r^5, r^6 \}$,
    and $s$ as an element of order $2$ with $P = \langle s \rangle = \{ 1, s \}$.
\end{enumerate}

For $n = 15$, we once again have a product of primes, $15 = 3 \cdot 5$.
However, $3 \nmid (5 - 1)$, so we have the case that any group of order $15$
is abelian (as shown in ~\ref{eqn:p_not_div_q-1}), therefore isometric to $C_{15}$.

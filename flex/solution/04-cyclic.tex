\section{Cyclic Groups of Non-Prime Order}~\label{sec:cyclic-non-prime}

In proposition~\ref{prop:n-has-n-cyclic}, we saw that any group of order $n$
(without loss of generality)
that contains an element of order $n$ is cyclic.
For an arbitrary element $g \in G$, let $x$ be the order of $g$.
As we saw in proposition~\ref{prop:prime-order-p},
the set $\langle g \rangle$ generated by $g$ contains $x$ elements
and is a subgroup of $G$.
This restricts $x$ to be a divisor of $n$.
But $n$ divides $n$, so $x$ might be $n$
(not an occurrence in \emph{every} group,
but an occurrence in at least one group of order $n$ -- for instance,
take $1$ in $\Z/n\Z$).
Since the order of $G$ is $n$, Proposition \ref{prop:n-has-n-cyclic} tells us
that $G$ is cyclic, Proposition~\ref{prop:cyclic-same-order-isomorphic}
tells us that all such groups are isomorphic for fixed $n$,
and Proposition \ref{prop:cyclic-then-abelian} tells us that
$G$ is abelian equivalent to the group $C_n$.

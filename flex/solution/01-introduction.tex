\section{Introduction}\label{sec:intro}

\begin{Answer}
  Motivate the classification of groups of small order up to isomorphism; 
  quickly review the case of groups of prime order, providing context to why the 
  factorization of the order $n=\#G$ is reflected in the list of possible groups of 
  order $n$ up to isomorphism.
\end{Answer}

In attempting to understand the structure of groups,
it is often important to understand smaller structures within the group ---
and these may include kernels, orbits, and most-importantly subgroups.
Since proper subgroups generally have smaller order,
they can be more easily understood.
But how do we find -- for certain -- \emph{all} the \emph{important} subgroups
in a group? 
First, we must motivate the unique classification of groups.
Some groups have similar structure --- take, for instance,
the two groups
\begin{align}
  S_2 &= \{ \epsilon, (1\ 2) \}~\label{ex:S_2} \\
  Z_2 &= {(\Z/2\Z)}^+ = \{ 0, 1 \}~\label{ex:Z_2}
\end{align}
Through some experimentation, we notice
that each group has order $2$, is commutative, cyclic, and the non-identity element is in-fact
its own inverse -- i.e. it has order $2$. Therefore, we can transfer
any function from one to the other by mapping $\epsilon \leftrightarrow 0$
and $(1\ 2) \leftrightarrow 1$. We call such a map a \emph{group isomorphism},
and we say $S_2$ and $(Z_2, +)$ are \emph{isomorphic}.
We often need to uniquely identify all groups of a given structure
For instance, we may consider examples~\ref{ex:S_2}~and~\ref{ex:Z_2} above to be
under the class $C_2$~\label{def:C_2}, the cyclic groups of order $2$.
We can trivially show that there is a single cyclic group of order $2$.


In example \ref{ex:Z_2} above, we can note that the group has only two possible
subgroups: the trivial group $\{ \epsilon \}$ and the group itself.
But is this always the case?
Given a group $G$, when can we expect to find subgroups of other orders than $1$ and $\#G$?
Is there any fundamental difference between such groups that only have subgroups of order $1$ 
and $\#G$, and those that have subgroups of different orders?
Let us recall Lagrange's theorem:
\begin{theorem}[Lagrange]\label{thm:lagrange}
  If $G$ is a finite group and $H$ is a subgroup of $G$,
  then $\abs{H}$ divides $\abs{G}$ and the number of left cosets of $H$
  in $G$ is equal to $\frac{\abs{H}}{\abs{G}}$.~\cite[p.89, Theorem 8]{DummitFoote}
\end{theorem}
The first part of lagrange's theorem~(\ref{thm:lagrange}), while offering no guarantees on the existence of
subgroups of given orders, tells us that the order of any
subgroup $S \le G$ divides the order of $G$.
We can immediately pick out that all groups of prime order $p$ must only have the
trivial group as a proper subgroup, since their order $p$ is only divisible by $1$ and $p$.
Furthermore, consider that the order of every element in the group must divide $p$.
Since the order of non-identity elements is clearly greater than $1$, we can conclude
that the order of every non-identity element in the group is $p$, hence the group is cyclic.

\newpage
\begin{theorem}\label{thm:cyclic-abelian}
  Every cyclic group is abelian.
\end{theorem}

\begin{proof}
  Recall that cyclic groups can be denoted as the powers of a single element, $g$,
  known as the \emph{generator} of the group.
  Consider two elements, $x = g^a$ and $y = g^b$.
  Then, $xy = g^a g^b = g^{a+b} = g^b g^a = yx$.
\end{proof}

Therefore, every group of prime order $p$ is cyclic and abelian.
We call $C_p$~\label{def:C_p} the class of cyclic groups of order $p$,
and every other group of order $p$ is isomorphic to the $C_p$.

\begin{center}
  \begin{table}~\label{tab:prime-groups}
    \begin{tabular}{ l r r }
      Order & Group & Isomorphisms \\
      \midrule
      $1$ & $C_1$ & $\{ \epsilon \}$, $S_1$ \\
      $2$ & $C_2$ & $\Z/2\Z$, $S_2$ \\
      $3$ & $C_3$ & $\Z/3\Z$ \\
      $5$ & $C_5$ & $\Z/5\Z$ \\
      $7$ & $C_7$ & $\Z/7\Z$ \\
      $11$ & $C_{11}$ & $\Z/11\Z$ \\
      $13$ & $C_{13}$ & $\Z/13\Z$ \\
    \end{tabular}
    \caption{(Abelian) Groups of Prime Order $n \le 15$}
  \end{table}
\end{center}

\bigskip
For groups of non-prime order $\#G = n$, Lagrange (Theorem~\ref{thm:lagrange}) tells
us that any subgroup $S \le G$ must have an order $\#S$ equal to one of the divisors of $n$,
and it follows that the left cosets of $S$ in $G$ is equal to $\frac{n}{\#S}$.

When groups are first introduced, they are described as sets equipped with a binary operation
satisfying certain axioms (associativity, identity, inverses).
As examples, we considered the symmetric groups $S_n$, the group of permutations of the set
$\{1,\dots,n\}$.  However, these are not so far apart.  Even for Galois (see 
Dummit--Foote~\cite[p.~14 (3)]{DummitFoote}), groups were made of ``substitutions''---i.e., 
Galois was working with permutation groups!  

For now, we restrict attention to \emph{finite} groups (but see section~\ref{sec:conc} below 
for infinite groups).  So the symmetric groups $S_n$ (for $n \geq 1$) are finite groups.  Is 
every finite a group a permutation group?  No: we have $\#S_n=n!$, so a group of order $4$ 
cannot be isomorphic to $S_n$ since $2! < 4 < 3!$.  But if we all ourselves \emph{subgroups} 
of permutation groups, the answer is yes.  Our main result is as follows.

% \begin{thm}[Cayley] \label{thm:cayley}
% Every finite group is isomorphic to a subgroup of $S_n$ for some $n \geq 1$.
% \end{thm}

% \subsection*{Contents} In section \ref{sec:setup}, we get set up by describing how the group 
% operation naturally describes permutations of the elements of the group.  We then prove 
% Cayley's theorem in section \ref{sec:thm}.  We then conclude in section \ref{sec:conc} with 
% some applications and next steps.

\section{Introduction}\label{sec:intro}

\begin{Answer}
  Motivate the classification of groups of small order up to isomorphism; 
  quickly review the case of groups of prime order, providing context to why the 
  factorization of the order $n=\#G$ is reflected in the list of possible groups of 
  order $n$ up to isomorphism.
\end{Answer}

When groups are first introduced, they are described as sets equipped with a binary operation
satisfying certain axioms (associativity, identity, inverses).
As examples, we considered the symmetric groups $S_n$, the group of permutations of the set
$\{1,\dots,n\}$.  However, these are not so far apart.  Even for Galois (see 
Dummit--Foote~\cite[p.~14 (3)]{DummitFoote}), groups were made of ``substitutions''---i.e., 
Galois was working with permutation groups!  

For now, we restrict attention to \emph{finite} groups (but see section~\ref{sec:conc} below 
for infinite groups).  So the symmetric groups $S_n$ (for $n \geq 1$) are finite groups.  Is 
every finite a group a permutation group?  No: we have $\#S_n=n!$, so a group of order $4$ 
cannot be isomorphic to $S_n$ since $2! < 4 < 3!$.  But if we all ourselves \emph{subgroups} 
of permutation groups, the answer is yes.  Our main result is as follows.

\begin{thm}[Cayley] \label{thm:cayley}
Every finite group is isomorphic to a subgroup of $S_n$ for some $n \geq 1$.
\end{thm}

\subsection*{Contents} In section \ref{sec:setup}, we get set up by describing how the group 
operation naturally describes permutations of the elements of the group.  We then prove 
Cayley's theorem in section \ref{sec:thm}.  We then conclude in section \ref{sec:conc} with 
some applications and next steps.

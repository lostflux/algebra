\section{Introduction}\label{sec:intro}

Groups are fundamental to life.
Indeed, many people have manipulated groups without realizing,
or, tragically, without knowing what groups are.
Take, for instance, a machine learning engineer using
neural networks to coerce sentiments out of text.
It is more obvious that linear algebra is involved --
since vectors are used to represent the text data
and matrices are used to represent the neural network --
but, fundamentally, we can still interpret the process as
a map between two abstract sets, that of all the text
and that of all possible sentiments.
We may also observe some likenesses of homomorphisms.
For instance, on the domain $-1 \le s \le 1$,
suppose good sentiments are positive and bad sentiments are negative,
le's say $\phi(\text{``bad''}) = -1$ and $\phi(\text{``good''}) = 1$.
Let's say I love food, so I give it an extremely positive sentiment of $0.8$.
In that case, I may have savored a lot of different dishes, so I
necessarily have a high bar for food. \emph{I do not appreciate bad food;
in fact, I hate it.} Then my sentiment to bad food will be negative.
In the neural network, this may be represented as:
\[
  \phi(\text{``bad food''}) = \phi(\text{``bad''}) \cdot \phi(\text{``food''}) = -1 \cdot 0.8 = -0.8
\]
The more I value something, the more I want it as perfect as possible.
If I do not care about something, then I assign it as neutral a sentiment as possible.
A group theorist would note that I may not include $0$ in my range of sentiments
since $0$ is a multiplicative annihilator ($0a = a0 = 0$ for all $a$),
so I may assign a neutral sentiment infinitesimal value.
Take, for instance, that I do not drink, therefore I do not care how good
or bad an alcoholic drink is and I assign $\phi(\text{``alcoholic drink''}) = \eps \approx 0.000001$.
By extension, my sentiment to a British alcoholic drink is
$\phi(\text{``British alcoholic drink''}) = \phi(\text{``British''}) \cdot \eps \approx \eps$,
and the same holds for all other types of alcoholic drinks.
Similarly, if ``not'' is assigned a negative sentiment $-x$,
then $\phi(\text{``not bad''}) = \phi(\text{``not''}) \cdot \phi(\text{``bad''})
= (-x) \cdot (-1) = x$, a positive sentiment reflecting the double negation.
Arguably, the two domains may be represented as groups, with the
sentiment map as some homomorphism of groups.
So why is this useful?

In some instances, groups and other structures from abstract algebra
shed new light on problems.
Take, for instance, the use of Rijndael fields in cryptography.
Viewed alone, the Advanced Encryption Standard (AES) algorithm
operating on binary numbers makes little sense:
an $\mathbf{xor}$ here, a substitution there, a $\mathbf{shift}$,
an addition... And good luck proving that it is unbreakable.
However, we can interpret AES as operations in the Rijndael field
$F = \F_2[x]/f(x)$, where $\F_2[x]$ is the set of polynomials over $\F_2$
and $f(x)$, the encryption key, is also a fixed polynomial in $\F_2[x]$.
We can then prove that AES is practically unbreakable 
using the group-theoretic properties of the field $F$ (*cite).

Similarly, algebraic structures such as groups are reflected in many scenarios.
However, to recognize such instances, we first need to understand
and identify the unique groups that exist of specific sizes
(or, in algebraic terms, we say the groups \emph{up to isomorphism}).
This paper attempts to identify all groups orders (sizes) less than or equal to $15$.

\subsection*{Contents}
In section~\ref{sec:trivial}, we start by looking at the trivial group, the simplest group to identify.
In section~\ref{sec:prime-order}, we look at groups of prime order.
We then look at cyclic groups of non-prime order in section~\ref{sec:cyclic-non-prime}.
In section~
We then look at groups of order $p^2$ with $p$ prime in section .
In section 4, we return to groups of order $6$ and classify those that behave differently.
In section 5, we look groups of order $8$ and classify those behave differently.
We will now have classified all groups of order $n \le 9$.
In section 6, we classify all groups of order $10$, $14$, and $14$.
In section 7, we finally classify groups of order $12$.
Finally, in section 8, we review the results and look at how
some occurrences of the groups can be identified.

\section{Introduction}\label{sec:intro}

Groups are fundamental to life.
Take, for instance, a machine learning engineer using
neural networks to coerce sentiments out of text.
It is more obvious that linear algebra is involved --
since vectors are used to represent the text data
and matrices are used to represent the neural network --
but, fundamentally, we can still interpret the process as
a map between two abstract sets, that of all the text
and that of all possible sentiments.
We may also observe some likenesses of group actions and homomorphisms.
For instance, on the continuous domain $-1 \le s \le 1$,
suppose good sentiments are positive and bad sentiments are negative,
le's say $\phi(\text{``bad''}) = -1$ and $\phi(\text{``good''}) = 1$.
Let's say I love food so I give it an extremely positive sentiment of $0.8$.
In that case, I may have savored a lot of different dishes, so I
necessarily have a high bar since I have tried many dishes and hotels.
Then my sentiment to bad food will be negative.
In the neural network, this may be represented as:
\[
  \phi(\text{``bad food''}) = \phi(\text{``bad''}) \cdot \phi(\text{``food''}) = -1 \cdot 0.8 = -0.8
\]
The more I value something, the more I want it as good as possible.
If I do not care about something, then I assign it as neutral a sentiment as possible.
We may not include $0$ since it is
multiplicative annihilator ($0a = a0 = 0$ for all $a$),
so I assign a neutral sentiment infinitesimal value.
Take, for instance, that I do not drink, so I assign $\phi(\text{``alcoholic drink''}) = \eps \approx 0.000001$.
By extension, my sentiment to a British alcoholic drink is
$\phi(\text{``British alcoholic drink''}) = \phi(\text{``British''}) \cdot \eps \approx \eps$,
and the same holds for all other types of alcoholic drinks.
Similarly, if ``not'' is assigned a negative sentiment $-x$,
then $\phi(\text{``not bad''}) = \phi(\text{``not''}) \cdot \phi(\text{``bad''})
= (-x) \cdot (-1) = x$, a positive sentiment reflecting the double negation.
Arguably, the two domains may be represented as groups, with the
sentiment map as some homomorphism of groups.
So why is this useful?

In some instances, groups and other structures from abstract algebra
shed new light on problems.
Take for instance, the use of Rijndael fields in cryptography.
Viewed alone, the Advanced Encryption Standard (AES) algorithm
operating on binary numbers makes little sense:
an $\mathbf{xor}$ here, a substitution there, a $\mathbf{shift}$,
an addition... And good luck proving that it is unbreakable.
However, we can interpret AES as operations in the Rijndael field
$F = \F_2[x]/f(x)$, where $\F_2[x]$ is the set of polynomials over $\F_2$
and $f(x)$, the encryption key, is also a fixed polynomial in $\F_2[x]$.
We can then prove that AES is practically unbreakable 
using the group-theoretic properties of the field $F$.

Similarly, algebraic structures such as groups are reflected in many scenarios.
However, to recognize such instances, we first need to understand
and identify the unique groups that exist of specific sizes
(we say the groups \emph{up to isomorphism}).
This paper attempts to identify all groups of orders (sizes) less than or equal to $15$.

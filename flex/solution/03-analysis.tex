\newpage
\section{Further Analysis}~\label{sec:analysis}

\begin{Answer}
  \begin{enumalph}
    \item Show that every nonabelian group $G$ of order $8$ is isomorphic to either 
    $D_8$ or $Q_8$, as follows.
    \begin{itemize}
    \item Show that $G$ has an element $a \in G$ of order $4$.  \emph{[Hint: what 
    happens if every nonidentity element has order $2$?]}
    \item Let $H \colonequals \langle a \rangle$ and let $b \not \in H$.  Observe that 
    $H \trianglelefteq G$ is normal; argue that $bab^{-1}=a^3$ (the order under 
    conjugation is preserved), and then that $G \simeq D_4,Q_8$ according as $b$ has 
    order $2$ or $4$.  
    \end{itemize}
    \item Pause to show we have classified groups of order $n \leq 9$.
  \end{enumalph}
\end{Answer}

Now, let us consider the nonabelian groups of order $n = 8$.
Note that $8 = 2^3$.
Let $G$ be a nonabelian group of order $8$.
By Lagrange's theorem (\ref{thm:lagrange}),
non-identity elements of $G$ may have orders $2, 4, 8$.
If $G$ has an element $a$ of order $8$, then $G$ is cyclic, therefore abelian.
On the other hand, if every nonidentity element $b_i \in G$ has order $2$, then:
\begin{enumalph}
  \item For any $b_i \in G$, $\langle b_i \rangle \le G$ is of prime order $p = 2$
    and is therefore abelian.
  \item Taking $b_j \in G, b_j \notin \langle b_i \rangle$,
    then $\langle bj \rangle \le G$ is also an abelian group of order $2$.
    Then $\langle b_i, b_j \rangle \le G$ is an abelian group of order $4$
    (direct product of abelian groups).
  \item Taking $b_k \in G, b_k \notin \langle b_i, b_j \rangle$,
    then $\langle b_k \rangle \le G$ is also an abelian group of order $2$.
    Then $\langle b_i, b_j, b_k \rangle \le G$ is an abelian group of order $8$
    (direct product of abelian groups).
  \item Since $\#G = 8$ and $\#\langle b_i, b_j, b_k \rangle = 8$
    yet $\langle b_i, b_j, b_k \rangle \le G$, it must be the case
    that $\langle b_i, b_j, b_k \rangle = G$.
  \item Therefore, if every non-identity element in $G$ has order $2$, then $G$ is abelian.
    Particularly, $G \isom C_2 \times C_2 \times C_2$.
\end{enumalph}

Therefore, if $G$ is a \emph{nonabelian} group of order $8$, then $G$
must have an element of order $4$.
Let $a \in G$ be such an element, with $H \colonequals \langle a \rangle$.
Then $\#H = 4$, and $H$ is cyclic meaning $a^{-1} = a^3$.
Let $b \in G, b \notin H$. Then $b$ may have order $2$ or order $4$.

\begin{enumerate}
  \item If $b$ has order $2$, then $b^2 = 1$, and
    $bab^{-1} = b(ab^{-1}) = b^2 a^3 = a^3 \in H$.
    Therefore $H \trianglelefteq G$ is normal in $G$ and
    $G$ is isometric to $D_8$.
  \item If $b$ has order $4$, then $G$ is isomorphic to $Q_8$.
    Particularly, $G = \langle x, y, z \rangle$ where $x^2 = y^2 = z^2 = -1$,
    each of $x, y, z$ has order $4$, and $xy = z, yz = x, zx = y$ and
    $yx = -z, zy = -x, xz = -y$.
    It follows that $aba^{-1} = ab(-a) = c(-a) = -ca = -b$ for any
    permutations of $a, b, c \in \{x, y, z \}$.
    Therefore, any subgroup $H = \langle a \rangle = \{1, a, -1, -a \}$
    having $a \in \{ x, y, z \}$ is normal in $G$.
\end{enumerate}

We have now classified all nonabelian groups of order $8$,
and all groups of order $n \leq 9$ (since $9 = 3^2$ is a square of a prime,
it may only have the abelian group $C_3 \times C_3$ and $C_9$).


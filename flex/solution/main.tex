\title{\Huge{Math 71: Algebra\\Groups of Small Order}}
\author{\Large{Amittai Siavava}}
\date{\Large{\today}}
\documentclass[11pt, reqno]{amsart}

\input{../../common/macros.tex}
\usepackage[backend=bibtex,style=science]{biblatex}
\bibliography{main.bib}
\pgfplotsset{compat=1.18}

\pagestyle{fancy}                       % fancy (allow headers, footers)
\fancyhf{}                              % clear all header/footer settings.
\cfoot{\thepage}                        % set page-numbers in footer.
% \lhead{\textit{\textbf{ Amittai, S}}}   % set name in header, left.
% \rhead{\textsc{Math 71: Algebra}}       % set class name in header, right.
\renewcommand{\headrulewidth}{0pt}
\renewcommand{\footrulewidth}{.3pt}


\newcounter{problem}
\setcounter{problem}{0}

\renewcommand{\theenumi}{\alph{enumi}}

\begin{document}

\setlength{\headheight}{13.0pt}
\setlength{\footskip}{15.0pt}


% TITLE
% \PSET{6 --- \today}{Fall 2022}{Voight}{Amittai Siavava}{Math 71: Algebra}




\title{\Huge{Cayley's Theorem}}
\date{\Large{\today}}


\begin{titlingpage}
  \maketitle
  
  \begin{abstract}
    \noindent This abstract describes the work. \emph{Sage}.
  \end{abstract}
  
  \begin{figure} [b!]
    \centering
    \includegraphics[width=0.8\textwidth]{Decoratie.png}\label{Decoratie}
  \end{figure}


  % CREDIT STATEMENT, Centered on a new page.
  \newpage
  \hspace{0pt}
  \vfill
  \CreditStatement{
    I worked on these problems alone,
    with reference to class notes and the following books:
    \begin{enumalph}
      \item \textit{\textbf{Abstract Algebra}} by \textbf {David S. Dummit \& Richard M. Foote}.
      \item \textit{\textbf{Algebra}} by \textbf {Jacob K. Goldhaber \& Gertrude Ehrlich}
    \end{enumalph}
  }
  \vfill
\end{titlingpage}

\bigskip

% INTRODUCTION
\section{Introduction}\label{sec:intro}

Groups are fundamental to life.
Indeed, many people have manipulated groups without realizing,
or, tragically, without knowing what groups are.
Take, for instance, a machine learning engineer using
neural networks to coerce sentiments out of text.
It is more obvious that linear algebra is involved --
since vectors are used to represent the text data
and matrices are used to represent the neural network --
but, fundamentally, we can still interpret the process as
a map between two abstract sets, that of all the text
and that of all possible sentiments.
We may also observe some likenesses of homomorphisms.
For instance, on the domain $-1 \le s \le 1$,
suppose good sentiments are positive and bad sentiments are negative,
le's say $\phi(\text{``bad''}) = -1$ and $\phi(\text{``good''}) = 1$.
Let's say I love food, so I give it an extremely positive sentiment of $0.8$.
In that case, I may have savored a lot of different dishes, so I
necessarily have a high bar for food. \emph{I do not appreciate bad food;
in fact, I hate it.} Then my sentiment to bad food will be negative.
In the neural network, this may be represented as:
\[
  \phi(\text{``bad food''}) = \phi(\text{``bad''}) \cdot \phi(\text{``food''}) = -1 \cdot 0.8 = -0.8
\]
The more I value something, the more I want it as perfect as possible.
If I do not care about something, then I assign it as neutral a sentiment as possible.
A group theorist would note that I may not include $0$ in my range of sentiments
since $0$ is a multiplicative annihilator ($0a = a0 = 0$ for all $a$),
so I may assign a neutral sentiment infinitesimal value.
Take, for instance, that I do not drink, therefore I do not care how good
or bad an alcoholic drink is and I assign $\phi(\text{``alcoholic drink''}) = \eps \approx 0.000001$.
By extension, my sentiment to a British alcoholic drink is
$\phi(\text{``British alcoholic drink''}) = \phi(\text{``British''}) \cdot \eps \approx \eps$,
and the same holds for all other types of alcoholic drinks.
Similarly, if ``not'' is assigned a negative sentiment $-x$,
then $\phi(\text{``not bad''}) = \phi(\text{``not''}) \cdot \phi(\text{``bad''})
= (-x) \cdot (-1) = x$, a positive sentiment reflecting the double negation.
Arguably, the two domains may be represented as groups, with the
sentiment map as some homomorphism of groups.
So why is this useful?

In some instances, groups and other structures from abstract algebra
shed new light on problems.
Take, for instance, the use of Rijndael fields in cryptography.
Viewed alone, the Advanced Encryption Standard (AES) algorithm
operating on binary numbers makes little sense:
an $\mathbf{xor}$ here, a substitution there, a $\mathbf{shift}$,
an addition... And good luck proving that it is unbreakable.
However, we can interpret AES as operations in the Rijndael field
$F = \F_2[x]/f(x)$, where $\F_2[x]$ is the set of polynomials over $\F_2$
and $f(x)$, the encryption key, is also a fixed polynomial in $\F_2[x]$.
We can then prove that AES is practically unbreakable 
using the group-theoretic properties of the field $F$ (*cite).

Similarly, algebraic structures such as groups are reflected in many scenarios.
However, to recognize such instances, we first need to understand
and identify the unique groups that exist of specific sizes
(or, in algebraic terms, we say the groups \emph{up to isomorphism}).
This paper attempts to identify all groups orders (sizes) less than or equal to $15$.

\subsection*{Contents}
In section~\ref{sec:trivial}, we start by looking at the trivial group, the simplest group to identify.
In section~\ref{sec:prime-order}, we look at groups of prime order.
We then look at cyclic groups of non-prime order in section~\ref{sec:cyclic-non-prime}.
In section~
We then look at groups of order $p^2$ with $p$ prime in section .
In section 4, we return to groups of order $6$ and classify those that behave differently.
In section 5, we look groups of order $8$ and classify those behave differently.
We will now have classified all groups of order $n \le 9$.
In section 6, we classify all groups of order $10$, $14$, and $14$.
In section 7, we finally classify groups of order $12$.
Finally, in section 8, we review the results and look at how
some occurrences of the groups can be identified.

\section{Setup}~\label{sec:setup}

\begin{Answer}
  \begin{enumalph}
    \item State Cauchy's theorem and the fundamental theorem of finite abelian groups 
      (giving references, but without proofs; if these are of interest, consider one of 
      the other projects!).  
    \item Recall the proof why every group of order $p^2$ with $p$ prime is abelian.  
    \item Classify the \emph{abelian} groups of order $n \leq 15$ up to isomorphism 
    using the fundamental theorem.
    \item Classify groups of order $6$ by hand: using Cauchy's theorem, there exists $a
    \in G$ of order $2$ and $b \in G$ of order $3$;
      show that $G=\{1,b,b^2,a,ab,ab^2\}$, in a direct manner that
      $ba=ab$ or $ba=ab^2$, and show that these two possibilities uniquely determine
      the Cayley table of $G$.
  \end{enumalph}
\end{Answer}


We start with a motivating example.  Recall the Cayley table for $D_6$, the dihedral group of order $6$:
\[ \begin{array}{c  cccccccc} 
& 1 & r & r^2 & s & sr & sr^2 \\
\toprule
1 & 1 & r & r^2 & s & sr & sr^2 \\
r & r & r^2 & 1 & sr^2 & s & sr \\
r^2 & r^2 & 1 & r & sr & sr^2 & s \\
s & s & sr & sr^2 & 1 & r & r^2 \\
sr & sr & sr^2 & s & r^2 & 1 & r \\
sr^2 & sr^2 & s & sr & r & r^2 & 1
\end{array} \]

We observed that Cayley table have the Sudoku property, as in the following lemma.

\begin{lem}
Each row (and column) of the Cayley table of a finite group $G$ contains all elements of $G$.
\end{lem}

As a reminder, this lemma follows directly from the cancellation law.

Returning to the above example, if we pick off just one row---say the row $sr$---by this property we get a permutation of the set $D_6$:
\begin{equation}  \label{eqn:sr}
\begin{pmatrix}
1 & r & r^2 & s & sr & sr^2 \\
sr & sr^2 & s & r^2 & 1 & r 
\end{pmatrix} 
\end{equation}
We denote this element $\sigma_{sr} \colon D_6 \to D_6$, since it is a symmetry that depends on $sr$: it is defined by $\sigma_{sr}(sr) = 1$, \dots, $\sigma_{sr}(sr^2)=r$, reading the input from the top row of the table and the output from the bottom row.  This is visibly a bijection from $D_6$ to itself: each element of $D_6$ appears exactly once.  

Recall that we write the set of bijections from a set $A$ to itself as
\[ \Sym(A) \colonequals \{\sigma \colon A \to A \textup{ bijection}\} \]
and this forms a group under composition.  This works for every set $A$, even though we mostly worked with $A=\{1,\dots,n\}$ and then abbreviate $S_n=\Sym(\{1,\dots,n\})$.  There is no loss of generality here.

\begin{lem}~\label{lem:An}
If $A$ is a finite set with $\#A=n$, then the groups $\Sym(A) \simeq S_n$ are isomorphic.
\end{lem}

\begin{proof}
Since $\#A=n$, there is a bijection from $A$ to $\{1,\dots,n\}$.  Each permutation of the elements of $A$ gives a permutation of the elements $\{1,\dots,n\}$ by how they are numbered.  
\end{proof}

Putting these together, we can define a function
\begin{equation}~\label{eqn:D3map}
\begin{aligned}
\sigma \colon D_6 &\to \Sym(D_6) \simeq S_6 \\
1 &\mapsto \sigma_1 = \begin{pmatrix}
1 & r & r^2 & s & sr & sr^2 \\
1 & r & r^2 & s & sr & sr^2
\end{pmatrix} \\
&\vdots \\
sr^2 &\mapsto \sigma_{sr^2} = \begin{pmatrix}
1 & r & r^2 & s & sr & sr^2 \\
sr^2 & s & sr & r & r^2 & 1
\end{pmatrix} 
\end{aligned}
\end{equation}
Usually we write functions like $f(x)$ with input $x$ from the domain; but here the output is itself a function which wants input, so in order not to get confused, we use a subscript.  

That is a start, but of course in group theory we want more than just a map of sets: we want to know it is a homomorphism!  One case of the homomorphism property in this example would read
\begin{equation} 
\sigma_{sr}\sigma_{r} \overset{?}{=} \sigma_{sr^2} 
\end{equation}
This is an equality we need to check on the right-hand side of \eqref{eqn:D3map}.  Composing the permutations, we see it checks out!  Once we have a homomorphism, we can also see that the kernel of the map $\sigma$ consists only of the identity: if an element maps to the identity permutation in $\Sym(D_6)$, it would come from a row of the Cayley table where they elements line up according to the identity, and that happens only for the top row.  So we get an injective map $D_6 \hookrightarrow S_6$.  By the fundamental homomorphism theorem, we see that $D_6$ is isomorphic to its image under this map; the following lemma reminds us of how this works in general.

\begin{thm}[Fundamental homomorphism theorem]
Let $\phi \colon G \to H$ be a group homomorphism.  Then $G/\ker \phi \simeq \phi(G) \leq H$.
\end{thm}

\begin{corollary}~\label{cor:injG}
If $\phi \colon G \to H$ is an injective group homomorphism, then $G \simeq \phi(G)$.
\end{corollary}

\begin{proof}
If $\phi$ is injective, then $\ker \phi =\{1\}$, and then $G/\ker \phi \simeq G$ (the cosets of the identity are just the elements of $G$!).
\end{proof}

We conclude that $D_6$ is isomorphic to a subgroup of $S_6$.  This is Cayley's theorem!  

\newpage
\section{Further Analysis}~\label{sec:analysis}

\begin{Answer}
  \begin{enumalph}
    \item Show that every nonabelian group $G$ of order $8$ is isomorphic to either 
    $D_8$ or $Q_8$, as follows.
    \begin{itemize}
    \item Show that $G$ has an element $a \in G$ of order $4$.  \emph{[Hint: what 
    happens if every nonidentity element has order $2$?]}
    \item Let $H \colonequals \langle a \rangle$ and let $b \not \in H$.  Observe that 
    $H \trianglelefteq G$ is normal; argue that $bab^{-1}=a^3$ (the order under 
    conjugation is preserved), and then that $G \simeq D_4,Q_8$ according as $b$ has 
    order $2$ or $4$.  
    \end{itemize}
    \item Pause to show we have classified groups of order $n \leq 9$.
  \end{enumalph}
\end{Answer}

Now, let us consider the nonabelian groups of order $n = 8$.
Note that $8 = 2^3$.
Let $G$ be a nonabelian group of order $8$.
By Lagrange's theorem (\ref{thm:lagrange}),
non-identity elements of $G$ may have orders $2, 4, 8$.
If $G$ has an element $a$ of order $8$, then $G$ is cyclic, therefore abelian.
On the other hand, if every nonidentity element $b_i \in G$ has order $2$, then:
\begin{enumalph}
  \item For any $b_i \in G$, $\langle b_i \rangle \le G$ is of prime order $p = 2$
    and is therefore abelian.
  \item Taking $b_j \in G, b_j \notin \langle b_i \rangle$,
    then $\langle bj \rangle \le G$ is also an abelian group of order $2$.
    Then $\langle b_i, b_j \rangle \le G$ is an abelian group of order $4$
    (direct product of abelian groups).
  \item Taking $b_k \in G, b_k \notin \langle b_i, b_j \rangle$,
    then $\langle b_k \rangle \le G$ is also an abelian group of order $2$.
    Then $\langle b_i, b_j, b_k \rangle \le G$ is an abelian group of order $8$
    (direct product of abelian groups).
  \item Since $\#G = 8$ and $\#\langle b_i, b_j, b_k \rangle = 8$
    yet $\langle b_i, b_j, b_k \rangle \le G$, it must be the case
    that $\langle b_i, b_j, b_k \rangle = G$.
  \item Therefore, if every non-identity element in $G$ has order $2$, then $G$ is abelian.
    Particularly, $G \isom C_2 \times C_2 \times C_2$.
\end{enumalph}

Therefore, if $G$ is a \emph{nonabelian} group of order $8$, then $G$
must have an element of order $4$.
Let $a \in G$ be such an element, with $H \colonequals \langle a \rangle$.
Then $\#H = 4$, and $H$ is cyclic meaning $a^{-1} = a^3$.
Let $b \in G, b \notin H$. Then $b$ may have order $2$ or order $4$.

\begin{enumerate}
  \item If $b$ has order $2$, then $b^2 = 1$, and
    $bab^{-1} = b(ab^{-1}) = b^2 a^3 = a^3 \in H$.
    Therefore $H \trianglelefteq G$ is normal in $G$ and
    $G$ is isometric to $D_8$.
  \item If $b$ has order $4$, then $G$ is isomorphic to $Q_8$.
    Particularly, $G = \langle x, y, z \rangle$ where $x^2 = y^2 = z^2 = -1$,
    each of $x, y, z$ has order $4$, and $xy = z, yz = x, zx = y$ and
    $yx = -z, zy = -x, xz = -y$.
    It follows that $aba^{-1} = ab(-a) = c(-a) = -ca = -b$ for any
    permutations of $a, b, c \in \{x, y, z \}$.
    Therefore, any subgroup $H = \langle a \rangle = \{1, a, -1, -a \}$
    having $a \in \{ x, y, z \}$ is normal in $G$.
\end{enumerate}

We have now classified all nonabelian groups of order $8$,
and all groups of order $n \leq 9$ (since $9 = 3^2$ is a square of a prime,
it may only have the abelian group $C_3 \times C_3$ and $C_9$).


\section{Main Theorem}~\label{sec:main-theorem}

\begin{Answer}
  \begin{enumerate}
    \item Now let $G$ be a group of order $pq$ where $p,q$ are primes with $p < q$ 
    (without loss of generality).  
    % \begin{itemize}
    % \item Let $P \leq G$ be a $p$-Sylow subgroup and $Q \leq G$ be a $q$-Sylow 
    % subgroup.  Show that $Q \trianglelefteq G$ is normal.  \emph{[Hint: it has index 
    % $p$, so go through DF, section 4.2, Corollary 5, pp.~120--121.]}  Write $P=\langle 
    % x \rangle$ and $Q=\langle y \rangle$ with $x,y \in G$.
    % \item Show that $xyx^{-1}=y^k$ with $k \in \{1,\\cdots,q-1\}$, and use this to define
    % a group homomorphism
    % \[ \phi \colon P \to {(\Z/q\Z)}^\times \]
    % where $x \mapsto k$.  \emph{[Hint: use $x^i y x^{-i} = y^{\phi(i)}$.]}  Conclude 
    % that either $\phi$ is the trivial homomorphism (mapping every element to $1$) or $\
    % phi$ is injective.
    % \item If $\phi$ is trivial, prove that $G$ is cyclic (DF, section 4.4, Exercise 2, 
    % p.~137).
    % \item Show that $\phi$ is injective if and only if $G$ is nonabelian and $p \mid 
    % (q-1)$.  (In particular, observe that if $p \nmid (q-1)$ then $G$ is abelian.)
    % \item If $p \mid (q-1)$, exhibit a nonabelian group of order $pq$ (following DF, 
    % Example, section 4.5, p.~143; see also DF, section 4.3, Exercise 34, p.~132).  Show
    % that when $p=2$ we obtain the dihedral group $D_{2q}$ of order $2q$ for $q \geq 3$ 
    % as a subgroup of $S_q$ via the action on the vertices of a $q$-gon.
    % \item Suppose that $G$ is nonabelian and $p \mid (q-1)$.  Show that $P \not\
    % trianglelefteq G$ (DF, Example, section 4.5, p.~143), so there exists an injective 
    % homomorphism $G \hookrightarrow S_q$ whose image up to conjugation lies in the the 
    % normalizer of the cyclic subgroup generated by the $q$-cycle $(1\ 2\ \\cdots \ q)$ 
    % (DF, section 4.3, Exercise 28, p.~132).  When $p=2$, show that this group is unique
    % up to conjugation.  \emph{[Hint: show that there is a unique subgroup of ${(\Z/q\
    % Z)}^\times$ of order $2$.]}
    % \end{itemize}
    \item[{8${}^\prime$.}] For +3 bonus going a more conceptual route, replace step 
    (8) as follows.
    \begin{itemize}
    \item Read DF, section 4.4.  
    \item Prove that the automorphism group of $Z_p$ is $\Aut(Z_p) \simeq
      {(\Z/p\Z)}^\times$ (DF, section 4.4, Proposition 16, p.~135 proves something more general).  
    State (but you do not need to prove) that ${(\Z/p\Z)}^\times \simeq Z_{p-1}$ is 
    cyclic.  
    \item Suppose that $p \nmid (q-1)$.  Show that $G$ is abelian (DF, Example, section
    4.4, p.~135--136)
    and therefore cyclic (DF, section 4.4, Exercise 2, p.~137).  
    \item Now suppose that $p \mid (q-1)$.  Let $P \leq G$ be a $p$-Sylow subgroup and 
    $Q \leq G$ be a $q$-Sylow subgroup.  Show that $Q \trianglelefteq G$ is normal in 
    $G$ (DF, Example, section 4.5, p.~143).
    \item Read section 5.5.  Show that if $p \mid (q-1)$ then either $G \simeq Z_p
    \times Z_q \simeq Z_{pq}$ or $G \simeq Z_p \rtimes Z_q$, the semi-direct product 
    with respect to the homomorphism $Z_p \to \Aut(Z_q) \simeq {(\Z/q\Z)}^\times =
    \langle g \rangle$ mapping $x \mapsto g^{(q-1)/p}$ (DF, Example, section 5.5, 
    pp.~181--182; section 5.5, Exercise 6, pp.~184--185).
    \end{itemize}
  \end{enumerate}
\end{Answer}

Moving on to groups of order $10$; we may notice that $10 = 5 \cdot 2$ = $pq$
with $p=2, q=5$ prime.

Consider the automorphism group of $C_p$, $\Aut(C_p)$,
defined to be the group of all homomorphisms from $C_p$ onto itself.
Let $\psi : C_p \to C_p \in \Aut(C_p)$ be an automorphism of $C_p$.
Then $\psi(x) = x^a$ for some $a \in \Z/p\Z$.
Precisely, the value of $a$ uniquely determines the automorphism $\psi$,
which we denote as $\psi_a$.~\cite[see DF Section 4.4, Proposition $16$]{DummitFoote}.

Now, consider that $C_p$ is a cyclic group of order $p$.
Taking $x$ as the minimal generator,
then  \\ $C_p = \{0, x, x^2, x^3, \ldots, x^{p-2}, x^{p-1}\}$.
Additionally,
\begin{align}
  x^{ip} = {(x^p)}^i = 0^i = 0\, \, \forall i \in \Z~\label{eq:nilpotency}
\end{align}
Now, consider any arbitrary element $x^\alpha \in C_p$
such that $\gcd{\alpha}{p} = g > 1$. Then, we say $\alpha$ is not coprime to $p$.
Define the \emph{least common multiple} of
$p$ and $\alpha$ to be
\[
  \lcm{p}{\alpha} = \frac{\alpha p}{g}
\]
Since $g \mid \alpha$, then $\lcm{p}{\alpha} = np$ for some $n \in \Z$.
Therefore, $x^{\lcm{p}{\alpha}} = x^{np} = 0$ (by equation~\ref{eq:nilpotency}).
This means that, whenever $(a, p) \ne 1$, then $x^{\lcm{p}{a}} = 0$,
therefore $\psi_a$ is not an automorphism (since it's kernel is not trivial,
it's image does not equal $C_p$).

Consequently, $\psi_a \in \Aut(C_p) \Leftrightarrow (a, p) = 1$.
We may also recognize that such elements $\psi_a \in \Aut(C_p)$
correspond to the units $a \in {(\Z/p\Z)}^\times$,
and $\abs{\Aut(C_p)} = \abs{{(\Z/p\Z)}^\times} = \phi(p)$.
Therefore, $\Aut(C_p) \simeq {(\Z/p\Z)}^\times$.
When $p$ is prime, then $\phi(p) = p-1$,
in which case
\begin{align}
  \Aut(C_p) \simeq (\Z/p\Z)^\times \simeq C_{p-1}~\label{eqn:aut_p}
\end{align}
and $\Aut(C_p)$ is cyclic.

Consider the case that $\#G = pq$ having $p \nmid (q-1)$.
Then, $G$ is abelian (by DF Example 4.4.1, p.~135--136).
Consider the center $Z(G) \le G$. If $Z(G) \ne 1$, then Lagrange's theorem (\ref{thm:lagrange})
forces $G/Z(G)$ to be cyclic.
Furthermore, Lagrange's theorem tells us that the order of any element in $G$ must divide the
order of $G$, therefore the order may be $n \in \{1, p, q, pq \}$.
Suppose $Z(G) = 1$. Then;
\begin{enumalph}
  \item Nonidentity elements may not have order $1$.
  \item If every nonidentity element of $G$ has order $p$,
    then the centralizer of every nonidentity
    element has index $q$, so the class equation reads \[pq = 1 + kq \]
    This is contradictory, since $q \mid pq$ but $q \nmid (1 + kq)$
    (because $q$ does not divide $1$).
    Therefore, if $G$ contains an element $x$ of order $q$.
  \item Let $H = \langle x \rangle$ be the subgroup generated by $x$.
    Since $H$ has index $p$ in $G$ and $p$ is the smallest prime dividing $\#G = pq$,
    $H$ is a normal subgroup in $G$ by Corollary 5~\cite[p. 120]{DummitFoote}.
  \item Since $Z(G) = 1$, then $C_G(H) = 1 \cdot H \cdot 1 = H$. Therefore, the
    quotient group $G/H = N_G(H)/C_G(H)$ has order $p$ and is isomorphic to a group of
    $\Aut(H)$ by Corollary 15 \cite[p. 134]{DummitFoote}.
  \item By Proposition 16 \cite[p. 135]{DummitFoote}, $\Aut(H) \simeq C_{q-1}$.
    and has order $q-1$, which implies that $p \mid (q-1)$ by Lagrange's theorem (\ref{thm:lagrange}),
    a contradiction of the assumption that $p \nmid (q-1)$. Therefore, $G$ must be abelian
    .~\label{eqn:p_not_div_q-1}

\end{enumalph}

Now, suppose $p \mid (q-1)$.

\begin{theorem}[Sylow's Theorem]~\label{thm:sylow}
  Let $G$ be a finite group of order $p^{\alpha}m$ where $p$ is a prime not dividing $m$.
  \begin{enumerate}[label=\arabic{enumi}.]
    \item Sylow $p$-subgroups of $G$ exist, i.e. $n_p \ne \varnothing$.
    \item If $P$ is a Sylow $p$-subgroup of $G$ and $Q$ is any $p$-subgroup of $G$,
      then there exists $g \in G$ such that $Q = gPg^{-1}$, i.e. $Q$ is conjugate to $P$.
    \item The number of Sylow $p$-subgroups of $G$ is of the form
      $n_p = 1 + kp$, i.e. \[ n_p \equiv 1 \pmod p. \]
      Further, $n-p$ is the index of $N_G(P)$ in $G$ for any Sylow $p$-subgroup $P$,
      hence \[ n_p \mid m. \]~see~\cite[Theorem 18,~p.~139]{DummitFoote}.\qed
  \end{enumerate}
  
\end{theorem}

Let $P \le G$ be a Sylow $p$-subgroup of $G$
and $Q \le G$ be a Sylow $q$-subgroup of $G$.
Sylow's theorem (see \ref{thm:sylow}) tells us that
$n_q = 1 + kq$ for some $k \ge 0$ and $n_q \mid p$.
Since we have $p < q$, it must be that $k = 0$ and $n_q = 1$.
By Corollary 20~\cite[p. 142]{DummitFoote}, $Q$ is normal in $G$.


Since $P$ and $Q$ are of prime order, they are cyclic and are each generated by a single element.
Let $P = \langle p \rangle$ and $Q = \langle q \rangle$.
Note that $\Aut(Q) \simeq C_{q-1}$ is cyclic and $p \mid (q-1)$,
$Q$ contains a unique \emph{cyclic} subgroup of order $p$, say $\langle \gamma \rangle$,
and any homomorphism $\psi : P \to \Aut(Q)$ must map $y$ to a power of $\gamma$.
Since $\abs{\gamma} = p$, there are, therefore, $p$ distinct homomorphisms
$\psi_i : P \to \Aut(Q)$ given by $\psi(y) = \gamma^i, 0 \le i \le p-1$.
There are two general cases:
\begin{enumalph}
  \item If $i = 0$, then $\psi_i$ is the trivial homomorphism
    and $Q \rtimes_{\psi_0} P \simeq Q \times P$. This is an abelian
    group isomorphic to to $C_q \times C_p$.
  \item If $i \ne 0$, then $\psi_i$ is nontrivial and $Q \rtimes_{\psi_i} P$
    is a nonabelian group of order $p^q$. We may also note that all these groups
    are isomorphic because for each $\psi_i, i \ne 0$, there is some generator
    element $y_i \in P$ such that $\psi_i(y_i) = \gamma$.~\cite[p. 181]{DummitFoote}
    .~\label{pro:nonabelian}
\end{enumalph}

\begin{Answer}
  \begin{enumerate}
  \item Classify groups of order $n \leq 15$ except $n=12$.
  \end{enumerate}
\end{Answer}

\section{Implications}~\label{sec:impl}

\begin{Answer}
  \begin{enumerate}
    \item Give a table of groups of order $n\leq 15$ up to isomorphism (DF, section 
    5.3, p.~168), giving the answer even if you don't do (10).
    \item As an application, identify in the table the three groups $G_1=Z_2 \times 
    D_6$, 
    \[ G_2 \colonequals
      \left\{ \begin{pmatrix}
        1 & a & b \\
        0 & 1 & c \\
        0 & 0 & 1
      \end{pmatrix} : a,b,c \in \Z/2\Z \right\} \leq \GL_3(\Z/2\Z) \]
    and $G_3=S/Z(S)$ where  
    \[ S=\SL_2(\Z/3\Z)=\left\{ \begin{pmatrix} a & b \\ c & d \end{pmatrix} : a,b,c,d
    \in \Z/3\Z \text{ and } ad-bc=1 \right\} \leq \GL_2(\Z/3\Z). \]
  \end{enumerate}
\end{Answer}

\section{Conclusion}~\label{sec:conc}

\begin{Answer}

\end{Answer}



\section{Main result}~\label{sec:thm}

We are now ready to prove Cayley's theorem, which we will prove in a slightly stronger form.

\begin{thm}[Cayley]~\label{thm:Cayleyn}
Let $G$ be a finite group of order $\#G = n$.  Then $G$ is isomorphic to a subgroup of $S_n$.
\end{thm}

We follow what we observed in the case $G=D_6$ in the previous section one step at a time.

Throughout, let $G$ be a finite group.  First, we define the permutations that come from the 
rows of the Cayley table.  Recall that in the row labelled $a$ (for $a \in G$), with $b \in G$ 
the column we have entry $ab$ (as usual suppressing $*$ and writing the group 
multiplicatively).

\begin{lem}
Let $a \in G$.  Define the map
\begin{align*}
\sigma_a \colon G &\to G \\
x &\mapsto ax
\end{align*}
Then $\sigma_a$ is a bijection, i.e., $\sigma_a \in \Sym(G)$.   
\end{lem}

\begin{proof}
We proved this in class, showing it is both injective and surjective and then that it has inverse $\sigma_{a^{-1}} \colon G \to G$ defined by $b \mapsto a^{-1}b$.
\end{proof}

We then built on this considering all of these bijections at once.

\begin{prop}~\label{prop:sigmaG}
Define the map
\begin{align*}
\sigma \colon G &\to \Sym(G) \\
a &\mapsto \sigma_a 
\end{align*}
Then $\sigma$ is an injective group homomorphism.  
\end{prop}

\begin{proof}
We first show that the map is a homomorphism.  Let $a,b \in G$.  We want to check
\[ \sigma_a \sigma_b \overset{?}{=} \sigma_{ab}. \]
These are two permutations of the set $G$.  To show that two functions are equal, we show that 
they give the same outputs.  So let $x \in G$.  Then on the left-hand side, by definition
\[ (\sigma_a \sigma_b)(x) = \sigma_a(\sigma_b(x)) = \sigma_a(bx) = a(bx) = abx. \]
This matches the right-hand side:
\[ \sigma_{ab}(x)=(ab)x = abx. \]
Thus $(\sigma_a\sigma_b)(x)=\sigma_{ab}(x)$ for all $x \in G$, so then $\sigma_a \sigma_b = 
\sigma_{ab} \in \Sym(G)$ as functions.  

To show that $\sigma$ is injective, we show that $\ker \sigma \subseteq \{1\}$.  Let $a \in G$ 
be such that $a$ maps to the identity: $\sigma_a=\id_G$.  Then for all $x \in G$ we have 
$\sigma_a(x)=\id_G(x)$, which means $ax=x$ for all $x \in G$.  If we plug in $x=1$ we get 
$a=1$, as desired.
\end{proof}

We may now conclude.

\begin{proof}[Proof of Theorem \ref{thm:Cayleyn}]
By Proposition \ref{prop:sigmaG}, we have an injective group homomorphism
$G \hookrightarrow \Sym(G)$. 
By Lemma \ref{lem:An}, we have an isomorphism $\Sym(G) \simeq S_n$, so composing 
these we get an injective group homomorphism $G \hookrightarrow S_n$.  Finally, by the 
fundamental homomorphism theorem (Corollary 
\ref{cor:injG}), we conclude that $G$ is isomorphic to its image, a subgroup of $S_n$.  
\end{proof}


\printbibliography

\end{document}

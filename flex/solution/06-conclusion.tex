\section{Conclusion}~\label{sec:conc}

\begin{Answer}
  Cayley's theorem shows that we can see every finite group as a subgroup of some permutation 
  group.  There can be more than one way to realize a finite group $G$ as a subgroup of 
  permutations: already for $D_6$ we showed that $D_6 \simeq S_3$ by considering the action of 
  the dihedral group on the vertices of the triangle.  
  
  Returning to our proof, we see that the arguments also work when $G$ is an infinite group: we 
  still get an injective group homomorphism $G \hookrightarrow \Sym(G)$; however, now $\Sym(G)$ 
  consists of permutations of an infinite set, so it is not of the form $S_n$!  
  
  The homomorphism constructed in Proposition \ref{prop:sigmaG} arises naturally in the context
  of groups acting on themselves (by left multiplication): this is described in detail by 
  Dummit--Foote \cite[\S 4.2]{DummitFoote}, with Cayley's theorem as a corollary \cite[\S 4.2, 
  Corollary 4, p.~120]{DummitFoote}.  Indeed, \emph{group actions} allow us to see all 
  homomorphisms from a finite group into permutation groups, whether that be on vertices of an 
  $n$-gon, or on the cosets of a group.
\end{Answer}

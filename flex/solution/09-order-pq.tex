\section{Groups of Product-of-Prime Orders}~\label{sec:main-theorem}

\begin{Answer}
  \begin{enumerate}
    \item Now let $G$ be a group of order $pq$ where $p,q$ are primes with $p < q$ 
    (without loss of generality).
    \item[{8${}^\prime$.}] For +3 bonus going a more conceptual route, replace step 
    (8) as follows.
    \begin{itemize}
    \item Read DF, section 4.4.  
    \item Prove that the automorphism group of $Z_p$ is $\Aut(Z_p) \simeq
      {(\Z/p\Z)}^\times$ (DF, section 4.4, Proposition 16, p.~135 proves something more general).  
    State (but you do not need to prove) that ${(\Z/p\Z)}^\times \simeq Z_{p-1}$ is 
    cyclic.  
    \item Suppose that $p \nmid (q-1)$.  Show that $G$ is abelian (DF, Example, section
    4.4, p.~135--136)
    and therefore cyclic (DF, section 4.4, Exercise 2, p.~137).  
    \item Now suppose that $p \mid (q-1)$.  Let $P \leq G$ be a $p$-Sylow subgroup and 
    $Q \leq G$ be a $q$-Sylow subgroup.  Show that $Q \trianglelefteq G$ is normal in 
    $G$ (DF, Example, section 4.5, p.~143).
    \item Read section 5.5.  Show that if $p \mid (q-1)$ then either $G \simeq Z_p
    \times Z_q \simeq Z_{pq}$ or $G \simeq Z_p \rtimes Z_q$, the semi-direct product 
    with respect to the homomorphism $Z_p \to \Aut(Z_q) \simeq {(\Z/q\Z)}^\times =
    \langle g \rangle$ mapping $x \mapsto g^{(q-1)/p}$ (DF, Example, section 5.5, 
    pp.~181--182; section 5.5, Exercise 6, pp.~184--185).
    \end{itemize}
  \end{enumerate}
\end{Answer}

Consider the automorphism group of $C_n$, $\Aut(C_n)$,
defined to be the group of all homomorphisms from $C_n$ onto itself.
Let $\psi : C_n \to C_n \in \Aut(C_n)$ be an automorphism of $C_n$.
$C_n$ is cyclic, so $\psi(x) = x^a$ for some $a \in \Z/n\Z$.
The value of $a$ uniquely determines the automorphism $\psi$,
which we denote as $\psi_a$.~\cite[see DF Section 4.4, Proposition $16$]{DummitFoote}.


\begin{proposition}
  $\psi_a \in \Aut(C_n)$ if and only if $\gcd{a}{n} = 1$

  \begin{proof}
    Take an arbitrary element $x^\alpha \in C_n$
    such that $g = \gcd{\alpha}{p} > 1$, then $\alpha$ is not coprime to $n$.
    Taking the least common multiple of
    Then $\lcm{p}{\alpha} = \frac{\alpha n}{g}$.
    Since $g \mid \alpha$, then $\lcm{p}{\alpha} = nm$ for some $m \in \Z$.
    Therefore, $x^{\lcm{n}{\alpha}} = x^{nm} = 0$ (since $x^n = 0$ in $C_n$)
    so the map $\psi_a$ cannot be automorphism of $C_n$
    (since its kernel is nontrivial, its image cannot equal $C_n$).
    Therefore, whenever $(a, n) > 1$ $\psi_a \notin \Aut(C_n)$. Consequently;
    \[ \psi_a \in \Aut(C_n) \implies (a, n) = 1 \]
    
  \end{proof}
\end{proposition}

\begin{proposition}~\label{prop:aut-cyclic}
  $Aut(C_n)$ is isomorphic to $(\Z/n\Z)^\times$.

  \begin{proof}
    For the cyclic group $\Z/n\Z \simeq C_n$, we define the group
    ${(\Z/n\Z)^\times}$ as the multiplicative  group of all the units of $\Z/n\Z$,
    which are all coprime to $n$.
    Since $\Aut(C_n)$ is the group of all automorphisms $\psi_a$ of $C_n$
    all having $a$ coprime to $n$, then $\Aut(C_n)$ has the same number of elements
    and ${(\Z/,\Z)}^\times$. Define the map
    \begin{align*}
      f \colon \Aut(C_n) &\to {(\Z/n\Z)}^\times \\
      f(1) &\mapsto 1 \\
      f(\psi_a) &\mapsto a
    \end{align*}
  \end{proof}
\end{proposition}

\begin{proposition}
  If $n = p$ is prime, then $\#Aut(C_p) = p - 1$.

  \begin{proof}
    By Proposition~\ref{prop:aut-cyclic}, $\Aut(C_p)$ is isomorphic to $(\Z/n\Z)^\times$.
    Recall that $(\Z/p\Z)^\times = \{x \colon 1 \le x < p, x \nmid p \}$.
    Since $p$ is prime, then $\#{(\Z/p\Z)}^\times = p - 1$,
    so $\#\Aut(C_p) = p - 1$.
  \end{proof}
\end{proposition}
\begin{proposition}
  If $G$ is a group of order $pq$ with $p, q$ prime and $p \nmid (q - 1)$,
  then $G$ is abelian.

  \begin{proof}
    Let $\#G = pq$ having $p \nmid (q-1)$.
    Consider the center $Z(G) \le G$. Suppose $Z(G) \ne 1$.
    Lagrange's theorem (see~\ref{thm:lagrange}) forces $Z(G)$ to be cyclic.
    Let $g$ be a generator of $Z(G)$,
    then Lagrange's theorem further limits the order of $g$ to be a divisor of $pq$.
    Therefore, the order of $g$ may be $1, p, q$, or $pq$. However;
    \begin{enumalph}
      \item Nonidentity elements may not have order $1$.
      \item If every nonidentity element of $G$ has order $p$,
        then the centralizer of every nonidentity
        element has index $q$, so the class equation reads \[pq = 1 + kq \]
        This is contradictory, since $q \mid pq$ but $q \nmid (1 + kq)$
        (because $q$ does not divide $1$).
        Therefore, all nonidentity elements of $G$ cannot have order $p$,
        implying that $G$ must have an element of order $q$.~\label{cor:order-p-then-q}        
      \item If $G$ contains an element $x$ of order $q$, then
        let $H = \langle x \rangle$ be the subgroup generated by $x$.
        Since $H$ has index $p$ in $G$ and $p$ is the smallest prime dividing $\#G = pq$,
        $H$ is a normal subgroup in $G$ by Corollary 5~\cite[p. 120]{DummitFoote}.
        Since $Z(G) = 1$, then $C_G(H) = 1 \cdot H \cdot 1 = H$. Therefore, the
        quotient group $G/H = N_G(H)/C_G(H)$ has order $p$ and is isomorphic to a group of
        $\Aut(H)$ by Corollary 15 \cite[p. 134]{DummitFoote}.
        By Proposition 16 \cite[p. 135]{DummitFoote}, $\Aut(H) \simeq C_{q-1}$.
        and has order $q-1$, which implies that $p \mid (q-1)$ by Lagrange's theorem (\ref{thm:lagrange}),
        a contradiction of the assumption that $p \nmid (q-1)$. Therefore, $G$ must be abelian.
    \end{enumalph}
  \end{proof}
\end{proposition}

Now, suppose $p \mid (q-1)$.

\begin{theorem}[Sylow's Theorem]~\label{thm:sylow}
  Let $G$ be a finite group of order $p^{\alpha}m$ where $p$ is a prime not dividing $m$.
  \begin{enumerate}[label=\arabic{enumi}.]
    \item Sylow $p$-subgroups of $G$ exist, i.e. $n_p \ne \varnothing$.
    \item If $P$ is a Sylow $p$-subgroup of $G$ and $Q$ is any $p$-subgroup of $G$,
      then there exists $g \in G$ such that $Q = gPg^{-1}$, i.e. $Q$ is conjugate to $P$.
    \item The number of Sylow $p$-subgroups of $G$ is of the form
      $n_p = 1 + kp$, i.e. \[ n_p \equiv 1 \pmod p. \]
      Further, $n-p$ is the index of $N_G(P)$ in $G$ for any Sylow $p$-subgroup $P$,
      hence \[ n_p \mid m. \]~see~\cite[Theorem 18,~p.~139]{DummitFoote}.\qed
  \end{enumerate}
  
\end{theorem}

Let $P \le G$ be a Sylow $p$-subgroup of $G$
and $Q \le G$ be a Sylow $q$-subgroup of $G$.
Sylow's theorem (see \ref{thm:sylow}) tells us that
$n_q = 1 + kq$ for some $k \ge 0$ and $n_q \mid p$.
Since we have $p < q$, it must be that $k = 0$ and $n_q = 1$.
By Corollary 20~\cite[p. 142]{DummitFoote}, $Q$ is normal in $G$.


Since $P$ and $Q$ are of prime order, they are cyclic
and are each generated by a single element (by~\ref{cor:prime-then-abelian}).
Let $P = \langle p \rangle$ and $Q = \langle q \rangle$.
Note that $\Aut(Q) \simeq C_{q-1}$ is cyclic and $p \mid (q-1)$,
so $Q$ contains a unique \emph{cyclic} subgroup of order $p$,
say $\langle \gamma \rangle$,
and any homomorphism $\psi : P \to \Aut(Q)$ must map $y \in P$ to a power of $\gamma$.
Since $\abs{\gamma} = p$, there are, therefore, $p$ distinct homomorphisms
$\psi_i : P \to \Aut(Q)$ given by $\psi(y) = \gamma^i, 0 \le i \le p-1$.
There are two general cases:
\begin{enumalph}
  \item If $i = 0$, then $\psi_i$ is the trivial homomorphism
  \item having $\psi_i(g) = 1$ for all $g \in G$. In this case,
    $Q \rtimes_{\psi_0} P \simeq Q \times P \simeq C_q \times C_p$.
  \item If $i \ne 0$, then $\psi_i$ is nontrivial and $Q \rtimes_{\psi_i} P$
    is a nonabelian group of order $p^q$. We may also note that all these groups
    are isomorphic because for each $\psi_i, i \ne 0$, there is some generator
    element $y_i \in P$ such that $\psi_i(y_i) = \gamma$.~\cite[p. 181]{DummitFoote}
    ~\label{case:nontrivial_psi}
\end{enumalph}

We may now classify all groups of order $n \le 15$, except those of order $12$.

For $n = 10$, note that $10 = 2 \cdot 5$, which is a product of primes.
Also, $2 \mid (5 - 1)$.
Let $G$ be a group of order $10$.
There are two possibilities:
\begin{enumalph}
  \item If $G$ contains an element of order $10$, then $G$ is cyclic and
    isomorphic to $C_{10}$ (see~\ref{prop:n-has-n-cyclic}).
  \item If $G$ does not contain an element of order $10$,
    then $G$ must have an element of order $2$ and an element of order $5$
    (see~case~\ref{cor:order-p-then-q}).
    If $p \in G, \abs{p} = 2$ and $q \in G, \abs{q} = 5$,
    let $P = \langle p \rangle \simeq C_2$ and $Q = \langle q \rangle \simeq C_5$.
    Then  \[ \Aut(Q) \simeq {(Z/5Z)}^\times \]
    Consider the homomorphism $\psi \colon C_2 \to {(\Z/5\Z)}^\times$.
    Since ${(\Z/5\Z)}^\times$ has the single element $4$:
    \begin{align*}
      \psi_4 : P \simeq C_2 &\to {(\Z/5\Z)}^\times \simeq \Aut(Q) \\
              1 \sim 0 &\mapsto 0 \sim 1 \\
              p \sim 1 &\mapsto 4 \sim q^4 \\
    \end{align*}
    Therefore, the map is the instance $\psi_2$ of the general case $psi_i$,
    and the homomorphism is nontrivial.
    This means $G \simeq C_5 \rtimes_{\psi_2} C_2$, making $G$ nonabelian
    (see case \ref{case:nontrivial_psi}).
    This group is more commonly seen as the dihedral group $D_{10}$,
    generated by $r$ as an element of order $5$ with
    and $s$ as an element of order $2$.
\end{enumalph}

For $n = 11$ and $n = 13$, notice that $n$ is prime, therefore the only group of order $11$
up to isomorphism is $C_{11}$ and $C_13$ respectively.

For $n = 14$, note that $14 = 2 \cdot 7$, a product of primes having $2 \mid (7 - 1)$.
Suppose $G$ is a group of order $14$, then;
\begin{enumerate}
  \item If $G$ contains an element of order $14$, then $G$ is isomorphic to $C_{14}$.
  \item Otherwise, $G$ must contain an element of order $7$ and an element of order $2$.
    If $p \in G, \abs{p} = 2$ and $q \in G, \abs{q} = 7$,
    then let's define $P = \langle p \rangle \simeq C_2$ and $Q = \langle q \rangle \simeq C_7$.
    Then $\Aut(Q) \simeq {(\Z/6\Z)}^\times$.
    Since $\Z/6\Z$ contains only a single element of order $2$,
    that is $6$ with $6^2 = 36 \equiv 1 \pmod 7$,
    the homomorphism between the two groups is defined as:
    \begin{align*}
      \psi_6 &: C_2 \to {(\Z/7\Z)}^\times \\
           & 0 \mapsto 1 \\
           & 1 \mapsto 6 \\
    \end{align*}
    More generally, $\psi_6 (y) = \gamma^6$ and the homomorphism is nontrivial.
    This means $G \simeq C_7 \rtimes_{\psi_3} C_2$, making $G$ nonabelian
    (see case \ref{case:nontrivial_psi}).
    This group is more commonly seen as the dihedral group $D_{14}$
    with $r$ as an element of order $7$ with
    and $s$ as an element of order $2$.
\end{enumerate}

For $n = 15$, we once again have a product of primes, $15 = 3 \cdot 5$.
However, $3 \nmid (5 - 1)$, so every group of order $15$
is abelian and isomorphic to $C_{15}$.

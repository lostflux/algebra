\section{Main Theorem}~\label{sec:main-theorem}

\begin{Answer}
  \begin{enumerate}
    \item Now let $G$ be a group of order $pq$ where $p,q$ are primes with $p < q$ 
    (without loss of generality).  
    \begin{itemize}
    \item Let $P \leq G$ be a $p$-Sylow subgroup and $Q \leq G$ be a $q$-Sylow 
    subgroup.  Show that $Q \trianglelefteq G$ is normal.  \emph{[Hint: it has index 
    $p$, so go through DF, section 4.2, Corollary 5, pp.~120--121.]}  Write $P=\langle 
    x \rangle$ and $Q=\langle y \rangle$ with $x,y \in G$.
    \item Show that $xyx^{-1}=y^k$ with $k \in \{1,\\cdots,q-1\}$, and use this to define
    a group homomorphism
    \[ \phi \colon P \to {(\Z/q\Z)}^\times \]
    where $x \mapsto k$.  \emph{[Hint: use $x^i y x^{-i} = y^{\phi(i)}$.]}  Conclude 
    that either $\phi$ is the trivial homomorphism (mapping every element to $1$) or $\
    phi$ is injective.
    \item If $\phi$ is trivial, prove that $G$ is cyclic (DF, section 4.4, Exercise 2, 
    p.~137).
    \item Show that $\phi$ is injective if and only if $G$ is nonabelian and $p \mid 
    (q-1)$.  (In particular, observe that if $p \nmid (q-1)$ then $G$ is abelian.)
    \item If $p \mid (q-1)$, exhibit a nonabelian group of order $pq$ (following DF, 
    Example, section 4.5, p.~143; see also DF, section 4.3, Exercise 34, p.~132).  Show
    that when $p=2$ we obtain the dihedral group $D_{2q}$ of order $2q$ for $q \geq 3$ 
    as a subgroup of $S_q$ via the action on the vertices of a $q$-gon.
    \item Suppose that $G$ is nonabelian and $p \mid (q-1)$.  Show that $P \not\
    trianglelefteq G$ (DF, Example, section 4.5, p.~143), so there exists an injective 
    homomorphism $G \hookrightarrow S_q$ whose image up to conjugation lies in the the 
    normalizer of the cyclic subgroup generated by the $q$-cycle $(1\ 2\ \\cdots \ q)$ 
    (DF, section 4.3, Exercise 28, p.~132).  When $p=2$, show that this group is unique
    up to conjugation.  \emph{[Hint: show that there is a unique subgroup of ${(\Z/q\
    Z)}^\times$ of order $2$.]}
    \end{itemize}
    \item[{8${}^\prime$.}] For +3 bonus going a more conceptual route, replace step 
    (8) as follows.
    \begin{itemize}
    \item Read DF, section 4.4.  
    \item Prove that the automorphism group of $Z_p$ is $\Aut(Z_p) \simeq
      {(\Z/p\Z)}^\times$ (DF, section 4.4, Proposition 16, p.~135 proves something more general).  
    State (but you do not need to prove) that ${(\Z/p\Z)}^\times \simeq Z_{p-1}$ is 
    cyclic.  
    \item Suppose that $p \nmid (q-1)$.  Show that $G$ is abelian (DF, Example, section
    4.4, p.~135--136)
    and therefore cyclic (DF, section 4.4, Exercise 2, p.~137).  
    \item Now suppose that $p \mid (q-1)$.  Let $P \leq G$ be a $p$-Sylow subgroup and 
    $Q \leq G$ be a $q$-Sylow subgroup.  Show that $Q \trianglelefteq G$ is normal in 
    $G$ (DF, Example, section 4.5, p.~143).
    \item Read section 5.5.  Show that if $p \mid (q-1)$ then either $G \simeq Z_p \
    times Z_q \simeq Z_{pq}$ or $G \simeq Z_p \rtimes Z_q$, the semi-direct product 
    with respect to the homomorphism $Z_p \to \Aut(Z_q) \simeq {(\Z/q\Z)}^\times = \
    langle g \rangle$ mapping $x \mapsto g^{(q-1)/p}$ (DF, Example, section 5.5, 
    pp.~181--182; section 5.5, Exercise 6, pp.~184--185).
    \end{itemize}
  \end{enumerate}
\end{Answer}

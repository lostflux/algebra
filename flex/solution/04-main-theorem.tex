\section{Main Theorem}~\label{sec:main-theorem}

\begin{Answer}
  \begin{enumerate}
    \item Now let $G$ be a group of order $pq$ where $p,q$ are primes with $p < q$ 
    (without loss of generality).  
    % \begin{itemize}
    % \item Let $P \leq G$ be a $p$-Sylow subgroup and $Q \leq G$ be a $q$-Sylow 
    % subgroup.  Show that $Q \trianglelefteq G$ is normal.  \emph{[Hint: it has index 
    % $p$, so go through DF, section 4.2, Corollary 5, pp.~120--121.]}  Write $P=\langle 
    % x \rangle$ and $Q=\langle y \rangle$ with $x,y \in G$.
    % \item Show that $xyx^{-1}=y^k$ with $k \in \{1,\\cdots,q-1\}$, and use this to define
    % a group homomorphism
    % \[ \phi \colon P \to {(\Z/q\Z)}^\times \]
    % where $x \mapsto k$.  \emph{[Hint: use $x^i y x^{-i} = y^{\phi(i)}$.]}  Conclude 
    % that either $\phi$ is the trivial homomorphism (mapping every element to $1$) or $\
    % phi$ is injective.
    % \item If $\phi$ is trivial, prove that $G$ is cyclic (DF, section 4.4, Exercise 2, 
    % p.~137).
    % \item Show that $\phi$ is injective if and only if $G$ is nonabelian and $p \mid 
    % (q-1)$.  (In particular, observe that if $p \nmid (q-1)$ then $G$ is abelian.)
    % \item If $p \mid (q-1)$, exhibit a nonabelian group of order $pq$ (following DF, 
    % Example, section 4.5, p.~143; see also DF, section 4.3, Exercise 34, p.~132).  Show
    % that when $p=2$ we obtain the dihedral group $D_{2q}$ of order $2q$ for $q \geq 3$ 
    % as a subgroup of $S_q$ via the action on the vertices of a $q$-gon.
    % \item Suppose that $G$ is nonabelian and $p \mid (q-1)$.  Show that $P \not\
    % trianglelefteq G$ (DF, Example, section 4.5, p.~143), so there exists an injective 
    % homomorphism $G \hookrightarrow S_q$ whose image up to conjugation lies in the the 
    % normalizer of the cyclic subgroup generated by the $q$-cycle $(1\ 2\ \\cdots \ q)$ 
    % (DF, section 4.3, Exercise 28, p.~132).  When $p=2$, show that this group is unique
    % up to conjugation.  \emph{[Hint: show that there is a unique subgroup of ${(\Z/q\
    % Z)}^\times$ of order $2$.]}
    % \end{itemize}
    \item[{8${}^\prime$.}] For +3 bonus going a more conceptual route, replace step 
    (8) as follows.
    \begin{itemize}
    \item Read DF, section 4.4.  
    \item Prove that the automorphism group of $Z_p$ is $\Aut(Z_p) \simeq
      {(\Z/p\Z)}^\times$ (DF, section 4.4, Proposition 16, p.~135 proves something more general).  
    State (but you do not need to prove) that ${(\Z/p\Z)}^\times \simeq Z_{p-1}$ is 
    cyclic.  
    \item Suppose that $p \nmid (q-1)$.  Show that $G$ is abelian (DF, Example, section
    4.4, p.~135--136)
    and therefore cyclic (DF, section 4.4, Exercise 2, p.~137).  
    \item Now suppose that $p \mid (q-1)$.  Let $P \leq G$ be a $p$-Sylow subgroup and 
    $Q \leq G$ be a $q$-Sylow subgroup.  Show that $Q \trianglelefteq G$ is normal in 
    $G$ (DF, Example, section 4.5, p.~143).
    \item Read section 5.5.  Show that if $p \mid (q-1)$ then either $G \simeq Z_p
    \times Z_q \simeq Z_{pq}$ or $G \simeq Z_p \rtimes Z_q$, the semi-direct product 
    with respect to the homomorphism $Z_p \to \Aut(Z_q) \simeq {(\Z/q\Z)}^\times =
    \langle g \rangle$ mapping $x \mapsto g^{(q-1)/p}$ (DF, Example, section 5.5, 
    pp.~181--182; section 5.5, Exercise 6, pp.~184--185).
    \end{itemize}
  \end{enumerate}
\end{Answer}

Moving on to groups of order $10$; we may notice that $10 = 5 \cdot 2$ = $pq$
with $p=2, q=5$ prime.

Consider the automorphism group of $C_p$, $\Aut(C_p)$,
defined to be the group of all homomorphisms from $C_p$ onto itself.
Let $\psi : C_p \to C_p \in \Aut(C_p)$ be an automorphism of $C_p$.
Then $\psi(x) = x^a$ for some $a \in \Z/p\Z$.
Precisely, the value of $a$ uniquely determines the automorphism $\psi$,
which we denote as $\psi_a$.~\cite[see DF Section 4.4, Proposition $16$]{DummitFoote}.

Now, consider that $C_p$ is a cyclic group of order $p$.
Taking $x$ as the minimal generator,
then  \\ $C_p = \{0, x, x^2, x^3, \ldots, x^{p-2}, x^{p-1}\}$.
Additionally,
\begin{align}
  x^{ip} = {(x^p)}^i = 0^i = 0\, \, \forall i \in \Z~\label{eq:nilpotency}
\end{align}
Now, consider any arbitrary element $x^\alpha \in C_p$
such that $\gcd{\alpha}{p} = g > 1$. Then, we say $\alpha$ is not coprime to $p$.
Define the \emph{least common multiple} of
$p$ and $\alpha$ to be
\[
  \lcm{p}{\alpha} = \frac{\alpha p}{g}
\]
Since $g \mid \alpha$, then $\lcm{p}{\alpha} = np$ for some $n \in \Z$.
Therefore, $x^{\lcm{p}{\alpha}} = x^{np} = 0$ (by equation~\ref{eq:nilpotency}).
This means that, whenever $(a, p) \ne 1$, then $x^{\lcm{p}{a}} = 0$,
therefore $\psi_a$ is not an automorphism (since it's kernel is not trivial,
it's image does not equal $C_p$).

Consequently, $\psi_a \in \Aut(C_p) \Leftrightarrow (a, p) = 1$.
We may also recognize that such elements $\psi_a \in \Aut(C_p)$
correspond to the units $a \in {(\Z/p\Z)}^\times$,
and $\abs{\Aut(C_p)} = \abs{{(\Z/p\Z)}^\times} = \phi(p)$.
Therefore, $\Aut(C_p) \simeq {(\Z/p\Z)}^\times$.
When $p$ is prime, then $\phi(p) = p-1$,
in which case
\begin{align}
  \Aut(C_p) \simeq (\Z/p\Z)^\times \simeq C_{p-1}~\label{eqn:aut_p}
\end{align}
and $\Aut(C_p)$ is cyclic.

Consider the case that $\#G = pq$ having $p \nmid (q-1)$.
Then, $G$ is abelian (by DF Example 4.4.1, p.~135--136).
Consider the center $Z(G) \le G$. If $Z(G) \ne 1$, then Lagrange's theorem (\ref{thm:lagrange})
forces $G/Z(G)$ to be cyclic.
Furthermore, Lagrange's theorem tells us that the order of any element in $G$ must divide the
order of $G$, therefore the order may be $n \in \{1, p, q, pq \}$.
Suppose $Z(G) = 1$. Then;
\begin{enumalph}
  \item Nonidentity elements may not have order $1$.
  \item If every nonidentity element of $G$ has order $p$,
    then the centralizer of every nonidentity
    element has index $q$, so the class equation reads \[pq = 1 + kq \]
    This is contradictory, since $q \mid pq$ but $q \nmid (1 + kq)$
    (because $q$ does not divide $1$).
    Therefore, if $G$ contains an element $x$ of order $q$.
  \item Let $H = \langle x \rangle$ be the subgroup generated by $x$.
    Since $H$ has index $p$ in $G$ and $p$ is the smallest prime dividing $\#G = pq$,
    $H$ is a normal subgroup in $G$ by Corollary 5~\cite[p. 120]{DummitFoote}.
  \item Since $Z(G) = 1$, then $C_G(H) = 1 \cdot H \cdot 1 = H$. Therefore, the
    quotient group $G/H = N_G(H)/C_G(H)$ has order $p$ and is isomorphic to a group of
    $\Aut(H)$ by Corollary 15 \cite[p. 134]{DummitFoote}.
  \item By Proposition 16 \cite[p. 135]{DummitFoote}, $\Aut(H) \simeq C_{q-1}$.
    and has order $q-1$, which implies that $p \mid (q-1)$ by Lagrange's theorem (\ref{thm:lagrange}),
    a contradiction of the assumption that $p \nmid (q-1)$. Therefore, $G$ must be abelian
    .~\label{eqn:p_not_div_q-1}

\end{enumalph}

Now, suppose $p \mid (q-1)$.

\begin{theorem}[Sylow's Theorem]~\label{thm:sylow}
  Let $G$ be a finite group of order $p^{\alpha}m$ where $p$ is a prime not dividing $m$.
  \begin{enumerate}[label=\arabic{enumi}.]
    \item Sylow $p$-subgroups of $G$ exist, i.e. $n_p \ne \varnothing$.
    \item If $P$ is a Sylow $p$-subgroup of $G$ and $Q$ is any $p$-subgroup of $G$,
      then there exists $g \in G$ such that $Q = gPg^{-1}$, i.e. $Q$ is conjugate to $P$.
    \item The number of Sylow $p$-subgroups of $G$ is of the form
      $n_p = 1 + kp$, i.e. \[ n_p \equiv 1 \pmod p. \]
      Further, $n-p$ is the index of $N_G(P)$ in $G$ for any Sylow $p$-subgroup $P$,
      hence \[ n_p \mid m. \]~see~\cite[Theorem 18,~p.~139]{DummitFoote}.\qed
  \end{enumerate}
  
\end{theorem}

Let $P \le G$ be a Sylow $p$-subgroup of $G$
and $Q \le G$ be a Sylow $q$-subgroup of $G$.
Sylow's theorem (see \ref{thm:sylow}) tells us that
$n_q = 1 + kq$ for some $k \ge 0$ and $n_q \mid p$.
Since we have $p < q$, it must be that $k = 0$ and $n_q = 1$.
By Corollary 20~\cite[p. 142]{DummitFoote}, $Q$ is normal in $G$.


Since $P$ and $Q$ are of prime order, they are cyclic and are each generated by a single element.
Let $P = \langle p \rangle$ and $Q = \langle q \rangle$.
Note that $\Aut(Q) \simeq C_{q-1}$ is cyclic and $p \mid (q-1)$,
$Q$ contains a unique \emph{cyclic} subgroup of order $p$, say $\langle \gamma \rangle$,
and any homomorphism $\psi : P \to \Aut(Q)$ must map $y$ to a power of $\gamma$.
Since $\abs{\gamma} = p$, there are, therefore, $p$ distinct homomorphisms
$\psi_i : P \to \Aut(Q)$ given by $\psi(y) = \gamma^i, 0 \le i \le p-1$.
There are two general cases:
\begin{enumalph}
  \item If $i = 0$, then $\psi_i$ is the trivial homomorphism
    and $Q \rtimes_{\psi_0} P \simeq Q \times P$. This is an abelian
    group isomorphic to to $C_q \times C_p$.
  \item If $i \ne 0$, then $\psi_i$ is nontrivial and $Q \rtimes_{\psi_i} P$
    is a nonabelian group of order $p^q$. We may also note that all these groups
    are isomorphic because for each $\psi_i, i \ne 0$, there is some generator
    element $y_i \in P$ such that $\psi_i(y_i) = \gamma$.~\cite[p. 181]{DummitFoote}
    .~\label{pro:nonabelian}
\end{enumalph}

\begin{Answer}
  \begin{enumerate}
  \item Classify groups of order $n \leq 15$ except $n=12$.
  \end{enumerate}
\end{Answer}

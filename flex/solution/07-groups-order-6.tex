\section{Groups of Order 6}~\label{sec:groups-order-6}

\begin{Answer}
  \begin{enumalph}
    \item State Cauchy's theorem and the fundamental theorem of finite abelian groups 
      (giving references, but without proofs; if these are of interest, consider one of 
      the other projects!).  
    \item Recall the proof why every group of order $p^2$ with $p$ prime is abelian.  
    \item Classify the \emph{abelian} groups of order $n \leq 15$ up to isomorphism 
    using the fundamental theorem.
    \item Classify groups of order $6$ by hand: using Cauchy's theorem, there exists $a
    \in G$ of order $2$ and $b \in G$ of order $3$;
      show that $G=\{1,b,b^2,a,ab,ab^2\}$, in a direct manner that
      $ba=ab$ or $ba=ab^2$, and show that these two possibilities uniquely determine
      the Cayley table of $G$.
  \end{enumalph}
\end{Answer}

Let $G$ be a group of order $6$.
Cauchy's theorem (see~\ref{thm:cauchy}) tells us that $G$ must have
an element of order $2$ and an element of order $3$.
Let $a \in$ be an element of order $2$ and $b \in G$ be an element of order $3$,
so $a^2 = b^3 = \epsilon$.
It is trivial that the subgroups $\langle a \rangle$ and $\langle b \rangle$
must intersect trivially:
\begin{enumalph}
  \item If $a = b$ then $a^2 = b^2$, but $a^2 = \epsilon$, which contradicts that $b$ has order $3$.
  \item If $a = b^2$ then $a^2 = b^4 = b$, which contradicts that $a^2 = \epsilon$.
\end{enumalph}
Consider the group $G = \langle x, y \rangle  \{ \epsilon, b, b^2, a, ab, ab^2 \}$.
\begin{enumalph}
  \item Since $a$ has order $2$, $a^2 = \epsilon$ so $a^{-1} = a$.
  \item Since $b$ has order $3$, $b^3 = \epsilon$, so $b^{-1} = b^2$.
  \item Therefore, either ${ab}^{-1} = ab$ or ${ab}^{-1} = ab^2$
\end{enumalph}

Suppose ${ab}^{-1} = ab$.
Since ${(xy)}^{-1} = y^{-1}x^{-1}$ (generally), this implies that
$ab = b^2a$ and $ab^2 = ba$.
Therefore, the Cayley table of $G$ is determined as:

\begin{center}
  \begin{table}[H]~\label{tab:cayley-table-order-6=a}
    \begin{tabular}{|c|c|c|c|c|c|c|}
      \hline
      & $1$ & $b$ & $b^2$ & $a$ & $ab$ & $ab^2$ \\
      \midrule
      $1$ & $1$ & $b$ & $b^2$ & $a$ & $ab$ & $ab^2$ \\
      $b$ & $b$ & $b^2$ & $1$ & $ab$ & $ab^2$ & $a$ \\
      $b^2$ & $b^2$ & $1$ & $b$ & $ab^2$ & $a$ & $ab$ \\
      $a$ & $a$ & $ab$ & $ab^2$ & $1$ & $b$ & $b^2$ \\
      $ab$ & $ab$ & $ab^2$ & $a$ & $b^2$ & $1$ & $b$ \\
      $ab^2$ & $ab^2$ & $a$ & $ab$ & $b$ & $b^2$ & $1$ \\
      \bottomrule
    \end{tabular}
    \caption{Cayley table of $G$ when $ba = ab^2$}
  \end{table}
\end{center}

The group is not symmetric across the diagonal, therefore not abelian.
We may recognize this group as either $D_6$ or $S_3$,
under the isomorphism:
\begin{multicols}{2}
  \textbf{Isomorphism to} $D_6$
  \begin{align*}
    \psi \colon D_6 &\to G \\
    \psi(1) &\mapsto 1 \\
    \psi(r) &\mapsto b \\
    \psi(s62) &\mapsto b^2 \\
    \psi(s) &\mapsto a \\
    \psi(sr) &\mapsto ab \\
    \psi(sr^2) &\mapsto ab^2 \\
  \end{align*}
  \columnbreak

  \textbf{Isomorphism to} $S_3$
  \begin{align*} \\
    \phi \colon S_3 &\to G \\
    \phi(1) &\mapsto 1 \\
    \phi((1\ 2\ 3)) &\mapsto b \\
    \phi((1\ 3\ 2)) &\mapsto b^2 \\
    \phi((1\ 2)) &\mapsto a \\
    \phi((2\ 3)) &\mapsto ab \\
    \phi((1\ 3)) &\mapsto ab^2 \\
  \end{align*}
\end{multicols}
In the second case, suppose ${ab}^{-1} = ab^2$.
Since ${(xy)}^{-1} = y^{-1}x^{-1}$ (generally), this implies that
$ab = ba$ and $ab^2 = b^2a$.
Therefore, the Cayley table of $G$ is determined as:

\begin{center}
  \begin{table}[H]~\label{tab:cayley-table-order-6=b}
    \begin{tabular}{|c|c|c|c|c|c|c|}
      \hline
      & $1$ & $b$ & $b^2$ & $a$ & $ab$ & $ab^2$ \\
      \midrule
      $1$ & $1$ & $b$ & $b^2$ & $a$ & $ab$ & $ab^2$ \\
      $b$ & $b$ & $b^2$ & $1$ & $ab$ & $ab^2$ & $a$ \\
      $b^2$ & $b^2$ & $1$ & $b$ & $ab^2$ & $a$ & $ab$ \\
      $a$ & $a$ & $ab$ & $ab^2$ & $1$ & $b$ & $b^2$ \\
      $ab$ & $ab$ & $ab^2$ & $a$ & $b$ & $b^2$ & $1$ \\
      $ab^2$ & $ab^2$ & $a$ & $ab$ & $b^2$ & $1$ & $b$ \\
      \bottomrule
    \end{tabular}
  \end{table}
\end{center}

Notice the symmetry across the diagonal, implying that $G$ is abelian.
Indeed, in this case $G$ is isomorphic to $C_6$,
with generator $1 \sim ab$
as the image of $1 \in C_6$:
\begin{align*}
  \psi \colon C_6 &\to G \\
  \psi(0) &\mapsto (1, 1) \sim 1 \\
  \psi(1) &\mapsto (a, b) \sim ab \\
  \psi(2) &\mapsto (1, b^2) \sim b^2 \\
  \psi(3) &\mapsto (a, 1) \sim a \\
  \psi(4) &\mapsto (1, b) \sim b \\
  \psi(5) &\mapsto (a, b^2) \sim ab^2 \\
\end{align*}

Therefore, the groups of order $6$ are, up to isomorphism, the groups $C_6$ and $S_3$.

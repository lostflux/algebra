% ENCODING
\usepackage[utf8]{inputenc}
\usepackage{pmboxdraw}
% \usepackage{pmboxdraw-extras}

% text alignment
\usepackage{array, ragged2e}

% URLs
\usepackage{hyperref}

% Sets
\newcommand{\FF}{\mathbb{F}}
\newcommand{\NN}{\mathbb{N}}
\newcommand{\QQ}{\mathbb{Q}}
\newcommand{\RR}{\mathbb{R}}
\newcommand{\ZZ}{\mathbb{Z}}

\newcommand{\F}{\mathbb{F}}
\newcommand{\N}{\mathbb{N}}
\newcommand{\Q}{\mathbb{Q}}
\newcommand{\R}{\mathbb{R}}
\newcommand{\Z}{\mathbb{Z}}
\newcommand{\C}{\mathbb{C}}
\renewcommand{\b}{\{0,1\}}

% More Math operators
\newcommand{\GL}{\mathrm{GL}}
\newcommand{\calC}{\mathcal{C}}
\newcommand{\calP}{\mathcal{P}}
\newcommand{\calF}{\mathcal{F}}
\newcommand{\calO}{\mathcal{O}}
\newcommand{\calS}{\mathcal{S}}
\newcommand{\calT}{\mathcal{T}}
\newcommand{\calU}{\mathcal{U}}
\newcommand{\calV}{\mathcal{V}}
\newcommand{\nequiv}{\not\equiv}
\renewcommand{\notin}{\not\in}

% Functions
\usepackage{amsmath}
\DeclareMathOperator{\cost}{cost}
\DeclareMathOperator{\len}{len}
\DeclareMathOperator{\rank}{rank}
\DeclareMathOperator{\sgn}{sgn}
\DeclareMathOperator{\wt}{wt}

% Algebra
% \newcommand{\eps}{\epsilon}
% \DeclareMathOperator{\GL}{GL}
\DeclareMathOperator{\id}{id}
% \DeclareMathOperator{\lcm}{lcm}
\DeclareMathOperator{\M}{M}
\DeclareMathOperator{\SL}{SL}
\DeclareMathOperator{\Syl}{Syl}
\DeclareMathOperator{\tr}{tr}

% Combinatorics

% Probability
\newcommand{\Hd}{\texttt{\color{BrickRed}H}}
\newcommand{\Tl}{\texttt{T}}
\newcommand{\EE}{\mathop{\mathbb{E}}}
\newcommand{\PP}{\mathop{\mathbb{P}}}
\newcommand{\indic}{\mathbb{1}}
\DeclareMathOperator{\Var}{Var}

% End of proof marker
\newcommand{\qedblack}{\hfill\ensuremath{\blacksquare}}
\newcommand{\qedwhite}{\hfill\ensuremath{\square}}

% Caligraphic caps
\newcommand{\cA}{\mathcal{A}}
\newcommand{\cB}{\mathcal{B}}
\newcommand{\cC}{\mathcal{C}}
\newcommand{\cD}{\mathcal{D}}
\newcommand{\cE}{\mathcal{E}}
\newcommand{\cF}{\mathcal{F}}
\newcommand{\cI}{\mathcal{I}}
\newcommand{\cL}{\mathcal{L}}
\newcommand{\cP}{\mathcal{P}}
\newcommand{\cS}{\mathcal{S}}
\newcommand{\cT}{\mathcal{T}}
\newcommand{\cU}{\mathcal{U}}
\newcommand{\cX}{\mathcal{X}}
\newcommand{\cY}{\mathcal{Y}}
\newcommand{\cZ}{\mathcal{Z}}

% Boldface letters
\newcommand{\ba}{\mathbf{a}}
\newcommand{\bb}{\mathbf{b}}
\newcommand{\bp}{\mathbf{p}}
\newcommand{\bq}{\mathbf{q}}
\newcommand{\br}{\mathbf{r}}
\newcommand{\bu}{\mathbf{u}}
\newcommand{\bv}{\mathbf{v}}
\newcommand{\bx}{\mathbf{x}}
\newcommand{\by}{\mathbf{y}}
\newcommand{\bz}{\mathbf{z}}
\newcommand{\tbp}{\mathbf{\widetilde{p}}}

% Special math symbols
\newcommand{\eps}{\varepsilon}
\newcommand{\ceq}{\subseteq}
\newcommand{\ang}[1]{\langle{} #1 \rangle}
\newcommand{\ceil}[1]{\lceil{} #1 \rceil}
\newcommand{\floor}[1]{\lfloor{} #1 \rfloor}

% Problem names and other small-caps constants
\newcommand{\inv}{\textsc{inv}\xspace}

% Useful for marking steps of a derivation to explain later
\newcommand{\circled}[1]{\raisebox{.5pt}{\textcircled{\raisebox{-.1pt}{\scriptsize #1}}}}


% Page size and margins
% \usepackage[left=1in,right=1in,top=1.3in,bottom=1.3in,nofoot]{geometry}
\usepackage{fancyhdr}   % for fancy header
\usepackage{fancyvrb}   % for fancy verbatim
\usepackage{graphicx}   % for including images
\usepackage{enumerate}  % for enumerating lists
\usepackage{enumitem}
\usepackage[rgb, dvipsnames]{xcolor}
\usepackage{tcolorbox}
% \newtcolorbox{codebox}[1]{
%   box align=top,
%   colback=white!5!white,
%   colframe=white!75!black,
%   title=#1
% }
\usepackage{multicol}
% \usepackage{accode}

% My Problem set macros

% Answer BOX
\usepackage{microtype}
\usepackage{mdframed}
\newmdenv[%
  skipabove=6pt,
  skipbelow=6pt,
  innertopmargin=6pt,
  leftmargin=-5pt,
  rightmargin=-5pt, 
  innerleftmargin=5pt,
  innerrightmargin=5pt,
  backgroundcolor=black!10
]{Answer}%


% Header BOX
\newcommand{\handout}[6]{
  \noindent
  \begin{center}
  \setlength{\fboxrule}{1.2pt}
  \framebox{
    \vbox{
      \hbox to 5.78in { \textbf{#6} \hfill {\bf #2} }
      \vspace{4mm}
      \hbox to 5.78in { {\Large \hfill {\textbf{ #5 }}  \hfill} }
      \vspace{2mm}
      \hbox to 5.78in { {\textit{\textbf{#3 \hfill #4}}} }
    }
  }
  \setlength{\fboxrule}{0.2pt}
  \end{center}
  \vspace*{4mm}
}

% Header BOX
\newcommand{\PSET}[5]{\handout{#1}{#2}{Prof.\ #3}{Student: #4}{PSET #1}{#5}}

% Credit Statement
\newcommand{\CreditStatement}[1]{
  \noindent
  \begin{center} {
    \bf Credit Statement
  }
  \end{center}
  { #1 }
}

% Problem counter
\newenvironment{problem}[1][]%
{%
\stepcounter{problem} \vspace{.2cm} \noindent {\bf \arabic{problem}.} {\textit{#1}}~%
}{%
\vspace{.2cm}%
}

% Package Imports
\usepackage{amssymb,amsthm,amsmath,amstext}
\usepackage{mathdots} % for \dots
  % \dotsc -- dots with commas.
  % \dotsb -- dots with binary operators.
  % \dotsm -- multiplication dots.
  % \dotsi -- dots with integrals.
  % \dotso -- "other dots".
\usepackage{wasysym, stackengine, makebox, tikz-cd}
\newcommand\isom{\mathrel{\stackon[-0.1ex]{\makebox*{\scalebox{1.08}{\AC}}{=\hfill\llap{=}}}{{\AC}}}}
\newcommand\nvisom{\rotatebox[origin=cc] {-90}{$ \isom $}}
\newcommand\visom{\rotatebox[origin=cc] {90} {$ \isom $}}


% Custom colors
\definecolor{crimson}{rgb}{0.86, 0.08, 0.24}
\definecolor{teal}{rgb}{0.0, 0.5, 0.5}
\definecolor{zaffre}{rgb}{0.0, 0.08, 0.66}
\definecolor{DarkOliveGreen}{rgb}{0.33, 0.42, 0.18}
\newcommand{\crim}{\textcolor{crimson}}
\newcommand{\teal}{\textcolor{teal}}
\newcommand{\zaff}{\textcolor{zaffre}}
\newcommand{\black}{\textcolor{black}}
\newcommand{\darkgreen}{\textcolor{DarkOliveGreen}}
\newcommand{\green}{\textcolor{OliveGreen}}

% \newcommand{\id}{\mathbf{id}\;}

% matrices -- vertical separators
\makeatletter
\renewcommand*\env@matrix[1][*\c@MaxMatrixCols{ c}]{%
  \hskip -\arraycolsep{}
  \let\@ifnextchar\new@ifnextchar{}
  \array{#1}}
\makeatother

% \usepackage{accode}
\usepackage{tikz}

% long multiplications
\usepackage{xlop}

% custom functions.
\renewcommand{\gcd}[2]{\mathbf{gcd}\;(#1,\;#2)}
\newcommand{\lcm}[2]{\mathbf{lcm}\;(#1,\;#2)}
\newcommand{\Therefore}{\dot{.\hspace{.095in}.}\hspace{.095in}}
\newcommand{\However}{\dot{}\hspace{.045in}.\hspace{.045in} \dot{}\hspace{.095in}}

% resume includes
\usepackage[utf8]{inputenc}
\usepackage[full]{textcomp}
\usepackage{CJKutf8}
\usepackage[lf]{ebgaramond}

\usepackage[OT1]{fontenc}
\usepackage{enumitem}
\usepackage[scale=.75]{geometry}
\usepackage{url}

\pagestyle{headings}

\setlength\parindent{2em}

\thispagestyle{empty}

\newcommand{\cvsubsection}[1]{\subsection*{\hspace{1.45em}#1}}


% enumalph
\newenvironment{enumalph}{
  \begin{enumerate}
  \renewcommand{\labelenumi}{
    \textnormal{(\alph{enumi})}
  }
}{\end{enumerate}}

% enumroman
\newenvironment{enumroman}{
  \begin{enumerate}
  \renewcommand{\labelenumi}{
    \textnormal{(\roman{enumi})}
  }
}{\end{enumerate}}

\newenvironment{problab}[1]
{\noindent\textbf{Problem #1}.}
{\vskip 6pt}
\theoremstyle{remark}
\newtheorem*{solu}{Solution}

% import := definition
\usepackage{colonequals}

\usepackage{listings}
\usepackage{color}

% -- Defining colors:
\usepackage[dvipsnames]{xcolor}
\definecolor{codegreen}{rgb}{0,0.6,0}
\definecolor{codegray}{rgb}{0.5,0.5,0.5}
\definecolor{codepurple}{rgb}{0.58,0,0.82}
\definecolor{backcolour}{rgb}{0.95,0.95,0.92}
\definecolor{dkgreen}{rgb}{0,0.6,0}
\definecolor{gray}{rgb}{0.5,0.5,0.5}
\definecolor{mauve}{rgb}{0.58,0,0.82}

\lstset{frame=tb,
  backgroundcolor=\color{backcolour},   
  commentstyle=\color{codepurple},
  keywordstyle=\color{NavyBlue},
  numberstyle=\tiny\color{codegray},
  stringstyle=\color{codepurple},
  basicstyle=\ttfamily\footnotesize\bfseries,
  breakatwhitespace=false,         
  breaklines=true,                 
  captionpos=t,                    
  keepspaces=true,                 
  numbers=left,                    
  numbersep=5pt,                  
  showspaces=false,                
  showstringspaces=false,
  showtabs=false,                  
  tabsize=2,
  % escapeinside={\%*}{*)},          % if you want to add LaTeX within your code
}


\definecolor{mygreen}{rgb}{0,0.6,0}
\definecolor{mygray}{rgb}{0.5,0.5,0.5}
\definecolor{mymauve}{rgb}{0.58,0,0.82}

\usepackage{booktabs}


\DeclareMathOperator{\Aut}{Aut}
\DeclareMathOperator{\opspan}{span}
\DeclareMathOperator{\Tr}{Tr}
\DeclareMathOperator{\Frac}{Frac}
\DeclareMathOperator{\ord}{ord}
\DeclareMathOperator{\Sym}{Sym}

\numberwithin{equation}{section}
\newtheorem{theorem}[equation]{Theorem}
\newtheorem{thm}[equation]{Theorem}
\newtheorem{lemma}[equation]{Lemma}
\newtheorem{lem}[equation]{Lemma}
\newtheorem{proposition}[equation]{Proposition}
\newtheorem{prop}[equation]{Proposition}
\newtheorem{corollary}[equation]{Corollary}
\newtheorem{cor}[equation]{Corollary}

\theoremstyle{definition}
\newtheorem{definition}[equation]{Definition}
\newtheorem{defn}[equation]{Definition}
\newtheorem{example}[equation]{Example}
\newtheorem{xca}[equation]{Exercise}
\newtheorem{notation}[equation]{Notation}
\theoremstyle{remark}
\newtheorem{remark}[equation]{Remark}

\numberwithin{equation}{section}


\usepackage[normalem]{ulem}
\usepackage{fullpage}
\usepackage{colonequals}
\usepackage{amssymb}
\usepackage{amsthm}
\usepackage{amsmath}
\usepackage{amsxtra}
\usepackage{mathtools}
\usepackage{mathrsfs}

\usepackage{hyperref}
\hypersetup{colorlinks=true,urlcolor=blue,citecolor=blue,linkcolor=blue}

\numberwithin{equation}{section}

\usepackage{amssymb}
\usepackage{amsthm}
\usepackage{amsmath}
\usepackage{amsxtra}

\setlength{\hfuzz}{4pt}

\DeclarePairedDelimiter{\abs}{\lvert}{\rvert}

\newcommand{\defi}[1]{\textsf{#1}} % for defined terms

\renewcommand{\baselinestretch}{1.5} 

\usepackage{titling}
\usepackage[english]{babel}
\usepackage[utf8]{inputenc}
\usepackage{amsmath, amsfonts, amsthm}
\usepackage{graphicx}
\usepackage[colorinlistoftodos]{todonotes}
\usepackage{subfig}
% \usepackage{mdframed} 
\usepackage{color}
\usepackage{tabu}
\usepackage{tikz}
\usepackage{enumerate}
\usepackage{multicol}
\usepackage{pgfplots}
\usepackage{csquotes}
\pgfplotsset{compat=1.18}
\usepackage[style=iso]{datetime2}
\usepackage{multirow}
